\documentclass[landscape,footrule]{foils}
\usepackage[lecture-serie]{foiltex-extra}
\usepackage{crysymb}
\usepackage[draft]{crygame}
\usepackage{crypto-ii}
\usepackage{graphics}
\usepackage[pdftex]{graphicx} 


\newcommand{\lecture}{How to Model Cryptographic\\  Primitives and Protocols}
\newcommand{\lserie}{MTAT.07.003 Cryptology II}
\newcommand{\ldate}{\today}
\newcommand{\lauthor}{Sven Laur}
\newcommand{\linst}{University of Tartu}
\graphicspath{{./illustrations/}}
\MyLogo{\lserie,\  How to Model Cryptographic  Primitives and Protocols, \ldate}


\newcommand{\leqm}{\ \leq_m}


\newcommand{\bigvskip}{\vskip 2em}
\newcommand{\lastline}{\vspace*{-2ex}}
\newcommand{\spreadappart}{\vspace*{\fill}}


\begin{document}
\titlefoil

\foilhead[-1cm]{Abstraction is a key to successs}

\begin{triangles}
  \item \textbf{Cryptographic constructions are complex}\vspace*{1ex}
    \begin{diamonds}
      \item Irrelevant techincal details obscure security proofs.
      \item A good abstraction clarifies what is meant by security. 
      \item An abstraction highlights which properties are relevant for security.
      \vspace*{3ex}
    \end{diamonds}
  \item \textbf{Cryptographic constructions are not provably secure}\vspace*{1ex}
    \begin{diamonds}
      \item Security of most cryptographic constructions is based on \emph{intractability}.
      \item So far provable lower bounds are \emph{trivial} for all computational problems. 
      \item It is also \emph{highly} unlikely that such proofs \emph{do} exist in a \emph{compact} form. 
      \vspace*{3ex}
    \end{diamonds}
  \item\textbf{Abstraction allows to escape intractability issues}
    \begin{diamonds}
      \item We just assume that necessary cryptographic primitives exist.
      \item The actual implementation of such primitives is out of our scope.  
    \end{diamonds}
\end{triangles}

\middlefoil{Illustrative Example}

\foilhead[-1cm]{2048-bit RSA}


\textbf{Key generation} 
\begin{enumerate}
\item Choose two 1024-bit prime numbers $p$ and $q$.
\item Compute Let $n=pq$, choose $e\getsu\ZZ_{\phi(n)}^*$ and set $d\gets e^{-1}\mod \phi(n)$.
\item Public key is $(n,e)$ and secret key is $(n,e,d)$.   
\end{enumerate}
\bigskip

\textbf{Encryption}
\begin{enumerate}
\item Let  $\mathsf{pad}:\set{0,1}^{128}\to\ZZ_n^*$ be a  predefined embedding.
\item To encrypt $m\in\set{0,1}^{128}$, output $c\gets\mathsf{pad}(m)^e \mod n$. 
\end{enumerate}
\bigskip

\textbf{Decryption}
\begin{enumerate}
\item To decrypt $c\in\ZZ_n$, compute $x\gets c^d\mod n$.
\item Extract $m$ form $x$ and verify that $\mathsf{pad}(m)=x$.
\item Output $\bot$ in case of failure and $m$ otherwise.
\end{enumerate}

\foilhead[-1cm]{The corresponding abstraction}


\illustration[scale=1.1, angle=0, trim=0cm 0cm 0cm 0cm]{rsa-abstraction}

\textbf{To get rid of unnecessary details}
\begin{triangles}
\item We split the system into algorithms and treat them as black
  boxes.
\item Functionality is guaranteed by specifying additional conditions.
\item Security is defined through  specifications of tolerable attack scenarios.
\end{triangles}

\foilhead[-1cm]{Naive security requirement}

\textbf{Goal:}
It should be infeasible to derive a secret key from accessible data.
\bigskip

\illustration[scale=1.1, angle=0, trim=0cm 0cm 0cm 0cm]{key-recovery-game-i.eps}
\bigskip

The \emph{advantage} of a \emph{key only attack} is defined as an \emph{average} success:
\begin{align*}
  \ADV(\AD)=\pr{\GAME^\AD=1}\enspace.
\end{align*}
\textbf{Caveat:}The attack scenario does not capture the security goal in real life.


\foilhead[-1cm]{Seemingly more advanced attack scenario}

\illustration[scale=1.1, angle=0, trim=0cm 0cm 0cm -0.5cm]{key-recovery-game-ii.eps}


\textbf{Caveat:}The attack scenario is not more powerful than the previous.
\begin{triangles}
  \item The adversary $\AD$ knows what is inside $(\GEN,\ENC,\DEC)$ blocks. 
  \item As adversary knows $\PK$,  she can compute $\ENC_\PK(m)$ by herself.  
  \item The oracle access to $\ENC_\PK(\cdot)$ function is redundant.
\end{triangles}

\foilhead[-1cm]{Classical chosen-ciphertext attack scenario}

\illustration[scale=1.1, angle=0, trim=0cm 0cm 0cm -0.7cm]{key-recovery-game-iii.eps}

\textbf{The difference:}
The attacker has an implicit access to secret key. 
\begin{triangles}
  \item Decryption operation can leak information about secret key.
  \item This can happen only for the messages not computed by $\ENC_\PK(\cdot)$.
  \item Such attacks are sometimes plausible in real life.
\end{triangles}


\foilhead[-1cm]{Time-success profiles}

Fix the security game and the advantage function $\ADV(\cdot)$.  Then any
concrete instantiation of a primitive can be broken with enough
resources.

\illustration[scale=0.92, angle=0, trim=0cm 0cm 0cm -0.7cm]{time-success-profile.eps}

As a result, there exist a time-success profile
$\varepsilon=\varepsilon(t)$, which captures the main security
properties. Unfortunately, this profile cannot be computed nor
approximated with our current knowledge.\lastline

\middlefoil{Examples of Low-level Primitives}


\foilhead[-1cm]{Discrete logarithm}

\begin{triangles}
\item Let $p$ be a prime such that $p= 2q+1$ for another
  $2048$-bit prime $q$.
\item Fix a generator $g$ such that $g^2\neq 1$ and define
  $\GG=\set{g^i: 0\leq i < q}$.
\item Then discrete logarithm defined below  is considered intractable
  \begin{align*}
   \forall y\in\GG:\ \log(y)=x \Leftrightarrow g^x\equiv y\pmod p\enspace.
  \end{align*}
\end{triangles}
\bigvskip

\textbf{Exercise.} Abstract away all details under the assumptions:
\begin{triangles}
  \item all construction based on it use only multiplication modulo $p$,
  \item strings are mapped to $\GG$ and elements of $\GG$ are mapped to strings.
\end{triangles}
\bigskip

How to model the primitive  if  constructions also use addition modulo $p$?

\foilhead[-1cm]{Discrete logarithm problem in an abstract group}

\illustration[scale=0.92, angle=0, trim=0cm 0cm 0cm -0.7cm]{abstract-group.eps}

\textbf{Definition.} Let $\GG=\langle g\rangle$ be a $q$-element
multiplicative group generated by the element $g$. Then for any
elements $y,z\in\GG$ the discrete logarithm $\log_z y$ is defined as
the smallest integer $x$ such that $z^x=y$ and $\bot$ if
$y\notin\langle z\rangle$.  \bigskip

\textbf{Advantage.} Let $\advDLXX{\GG}{\AD}=\pr{\GAME^\AD=1}$ be defined through the game
\begin{small}
\begin{align*}
  \begin{game}{\GAME^\AD}
   & x\getsu\ZZ_q\\
   & \RETURN [x \iseq \AD(g,g^x)]
  \end{game}
\end{align*}%
\end{small}

\foilhead[-1cm]{Discrete logarithm problem in an abstract group}

\textbf{Definition.} Let $\GG=\langle g\rangle$ be a $q$-element
multiplicative group generated by the element $g$. Then for any
elements $y,z\in\GG$ the discrete logarithm $\log_z y$ is defined as
the smallest integer $x$ such that $z^x=y$ and $\bot$ if
$y\notin\langle z\rangle$.  \bigskip

\textbf{Advantage.} Let $\advDLXX{\GG}{\AD}=\pr{\GAME^\AD=1}$ be defined through the game
\begin{small}
\begin{align*}
  \begin{game}{\GAME^\AD}
   & x\getsu\ZZ_q\\
   & \RETURN [x \iseq \AD(g,g^x)]
  \end{game}
\end{align*}%
\end{small}

\textbf{Security.} A group $\GG$ is \emph{$(t,\varepsilon)$-secure}
DL-group iff for any $t$-time adversary $\AD$ the corresponding
advantage $\advDLXX{\GG}{\AD}\leq\varepsilon$.


\foilhead[-1cm]{Diffie-Hellman protocol}

\illustration[scale=1.1, angle=0, trim=0cm 0cm 0cm -0.7cm]{diffie-hellman-protocol.eps}

\textbf{Exercise.} Formalise the security requirements for Diffie-Hellman protocol.

\begin{enumerate}
\item Eavesdropper cannot reconstruct the common secret $g^{xy}$.
\item Eavesdropper learns nothing about the common secret $g^{xy}$.
\end{enumerate}
\bigskip

How to convert the common secret $g^{xy}$ to a valid secret key  $\SK\in\set{0,1}^n$? 
\lastline

\foilhead[-1cm]{Computational Diffie-Hellman problem} 

\textbf{Security.} A group $\GG$ is \emph{$(t,\varepsilon)$-secure}
CDH-group iff for any $t$-time adversary $\AD$ the corresponding
advantage $\advCDHXX{\GG}{\AD}\leq\varepsilon$ where the corresponding
security game is defined as follows.


\illustration[scale=1.1, angle=0, trim=0cm 0cm 0cm -1cm]{cdh-problem.eps}



\foilhead[-1cm]{Decisional Diffie-Hellman}

\textbf{Security.} A group $\GG$ is \emph{$(t,\varepsilon)$-secure}
DDH-group iff for any $t$-time adversary $\AD$ the corresponding
advantage $\advDDHXX{\GG}{\AD}\leq\varepsilon$ where the corresponding
security games $\GAME_0$ and $\GAME_1$ and the advantage are defined as follows.

\illustration[scale=1.1, angle=0, trim=0cm 1.5cm 0cm -1cm]{ddh-problem.eps}

\begin{align*}
  \advDDHXX{\GG}{\AD}=\abs{\pr{\GAME_0^\AD=1}-\pr{\GAME_1^\AD=1}}
\end{align*}

\foilhead[-1cm]{Factorisation}

Factorisation of $n$-bit composite numbers is considered difficult
\begin{triangles}
  \item Naive factorisation takes $\Theta(2^{\frac{n}{2}})$ division operations.
  \item Pollard $\rho$ algorithm takes $\Oh(2^{\frac{n}{4}})$ multiplication operations on average.
  \item Quadratic sieve  takes $\Oh(2^{c\sqrt{n}})$ multiplication operations on average.
  \item Number field sieve  takes $\Oh(2^{c\sqrt[3]{n}})$ multiplication operations on average.
\end{triangles}
\medskip
\medskip

\textbf{Current records}
\begin{triangles}
  \item Largest RSA challenge factored had $768$ bits (2009).
  \item Largest number factored using quantum annealing is 4,088,459 (2018). 
  \item Largest partially factored Mersenne number has $5,240,707$ bits (2016).
  \item Approximate running-times are in thousands of computer years.
  \item Shor's algoritm failed to factor $35$ on \textsc{ibmqx5} quantum computer (2019).
\end{triangles}\vspace*{-5ex}

\foilhead[-1.5cm]{Abstract distribution of RSA moduli}

\textbf{Definition}. A \emph{distribution of RSA moduli} $\mathfrak{N}$ is
defined by an efficient algorithm $\GEN$ that outputs $n,p,q$ such
that $n=pq$ and $p,q$ are primes.
\bigskip

\textbf{Security.} A distribution $\mathfrak{N}$ is
\emph{$(t,\varepsilon)$-secure} RSA-distribution iff for any $t$-time
adversary $\AD$ the corresponding advantage
$\advRSAXX{\GG}{\AD}\leq\varepsilon$ where the security game is
defined as follows
 
\illustration[scale=0.9, angle=0, trim=0cm 0cm 0cm -1cm]{rsa-factorisation-problem.eps}

\textbf{Example.}  Let $\mathfrak{P}$ be an efficiently samplable set
of primes. Then the distribution of products $pq$ where
$p\gets\mathfrak{P}$ and $q\gets\mathfrak{P}$ is RSA distribution. 
\lastline


\middlefoil{Relations Between\vspace*{1ex}\\ Problems}


\foilhead[-1cm]{CDH group is also DH group}

\textbf{Intuition:} If we can compute discrete logarithm then CDH is easy.


\textbf{Reduction.} Let $\AD$ be a DL-finder algorithm. Then the adversary $\ADB$

\illustration[scale=1.0, angle=0, trim=0cm 0.5cm 0cm -0.5cm]{cdh-dl-reduction.eps}

is as successful as the adversary $\AD$:
\begin{align*}
  \advCDHXX{\GG}{\ADB}=\advDLXX{\GG}{\AD}\enspace.
\end{align*}
Hence $(t,\varepsilon)$-secure CDH group must be also $(t,\varepsilon)$-secure DL group.
\lastline

\foilhead[-1cm]{Formal proof}

The adversary $\AD$ sees the following chain of events 
\illustration[scale=1.0, angle=0, trim=0cm 0.5cm 0cm -0.5cm]{cdh-dl-reduction-i.eps}

As $z=g^{xy} \Leftrightarrow xy = \overline{x}y \Leftrightarrow x=\overline{x}$ we can further simplify

\illustration[scale=1.0, angle=0, trim=0cm 0.5cm 0cm -0.5cm]{cdh-dl-reduction-ii.eps}

\foilhead[-1cm]{Simple and difficult puzzles}

\textbf{Intuition:} A good algorithm \emph{should} work uniformly well on each instance.

\illustration[scale=0.8, angle=0, trim=0cm 0.5cm 0cm -0.5cm]{instance-profile.eps}


\foilhead[-1cm]{Random self-reducibility}

Any instance of a discrete logarithm can be reduced to a random instance.  

\illustration[scale=1.0, angle=0, trim=0cm 0.5cm 0cm -0.5cm]{dl-dl-reduction.eps}

The adversary $\AD$ sees the following chain of events 

\illustration[scale=1.0, angle=0, trim=0cm 0.5cm 0cm -0.5cm]{dl-dl-reduction-i.eps}

and thus the worst case advantage $\pr{x=\ADB(g^x)}=\advDLXX{\GG}{\AD}$.\lastline


\foilhead[-1cm]{Consequences of random self-reducibility}

\textbf{Consequence:} There are no hard instances but easy instances may exist.

\illustration[scale=1.1, angle=0, trim=0cm -0.5cm 0cm -0.5cm]{self-reducibility-profiles.eps}

\begin{triangles}
  \item The average success is larger for hard instances.
  \item Easy instances are handled worse than by the original algorithm. 
  \item Specialised algorithms for specific instance classes might work better.
\end{triangles}


\foilhead[-1cm]{Consequences of random self-reducibility}

\textbf{Consequence:} There are various trade-offs between time and success.
\begin{triangles}
  \item By repeating the DL-computations we can increase the success.
  \item Any estimate on parameters $t,\varepsilon$ gives a lower bound to success.
\end{triangles}

\illustration[scale=1.1, angle=0, trim=0cm -0.5cm 0cm -1.0cm]{sr-time-success-tradeoffs.eps}

\end{document}
