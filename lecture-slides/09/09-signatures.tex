\documentclass[landscape,footrule]{foils}
\usepackage[lecture-serie]{foiltex-extra}
\usepackage{color}
\usepackage{crysymb}
\usepackage[draft]{crygame}
\usepackage{crypto-ii}
\usepackage{graphics}
\usepackage[pdftex]{graphicx} 


\newcommand{\lecture}{Digital Signatures}
\newcommand{\lserie}{MTAT.07.003 Cryptology II}
\newcommand{\ldate}{16 April, 2010}
\newcommand{\lauthor}{Sven Laur}
\newcommand{\linst}{University of Tartu}
\graphicspath{{./illustrations/}}


\renewcommand{\SK}{{\red{\mathsf{sk}}}}
\renewcommand{\PK}{{\blue{\mathsf{pk}}}}
\newcommand{\lastline}{\vspace*{-2ex}}
\newcommand{\spreadappart}{\vspace*{\fill}}


\newcommand{\probes}{\mathsf{probes}}
\newcommand{\HALL}{\HHH_{\mathrm{all}}}


\begin{document}
\titlefoil

\middlefoil{Formal Syntax}



\foilhead[-2cm]{Digital signature scheme}

\illustration[scale=0.75, angle=-90, clip, trim=1.5cm 4.0cm 13.0cm 1.0cm]{active-attack.eps}

\begin{triangles}
\item To establish electronic identity, Charlie must generate
  $(\PK,\SK)\gets\GEN$ and convinces others that the public
  information $\PK$ represents him.
\item A \emph{keyed hash function} $\SIGN_\SK:\MSPACE\to\SSPACE$
  takes in a \emph{plaintext} and outputs a corresponding
  \emph{digital signature}.
\item A \emph{public verification algorithm}
  $\VERIFY_\PK:\MSPACE\times\SSPACE\to\set{0,1}$ tries to distinguish
  between altered and original message pairs.
\item The signature scheme is \emph{functional} if for all
  $(\PK,\SK)\gets\GEN$ and $m\in\MSPACE$:
  \centerline{$\VERIFY_\PK(m,\SIGN_\SK(m))=1$\enspace.}
\end{triangles}


\foilhead[-1cm]{Example. RSA-1024 signature scheme}

\textbf{Key generation $\GEN$:}
\begin{enumerate}
\item Choose uniformly $512$-bit prime numbers $p$ and $q$.
\item Compute $N=p\cdot q$ and $\phi(N)=(p-1)(q-1)$.
\item Choose uniformly $e\gets\ZZ_{\phi(N)}^*$ and set $d=e^{-1}\mod \phi(N)$.
\item Output $\SK=(p,q,e,d)$ and $\PK=(N,e)$.
\end{enumerate}
\Bigskip

\textbf{Signing and verification:}
\begin{align*}
 &\MSPACE=\ZZ_N,\quad \SSPACE=\ZZ_N,\quad \RSPACE=\emptyset\\
 &\begin{aligned}
 \SIGN_\SK(\blue{m})&=\blue{m}^{\red{d}}\mod N \\
 \VERIFY_\PK(\blue{m},\blue{s})&=1\quad\Leftrightarrow\quad  \blue{m}=\blue{s}^e\mod N\enspace.
 \end{aligned}
\end{align*}



\foilhead[-1cm]{Security definitions}

\textbf{Attack scenarios}
\begin{triangles}
\item \emph{Key only attack}. The adversary has access only to the
  public key $\PK$.

\item \emph{Chosen message attack.} Besides the public key $\PK$, the
  adversary can adaptively query a list of valid signatures
  $(m_1,s_1),\ldots,(m_n,s_n)$.
\end{triangles}
\bigskip


\textbf{Attack types}
\begin{triangles}
\item \emph{Universal forgery.} The adversary must create a valid
  signature for a prescribed message $m$ that is chosen from a
  distribution $\red{\MSPACE_0}$.

\item \emph{Existential forgery.} The adversary must create a valid
  signature for some message $m$, i.e., there are no limitations on
  the choice of message.
\end{triangles}
\Bigskip

\centerline{\fbox{One more signature attack = Chosen message attack + Existential forgery}}


\foilhead[-1cm]{Security against one more signature attack}

A signature scheme is $(t,q,\varepsilon)$-\emph{secure against one more
signature} attack if\vspace*{-1ex}
\begin{align*}
\advFORGEXX{}{\AD}=\pr{\smash{\GAME^\AD=1}}\leq\varepsilon
\end{align*}
\vspace*{-3.5ex}\\   
for any $t$-time adversary $\AD$ where\vspace*{-1.5ex}
\begin{small}
\begin{align*}
  \begin{game}{\GAME^\AD}
    &(\PK,\SK)\gets\GEN\\
    &\!\!\!\begin{forblock}{i\in\set{1,\ldots,q}}
    & m_i\gets\AD(s_{i-1})\\
    & s_i\gets\SIGN_\SK(m_i)\\
    \end{forblock}\\
    &(m,s)\gets\AD\\
    &\IF (m,s)\in\set{(m_1,s_1),\ldots,(m_q,s_q)} \RETURN 0\\
    &\RETURN \VERIFY_\PK(m,s)
  \end{game}
\end{align*}%
\end{small}%


\foilhead[-1.0cm]{A conceptual description of digital signatures}

A digital signature is a \emph{non-interactive} and
\emph{transferable} counterpart of the following extended
identification protocol:
\begin{enumerate}
\item Verifier sends a message $m$ to a prover.
\item The prover accepts the message $m$.
\item The prover authenticates him or herself by using $\SK$.
\end{enumerate}
\bigskip\spreadappart

\emph{Transferability} means that signature must be verifiable by
other parties that did not participate in the creation of the
signature.
\bigskip\spreadappart

As a result, digital signature is either 
\begin{triangles}
\item a non-interactive proof of possession that is linked to the
  message $m$.
\item a non-interactive proof of knowledge that is linked to the
  message $m$.
\end{triangles}


\middlefoil{Digital Signatures Based on\vspace*{1ex}\\ Proofs of Possession}


\foilhead[-1cm]{Lamport one-time signatures}

\textbf{Public parameters:}\vspace*{1ex}\\
\hspace*{1em}
\begin{minipage}{0.92\linewidth}
  Let $\set{0,1}^n$ be the message space and $f:\XXX\to\YYY$ a one-way
  function.
\end{minipage}
\bigskip

\textbf{Key Generation:}\vspace*{1ex}\\
\hspace*{1em}
\begin{minipage}{0.92\linewidth}
  Generate $2\times n$ random elements $x_{ij}\getsu\XXX$
  and compute  $y_{ij}\gets f(x_{ij})$. The secret key is
  $\SK=(x_{ij})$ and the public key is $\PK=(y_{ij})$.
\end{minipage}
\bigskip

\textbf{Signing:}\vspace*{1ex}\\
\hspace*{1em}
\begin{minipage}{0.92\linewidth}
  To sign a message $m=m_n\ldots m_1$ release  $\sigma=(x_{1m_1},\ldots,x_{nm_n})$.
\end{minipage}
\bigskip

\textbf{Verification:}\vspace*{1ex}\\
\hspace*{1em}
\begin{minipage}{0.92\linewidth}
  A signature $(m,\sigma)$ is valid if $f(\sigma_i)=y_{im_i}$ for $i=1,\ldots,n$.
\end{minipage}
\lastline

\foilhead[-1cm]{Security of one-time signatures}

\textbf{Theorem.}  If $f$ is $(t,\varepsilon)$-one-way function, then
Lamport signature scheme is $(t,1,2n\varepsilon)$-secure against one
more signature attack.

\textbf{Proof.}
A successful forgery must reveal an inverse of $y_{i\neg m_i}$ for some $i$. 
\begin{align*}
  \begin{fblock}{\ADB(y)}
    & \text{Choose $2n-1$ elements of $(x_{ij})\getsu\XXX$.}\\
    & \text{Compute $2n-1$ entries $y_{ij}\gets f(x_{ij})$.}\\
    & \text{Put $y$ to the missing place and send $\PK$ to $\AD$.}\\
    & \text{Reveal $n$ elements of $(x_{ij})$ to $\AD$ if possible.}\\
    & \text{Return $x$ if $\AD$ reveals $x$ such that $f(x)=y$.} 
  \end{fblock}
\end{align*}

\foilhead[-1cm]{Merkle signature scheme}
 
\illustration[scale=0.90, angle=-90, clip, trim=3.0cm 2.5cm 11.0cm 3.5cm]{hash-tree-signatures.eps}

\textbf{Detailed description}
\begin{triangles}
  \item All intermediate values $c_{u}$ are computed by hashing $c_{u}\gets h(c_{u0},c_{u1})$.
  \item The second level values $y_{u}$ are computed as before $y_u\gets f(x_u)$.
  \item The secret key can be further compacted by computing $x_u\gets
    g(u,k)$ where $g:\III\times\KSPACE\to\XXX$ induces a pseudorandom
    function family $\GGG=\set{g_k}$.
\end{triangles}





\foilhead[-1cm]{Trapdoor one-way permutations}

A collection of trapdoor permutations $\FFFTPERM$ is determined by
three algorithms $(\GEN,\MAP,\INV)$ such that
\begin{align*}
  \forall (\PK,\SK)\gets\GEN,\quad \forall m\in\MSPACE_\PK:\qquad
  \INV_\SK(\MAP_\PK(m))=m\enspace
\end{align*}
and both algorithms $\MAP_\PK(\cdot)$ and $\INV_\SK(\cdot)$ are
deterministic.


\foilhead[-1cm]{OW-CPA security} 


A collection of trapdoor permutations $\FFFTPERM$ is
$(t,\varepsilon)$-secure if for any $t$-time adversary $\AD$
\begin{align*}
  \advINVCPAXX{}{\AD}=\pr{\GAME^\AD=1}\leq \varepsilon
\end{align*}
where
\begin{align*}
  \begin{game}{\GAME^\AD}
    &(\PK,\SK)\gets\GEN\\
    &m\getsu\MSPACE_\PK\\
    &y\gets\MAP_\PK(m)\\
    &\RETURN [\AD(\PK,y)\iseq m]
  \end{game}
\end{align*}



\foilhead[-1cm]{Full Domain Hash}

\textbf{Setup}\vspace*{1ex}\\
\hspace*{1.5em} Run $(\PK,\SK)\gets\GEN$ to create an instance of
trapdoor permutation.
\hspace*{1.5em} Choose $h:\MSPACE\to\MSPACE_\PK$  from a collision resistant function family $\HHH$.\\ 


\textbf{Signing}\vspace*{1ex}\\
\hspace*{1.5em} To sign  $m\in\MSPACE$, compute $x\gets h(m)$ and output $s\gets\INV_\SK(x)$.\\

\textbf{Verification}\vspace*{1ex}\\
\hspace*{1.5em} A pair $(m,s)$ is a valid signature if $h(m)=\MAP_\PK(s)$.


\foilhead[-1cm]{Random oracle model}

Let us model the hash function by an oracle $\ORA_\PK(\cdot)$ that
evaluates $h\getsu\HALL$ where $\HALL=\set{h:\MSPACE\to\MSPACE_\PK}$
is a set of all functions.

Thus, the advantage is now computed as average
\begin{align*}
  \advFORGEXX{\HALL}{\AD}=\frac{1}{\abs{\HALL}}\cdot\sum_{h\in\HALL}\advFORGEXX{h}{\AD}\enspace.
\end{align*}


In reality, we substitute $\HALL$ with a suitable hash family $\HHH$
and hope
\begin{align*}
  \advFORGEXX{\HHH}{\AD}=\frac{1}{\abs{\HHH}}\cdot\sum_{h\in\HHH}\advFORGEXX{h}{\AD}
  \approx \advFORGEXX{\HALL}{\AD}\enspace
\end{align*}
for all relevant adversaries $\AD\in\ADDIST$ humans devise.

\foilhead[-1cm]{Security in the random oracle model}

\textbf{Theorem.} Let $\FFFTPERM$ is $(t,\varepsilon)$-secure
collection of trapdoor permutations such that $\MAP_\PK(\cdot)$ is
always $\tau$-time computable. Then the FDH signature scheme is
$(t-(q_h+q_s+1)\cdot\tau,q_s,(q_h+q_s+1)\cdot\varepsilon)$-secure
against one more signature attack provided that the adversary can do
up to $q_h$ hash queries.\vspace*{1cm}


\textbf{Drawbacks}
\begin{triangles}
  \item The result holds only in random oracle model.
  \item The security bound increases linearly with $q_h$ and $d_s$.
  \item PSS signature scheme achieves a ``sublinear'' bound. 
  \item The proof cannot be generalised to the standard model~[Pallier2007].
\end{triangles}

\foilhead[-1cm]{The corresponding proof}

\begin{triangles}
\item We can always assume that $\AD$ queries hash for the forgery.
\item The following evaluation strategies lead to identical results
    \begin{align*}
      \begin{aligned}
      \begin{fblock}{\ORA_\PK(m)}
        &\IF H[m]=\bot\ \THEN\\
        &\begin{cblock}
         & H[m]\getsu\MSPACE_\PK\\
        \end{cblock}\\
        &\RETURN H[m]
      \end{fblock}\\
       \
      \end{aligned}
      \qquad\qquad
      \begin{fblock}{\ORA_\PK(m)}
        &\IF H[m]=\bot\ \THEN\\
        &\begin{cblock}
         & S[m]\getsu\MSPACE_\PK\\
         & H[m]\gets\MAP_\PK(S[m])\\
        \end{cblock}\\
        &\RETURN H[m]
      \end{fblock}
    \end{align*}
  \item Let $\varepsilon_i$ be the probability that forgery was
    successful and the $i$th hash query corresponded to the message.
  \item Then $\varepsilon_i\leq \varepsilon$, since otherwise we would
    have a too efficient inverter.
\end{triangles}



\middlefoil{Digital Signatures Based on\vspace*{1ex}\\ Proofs of Knowledge}



\foilhead[-1.5cm]{Sigma protocols}

\illustration[scale=0.82, angle=-90, clip, trim=3.5cm 2.5cm 11.0cm 2.5cm]
 {sigma-protocols.eps}


A \emph{sigma protocol} for an efficiently computable relation
$R\subseteq\set{0,1}^*\times\set{0,1}^*$ is a three move
protocol that satisfies the following properties.
\begin{triangles}
\item\textbf{$\boldsymbol{\Sigma}$-structure.} A prover first sends a
  commitment, next a verifier sends \emph{varying} challenge, and
  then the prover must give a consistent response.
\item \textbf{Functionality.} The protocol run between an honest
  prover $\PROVER(\SK)$ and verifier $\VERIFIER(\PK)$ is always
  accepting if $(\SK,\PK)\in R$.
\end{triangles}

\foilhead[-1.5cm]{Security properties of sigma protocols}

\illustration[scale=0.82, angle=-90, clip, trim=3.5cm 2.5cm 11.0cm 2.5cm]
 {sigma-protocols.eps}
\Bigskip

\begin{triangles}
\item \textbf{Perfect simulatability.} There exists an efficient
  \emph{non-rewinding} simulator $\SSS$ such that the output
  distribution of a semi-honest verifier $\VERIFIER_*$ in the real
  world and the output distribution of $\SSS^{\VERIFIER_*}$ in the
  ideal world coincide.
\item \textbf{Special soundness.} There exists an efficient extraction
  algorithm $\EXTR$ that, given two accepting protocol runs
  $(\alpha,\beta_0,\gamma_0)$ and $(\alpha,\beta_1,\gamma_1)$ with
  $\beta_0\neq \beta_1$ that correspond to $\PK$, outputs $\SK_*$ such
  that $(\SK_*,\PK)\in R$

\end{triangles}



\foilhead[-1.5cm]{Fiat-Shamir heuristics}
Fiat-Shamir heuristics converts any sigma-protocol to a signature scheme.

\illustration[scale=0.82, angle=-90, clip, trim=3.5cm 2.5cm 4.0cm 2.5cm]
 {fiat-shamir-heuristics.eps}


\foilhead[-1.0cm]{Generic signature schemes}\enlargethispage{1cm}

Let $\PROT$ be a sigma protocol specified by the prover and verifier
algorithms $\PROVER_\SK$ and $\VERIFIER_\PK$. Then the corresponding
\emph{generic signature scheme} is following.

\textbf{Setup}\vspace*{1ex}\\
\hspace*{1.5em} Use output $(\PK,\SK)\gets\GEN$ of the sigma protocol setup.

\textbf{Signing}\vspace*{1ex}\\
\hspace*{1.5em} To sign a message $m$, output $(\alpha,\beta,\gamma)$
as a signature where
\begin{align*}
  \alpha\gets \PROVER_\SK,\quad \beta\gets h(m,\alpha),\quad
  \gamma\gets\PROVER_\SK(\beta)\enspace.
\end{align*}\ \vspace*{-8ex}\\
 

\textbf{Verification}\vspace*{1ex}\\
\hspace*{1.5em} A tuple $(m,\alpha,\beta,\gamma)$ is a valid signature
if
\begin{align*}
\beta=h(m,\alpha)\wedge\VERIFY_\PK(\alpha,\beta,\gamma)=1\enspace.  
\end{align*}

\foilhead[-1.0cm]{Famous examples}

Schnorr authentication protocol
\begin{diamonds}
\item Schnorr signature scheme
\item DSA algorithm
\item ElGamal signature scheme
\item Nyberg-Rueppel signature scheme
\end{diamonds}
\vspace*{1cm}

Fiat-Shamir identification protocol 
\begin{diamonds}
  \item Feige-Fiat-Shamir signature scheme 
\end{diamonds}
\vspace*{1cm}

Guillou-Quisquater identification protocol
\begin{diamonds}
  \item Guillou-Quisquater signature scheme 
\end{diamonds}



\foilhead[-1.0cm]{Security in the random oracle model}

\textbf{Basic Forking Lemma.}  Let $\kappa$ be the knowledge error of
the sigma protocol. If for a particular public key $\PK$, a $t$-time
adversary $\AD$ manages to output valid signature by making at most
$\ell$ queries to the random oracle and no queries to the signing oracle
with probability $\varepsilon$, then there exist an extraction
algorithm $\KKK$ for that runs in expected time
\begin{align*}
  \EXP[\tau]=\Theta\brak{\frac{(\ell+2)\cdot
      t}{\varepsilon-\kappa}}
\end{align*}
and returns the corresponding secret key $\SK$.\vspace*{1cm}


\textbf{Simulatability Lemma.}  Oracle calls to the signing oracle can
be replaced with the runs of the sigma protocol simulator. The
probability of simulation failures due to the contradicting
assignments for $\ORA(\cdot)$ are negligible.



\foilhead[-1cm]{An encoding of the extraction task}

Assume that $\AD$ never queries the same value $h(m_i,\alpha_i)$ twice
and that $\AD$ itself verifies the validity of the candidate signature
$(m_{n+1},s_{n+1})$.

Let $\omega_0$ denote the randomness used by $\AD$ and let
$\omega_1,\ldots \omega_{\ell+1}$ be the replies for the hash queries
$h(m_i,\alpha_i)$. Now define
\begin{align*}
A(\omega_0,\omega_1,\ldots,\omega_{\ell+1})=  
 \begin{cases}
   i,&\text{ if the }i^{\text{th}}\text{ reply $\omega_i$ is used in forgery}\enspace,\\
   0,&\text{ if }\AD \text{ fails}\enspace.
 \end{cases}
\end{align*}
\begin{triangles}
\item For any
  $\overline{\vec{\omega}}=(\omega_0,\ldots,\omega_{i-1},\overline{\omega}_i,\ldots,\overline{\omega}_{\ell+1})$,
  $\AD$ behaves identically up to the $i^{\text{th}}$ query as with
  the randomness $\vec{\omega}$.
\item To extract the secret key $\SK$, we must find $\vec{\omega}$ and
  $\overline{\vec{\omega}}$ such that
  $A(\vec{\omega})=i$ and $A(\overline{\vec{\omega}})=i$ and $\omega_i\neq\overline{\omega}_{i}$.
\end{triangles}

\foilhead[-1cm]{Classical algorithm}

$\underline{\mathsf{Rewind}}$:
\begin{enumerate}
 \item Probe random entries $A(\vec{\omega})$ until $A(\vec{\omega})\neq 0$.
 \item Store the matrix location $\vec{\omega}$ and the rewinding point $i\gets A(\vec{\omega})$.
 \item Probe random entries $A(\overline{\vec{\omega}})$ until $A(\overline{\vec{\omega}})=i$.
 \item Output the location tuple $(\vec{\omega},\overline{\vec{\omega}})$.
\end{enumerate}
\Bigskip

$\underline{\smash{\mathsf{Rewind\text{-}Exp}}}$:
\begin{enumerate}
  \item Repeat the procedure $\mathsf{Rewind}$ until $\omega_i\neq\overline{\omega}_i$.
  \item Use the knowledge extraction lemma to extract $\SK$.
\end{enumerate}




\foilhead[-1cm]{Average-case running time}

\textbf{Theorem.}  If a array $A(\vec{\omega})$ with entries in
$\set{0,\ldots,\ell}$ contains $\varepsilon$-fraction of nonzero entries,
then $\mathsf{Rewind}$ and $\mathsf{Rewind\text{-}Exp}$ make on
average
\begin{align*}
  \EXP[\probes| \mathsf{Rewind}] &=\frac{\ell+1}{\varepsilon}\\
  \EXP[\probes|\mathsf{Rewind\text{-}Exp}] &=\frac{\ell+1}{\varepsilon-\kappa}
\end{align*}
probes where the \emph{knowledge error} 
\begin{align*}
 \kappa=\sum_{i=1}^\ell\pr{\omega_i=\overline{\omega}_i}\enspace.  
\end{align*}

\textbf{Proof.} We prove this theorem in another lecture.



\middlefoil{Interpretation of the Results}

\foilhead[-1cm]{Objective and subjective security}


All forms of the Forking Lemma are stated with respect to a fixed key
and thus give only soundness guarantees.\vspace*{0.5cm}

If we believe that no human can devise an algorithm that for a
\emph{particular} public key $\PK$ computes the corresponding secret
key, then we also get the corresponding \emph{subjective} security
guarantee.\vspace*{0.5cm}

For objective security guarantees, we have to consider average success
probability over all public keys. In this situation, the knowledge
extraction with high probability is overkill.  Fixed number of probes
together with Jensen's inequality give more tight bounds.\vspace*{0.5cm}

\red{All these security guarantees hold in the random oracle model!}



\foilhead[-2cm]{Average case nature of advantages}

\illustration[scale=0.90, angle=-90, clip, trim=3.5cm 2.5cm 11.0cm 2.5cm]{average-nature.eps}


The limit on the average advantage over all functions means:
\begin{triangles}
  \item An attack algorithm $\AD$ can be successful on few functions
  \item For randomly chosen function family $\HHH$ the corresponding
    average advantage is comparable with high probability over
    the choice of $\HHH$.
\end{triangles}

\textcolor{red}{Such argumentation does not rule out possibility that
  one can choose adaptively a specialised attack algorithm
  $\AD$ based on the description of $h$.}

\foilhead[-1cm]{Security against generic attacks}

An adaptive choice of a specialised attack algorithm implies that the
attack depends on the description of the hash function and not the family $\HHH$.

Often, it is advantageous to consider only generic attacks that depend
on the description of function family $\HHH$ and use only black-box
access to the function $h$. Therefore, we can consider two oracles
$\ORA_{\HALL}$ and $\ORA_{\HHH}$.

If $\HHH$ is pseudorandom function family then for any generic attack,
we can substitute $\HHH$ with the $\HALL$ and the success decreases marginally.

\textbf{Theorem.}  Security in the random oracle model implies
security against generic attacks if $\HHH$ is a pseudorandom function family.
\begin{triangles}
  \item The assumption that attackers use only generic attacks is subjective.
  \item Such an assumption are not universal, i.e., there are settings
    where this assumption is clearly irrational (various non-instantiability results).
\end{triangles}

\foilhead[-1cm]{Why the random oracle argument truly fails?}

All currently used hash functions are iterative
\begin{align*}
  h(m_1,\ldots,m_k)=f(\cdots f(f(\mathsf{iv},m_1),m_2),\ldots,m_k)\enspace.
\end{align*}
Consequently, the class of generic attack $\ADDIST_{\mathrm{bb}}$ that
treats $h(\cdot)$ as a black box without internal structure is
sub-optimal.

Instead, we should consider a wider class of attacks that treat $f$ as
a black box, but still try to exploit iterative structure of the hash
function.  

However, the corresponding black box adversaries
$\AD^{f(\cdot),h(\cdot)}$ can distinguish $\HHH$ form $\HALL$ with
high probability.

As a result, there might be a successful attack strategy
$\AD^{f(\cdot),h(\cdot)}$ that works for all possible iterative hash
functions, although the signature scheme is secure in the random
oracle model.





\end{document}

\foilhead[-1.0cm]{OW-CCA2 security}\enlargethispage{2ex}

A collection of trapdoor permutations $\FFFTPERM$ is
$(t,\varepsilon)$-OW-CCA2 secure if for any $t$-time adversary $\AD$
\begin{align*}
  \advINVCCAIIXX{}{\AD}=\pr{\GAME^\AD=1}\leq \varepsilon
\end{align*}\ \vspace*{-3ex}\\
where\vspace*{-2ex}
\begin{align*}
  \begin{game}{\GAME^\AD}
    &(\PK,\SK)\gets\GEN\\
    &m\getsu\MSPACE_\PK\\
    &y\gets\MAP_\PK(m)\\
    &\RETURN [\AD^{\ORA(\cdot)}(y)\iseq x]
  \end{game}
\end{align*}
and $\ORA(c)=\INV_\SK(c)$ if $y\neq c$ and $\bot$ otherwise.
