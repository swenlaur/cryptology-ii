\documentclass[landscape,footrule]{foils}
\usepackage[lecture-serie]{foiltex-extra}
\usepackage{color}
\usepackage{crysymb}
\usepackage[draft]{crygame}
\usepackage{crypto-ii}
\usepackage{graphics}
\usepackage[pdftex]{graphicx} 


\newcommand{\lecture}{Entity Authentication}
\newcommand{\lserie}{MTAT.07.003 Cryptology II}
\newcommand{\ldate}{9 April, 2010}
\newcommand{\lauthor}{Sven Laur}
\newcommand{\linst}{University of Tartu}
\graphicspath{{./illustrations/}}


\newcommand{\probes}{\mathsf{probes}}
\newcommand{\lastline}{\vspace*{-2ex}}
\newcommand{\spreadappart}{\vspace*{\fill}}


\renewcommand{\SK}{{\red{\mathsf{sk}}}}
\renewcommand{\PK}{{\blue{\mathsf{pk}}}}





\begin{document}
\titlefoil

\middlefoil{Formal Syntax}

\foilhead[-1.5cm]{Entity authentication}

\illustration[scale=0.8, angle=-90, clip, trim=1.5cm 0.0cm 12.5cm 0.7cm]{entity-authentication.eps} 

\begin{triangles}
\item The communication between the prover and verifier must be
  authentic.
\item To establish electronic identity, Charlie must generate
  $(\PK,\SK)\gets\GEN$ and convinces others that the public
  information $\PK$ represents him.
\item The entity authentication protocol must convince the verifier
  that his or her opponent possesses the secret $\SK$.
\item An entity authentication protocol is \emph{functional} if an
  honest verifier $\VERIFIER_{\blue{\PK}}$ always accepts an honest prover
  $\PROVER_{\red{\SK}}$.
\end{triangles}

\foilhead[-1.0cm]{Classical impossibility results}

\textbf{Inherent limitations.}  Entity authentication is impossible 
\begin{itemize}
 \item[(i)\,] \textbf{if} authenticated communication is unaffordable in the setup
  phase;
\item[(ii)] \textbf{if} authenticated communication is unaffordable in the second phase.
\end{itemize}
\bigskip


\textbf{Proof.} Man-in-the-middle attacks. Chess-master attacks.
\Bigskip  
\spreadappart

\textbf{Conclusions}
\begin{triangles}
  \item It is impossible to establish legal identity without physical measures.
  \item Any smart card is susceptible to physical attacks regardless of
    the cryptographic countermeasures used to authenticate transactions.
  \item Secure e-banking is impossible if the user does not have full
    control over the computing environment (secure e-banking is
    practically impossible).
\end{triangles}


\foilhead[-1.5cm]{Physical and legal identities}

\illustration[scale=0.8, angle=-90, clip, trim=0.5cm 0.0cm 13.0cm 0.7cm]{physical-and-legal-identities.eps}
\bigskip
\spreadappart

\begin{triangles}
\item Entity authentication is possible only if all participants have
  set up a network with authenticated communication links.
\item A role of a entity authentication protocol is to establish a
  convincing bound between physical network address and legal
  identities.
\item A same legal identity can be in many physical locations and move
  from one physical node to another node.
\end{triangles}


\middlefoil{Challenge-Response\vspace*{1ex}\\ Paradigm}

\foilhead[-1.5cm]{Salted hashing}

\textbf{Global setup:}  \\
\hspace*{0.03\linewidth}
\begin{minipage}[t]{0.97\linewidth}
  Authentication server $\VERIFIER$ outputs a description of a hash function $h$.
\end{minipage}

\textbf{Entity creation:}  \\
\hspace*{0.03\linewidth}
\begin{minipage}[t]{0.97\linewidth}
  A party $\PROVER$ chooses a password $\SK\getsu\set{0,1}^\ell$ and
  a nonce $r\getsu\set{0,1}^k$. The public authentication information
  is $\PK=(r,c)$ where $c\gets h(\SK,r)$.
\end{minipage}

\textbf{Entity authentication:}  \\
\hspace*{0.03\linewidth}
\begin{minipage}[t]{0.97\linewidth}
  To authenticate him- or herself, $\PROVER$ releases $\SK$ to the
  server $\VERIFIER$ who verifies that the hash value is correctly
  computed, i.e., $c= h(\SK,r)$.
\end{minipage}\vspace*{2ex}

\textbf{Theorem.}  If $h$ is $(t,\varepsilon)$-secure one-way
function, then no $t$-time adversary $\AD$ without $\SK$ can succeed in the protocol
with probability more than $\varepsilon$.
\begin{triangles}
\item There are no secure one-way functions for practical sizes of
  $\SK$.
\item A malicious server can completely break the security. 
\end{triangles}


\foilhead[-1.0cm]{RSA based entity authentication}

\textbf{Global setup:}  \\
\hspace*{0.03\linewidth}
\begin{minipage}[t]{0.97\linewidth}
  Authentication server $\VERIFIER$ fixes the minimal size of RSA keys. 
\end{minipage}


\textbf{Entity creation:}  \\
\hspace*{0.03\linewidth}
\begin{minipage}[t]{0.97\linewidth}
  A party $\PROVER$ runs a RSA key generation algorithm
  $(\PK,\SK)\gets\GEN_{\mathrm{rsa}}$ and outputs the public key $\PK$
  as the authenticating information.
\end{minipage}

\textbf{Entity authentication:}  \\
\hspace*{0.03\linewidth}
\begin{minipage}[t]{0.97\linewidth}
  \begin{enumerate}
  \item $\VERIFIER$ creates a challenge $c\gets\ENC_\PK(m)$ for
    $m\getsu\MSPACE$ and sends $c$ to $\PARTY$.
  \item $\PROVER$ sends back $\overline{m}\gets\DEC_\SK(c)$.
  \item $\VERIFIER$ accepts the proof if $m=\overline{m}$.
  \end{enumerate}
\end{minipage}
\bigskip
\spreadappart

This protocol can be generalised for any public key cryptosystem.\\
The general form of this protocol is known as
\emph{challenge-response protocol}.\\
This mechanism provides explicit security guarantees in the TLS protocol.
\lastline

\foilhead[-1.5cm]{The most powerful attack model}

\illustration[scale=0.65, angle=-90, clip, trim=1.0cm 0.0cm 13.0cm 0.7cm]{active-attack.eps} 

Consider a setting, where an adversary $\AD$ can impersonate verifier $\VERIFIER$
\begin{triangles}
\item The adversary $\AD$ can execute several protocol instances with
  the honest prover $\PROVER$ in parallel to spoof the challenge
  protocol.
\item The adversary $\AD$ may use protocol messages arbitrarily as
  long as $\AD$ does not conduct the crossmaster attack.
\end{triangles}
Let us denote the corresponding success probability by
\begin{align*}
\advEAUTHXX{}{\AD}=\pr{\smash{(\PK,\SK)\gets\GEN: \VERIFIER^{\AD}=1}}\enspace.
\end{align*}


\foilhead[-1.0cm]{Corresponding security guarantees}



\textbf{Theorem.}  If a cryptosystem used in the challenge-response
protocol is $(t,\varepsilon)$-IND-CCA2 secure, then for any  $t$-time
adversary $\AD$ the corresponding success probability 
$\advEAUTHXX{}{\AD}\leq\frac{1}{\abs{\MSPACE}}+\varepsilon$.

\textbf{Proof.} A honest prover acts as a decryption
oracle.
\bigskip

\textbf{The nature of the protocol}
\begin{triangles}
\item The protocol proves only that the prover has access to the
  decryption oracle and therefore the prover must \emph{possess} the
  secret key $\SK$.
\item The possession of the secret key $\SK$ does not imply the
  \emph{knowledge} of it. For example, the secret key $\SK$ might be
  hardwired into a smart card.
\item Usually, the inability to decrypt is a strictly stronger
  security requirement than the ability to find the secret key.
\item \emph{Knowledge} is permanent whereas \emph{possession} can be
  temporal.
\end{triangles}


\middlefoil{Proofs of knowledge}

\foilhead[-1.0cm]{Schnorr identification protocol}

\illustration[scale=0.85, angle=-90, clip, trim=3.5cm 0.0cm 11.0cm 0.7cm]{schnorr-identification-scheme.eps} 

The group $\GG=\langle g\rangle$ must be a DL group with a prime cardinality $q$.
\begin{triangles}
  \item The secret key $x$ is the discrete logarithm of $y$.
  \item The verifier $\VERIFIER$ is assumed to be semi-honest.
  \item The prover $\PROVER$ is assumed to be potentially malicious. 
  \item We consider only security in the standalone setting.
\end{triangles}

\foilhead[-1.0cm]{Zero-knowledge principle}

\illustration[scale=0.8, angle=-90, clip, trim=3.5cm 2.0cm 4.0cm 2.0cm]{zero-knowledge.eps} 

Lucy should be equally \emph{successful} in both experiments.

\foilhead[-1.0cm]{Simulation principle}

\illustration[scale=0.80, angle=-90, clip, trim=3.5cm 2.0cm 4.0cm 2.0cm]{simulation-principle.eps} 

Lucy should not be able to distinguish between these two experiments.

\foilhead[-1.0cm]{Zero-knowledge property}

\textbf{Theorem.}  If a $t$-time verifier $\VERIFIER_*$ is semi-honest
in the Schnorr identification protocol, then there exists
$t+\Oh(1)$-algorithm $\VERIFIER_{\circ}$ that has the same output
distribution as $\VERIFIER_*$ but do not interact with the prover
$\PROVER$.
\bigskip
\spreadappart

\textbf{Proof.}\\
Consider a code wrapper $\SSS$ that chooses $\beta\getsu\ZZ_q$ and
$\gamma\getsu\ZZ_q$ and computes $\alpha\gets g^\gamma\cdot
y^{-\beta}$ and outputs whatever $\VERIFIER_*$ outputs on the
transcript  $(\alpha,\beta,\gamma)$.


\begin{triangles}
\item If $x\neq 0$, then $\gamma=\beta+xk$ has indeed a uniform
  distribution.
  \item For fixed $\beta$ and $\gamma$, there exist only a single consistent value of $\alpha$.
\end{triangles}
{$\square$}
\bigskip

\textbf{Rationale:}
Semi-honest verifier learns nothing from the interaction with the prover.
The latter is known as \emph{zero-knowledge} property.

 

\foilhead[-1.5cm]{Knowledge-extraction lemma}
\illustration[scale=0.85, angle=-90, clip, trim=0.5cm 0.0cm 14.0cm 0.7cm]{knowledge-extraction-lemma.eps}

This property is known as  \emph{special-soundness.}
\begin{triangles}
\item If adversary $\AD$ succeeds with probability $1$, then we can
  extract the secret key $x$ by rewinding $\AD$ to get two runs with a
  coinciding prefix $\alpha$.
\item If adversary $\AD$ succeeds with a non-zero probability
  $\varepsilon$, then we must use more advanced knowledge-extraction
  techniques.
\end{triangles}


\foilhead[-1.5cm]{Find two ones in a row}
\illustration[scale=0.85, angle=-90, clip, trim=4.5cm 0.0cm 10.5cm 0.7cm]{knowledge-extraction.eps}

Let $A(r,c)$ be the output of the honest verifier $\VERIFIER(c)$ that
interacts with a potentially malicious prover $\PROVER_*(r)$.
\begin{triangles}
\item Then all matrix elements in the same row $A(r,\cdot)$ lead to same $\alpha$ value.
\item To extract the secret key $\SK$, we must find two ones in the same row.
\item We can compute the entries of the matrix on the fly.
\end{triangles}

We derive the corresponding security guarantees a \emph{bit later}.


\foilhead[-1.5cm]{Modified Fiat-Shamir identification protocol}

\illustration[scale=0.85, angle=-90, clip, trim=3.5cm 0.0cm 10.0cm 0.7cm]
{fiat-shamir-identification-scheme.eps} 

All computations are done in $\ZZ_n$, where $n$ is an RSA modulus.
\begin{triangles}
  \item The secret key $s$ is a square root of $v$.
  \item The verifier $\VERIFIER$ is assumed to be semi-honest.
  \item The prover $\PROVER$ is assumed to be potentially malicious. 
  \item We consider only security in the standalone setting.
\end{triangles}



\foilhead[-1.0cm]{Zero-knowledge property}

\textbf{Theorem.}  If a $t$-time verifier $\VERIFIER_*$ is semi-honest
in the modified Fiat-Shamir identification protocol, then there exists
$t+\Oh(1)$-algorithm $\VERIFIER_{\circ}$ that has the same output
distribution as $\VERIFIER_*$ but do not interact with the prover
$\PROVER$.
\bigskip

\textbf{Proof.}\\
Consider a code wrapper $\SSS$ that chooses $\beta\getsu\set{0,1}$,
$\gamma\getsu\ZZ_n^*$, computes $\alpha\gets v^{-\beta}\cdot \gamma^2$
and outputs whatever $\VERIFIER_*$ outputs on the transcript
$(\alpha,\beta,\gamma)$.

\begin{triangles}
\item Since $s$ is invertible, we can prove that $s\cdot\ZZ_n^*=\ZZ_n^*$
  and $s^2\cdot\ZZ_n^*=\ZZ_n^*$. As a result, $\gamma$ is independent of
  $\beta$ and has indeed a uniform distribution.
  \item For fixed $\beta$ and $\gamma$, there exist only a single consistent value of $\alpha$.
\end{triangles}
{$\square$}

\foilhead[-1.0cm]{Knowledge-extraction lemma}

\textbf{Theorem.} The Fiat-Shamir protocol is specially sound.

\textbf{Proof.}  Assume that a prover $\PROVER_*$ succeeds for both
challenges $\beta\in\set{0,1}$:
\begin{align*}
  \gamma_0^2=\alpha,\quad \gamma_1^2=\alpha
  v\qquad\Longrightarrow\qquad
  \frac{\gamma_1}{\gamma_0}=\sqrt{v}\enspace.
\end{align*}

The corresponding extractor construction $\KEXTR$:
\begin{triangles}
  \item Choose random coins $r$ for $\PROVER_*$.  
  \item Run the protocol with $\beta=0$ and record $\gamma_0$
  \item Run the protocol with $\beta=1$ and record $\gamma_1$
  \item Return $\zeta=\frac{\gamma_1}{\gamma_0}$
\end{triangles}

\foilhead[-1.0cm]{Bound on success probability}

\textbf{Theorem.}  Let $v$ and $n$ be fixed. If a potentially
malicious prover $\PROVER_*$ succeeds in the modified Fiat-Shamir
protocol with probability $\varepsilon> \frac{1}{2}$, then the
knowledge extractor $\KEXTR^{\PROVER_*}$ returns $\sqrt{v}$ with
probability $\varepsilon-\frac{1}{2}$.
\bigskip

\textbf{Proof.}  Consider the success matrix $A(r,c)$ as before.  Let
$p_1$ denote the fraction rows that contain only single one and $p_2$
the fraction of rows that contain two ones. Then evidently
$p_1+p_2\leq 1$ and $\frac{p_1}{2}+ p_2\geq \varepsilon$
and thus we can establish $p_2\geq \varepsilon-\frac{1}{2}$.
$\square$
\Bigskip

\textbf{Rationale:} The knowledge extraction succeeds in general only
if the success probability of $\PROVER_*$ is above $\frac{1}{2}$. The
value $\kappa=\frac{1}{2}$ is known as \emph{knowledge error}.


\middlefoil{Matrix Games}

\foilhead[-1cm]{Classical algorithm}

\textbf{Task:} Find two ones in a same row.

$\underline{\mathsf{Rewind}}$:
\begin{enumerate}
 \item Probe random entries $A(r,c)$ until $A(r,c)=1$.
 \item Store the matrix location $(r,c)$.
 \item Probe random entries $A(r,\overline{c})$ in the same row until $A(r,\overline{c})=1$.
 \item Output the location triple $(r,c,\overline{c})$.
\end{enumerate}
\vspace*{2cm}

$\underline{\smash{\mathsf{Rewind\text{-}Exp}}}$:
\begin{enumerate}
  \item Repeat the procedure $\mathsf{Rewind}$ until $c\neq \overline{c}$.
  \item Use the knowledge-extraction lemma to extract $\SK$.
\end{enumerate}

\foilhead[-1cm]{Average-case running time}

\textbf{Theorem.}  If a $m\times n$ zero-one matrix $A$ contains
$\varepsilon$-fraction of nonzero entries, then the $\mathsf{Rewind}$
and $\mathsf{Rewind\text{-}Exp}$ algorithm make on average
\begin{align*}
  \EXP[\probes| \mathsf{Rewind}] &=\frac{2}{\varepsilon}\\
  \EXP[\probes|\mathsf{Rewind\text{-}Exp}] &=\frac{2}{\varepsilon-\kappa}
\end{align*}
probes where $\kappa=\frac{1}{n}$ is a \emph{knowledge error}.

\textbf{Proof.} We prove this theorem in another lecture.

\foilhead[-1.0cm]{Strict time bounds}

Markov's inequality assures that for a non-negative random variable $\probes$
\begin{align*}
  \pr{\probes\geq \alpha}\leq \frac{\E{\probes}}{\alpha}
\end{align*}
and thus $\mathsf{Rewind\text{-}Exp}$ succeeds with probability at
least $\frac{1}{2}$ after $\frac{4}{\varepsilon-\kappa}$ probes.

If we repeat the experiment $\ell$ times, we the failure probability
goes to $2^{-\ell}$.




\middlefoil{From Soundness to Security}

\foilhead[-1.0cm]{Soundness and subjective security}

Assume that we know a constructive proof:\vspace*{-2ex}
\begin{quote}
  If for fixed $\PK$ a potentially malicious $t$-time prover
  $\PROVER_*$ succeeds with probability $\varepsilon>\kappa$, then a
  knowledge extractor $\KEXTR^{\PROVER}$ that runs in time
  $\tau(\varepsilon)=\Oh\bigl(\frac{t}{\varepsilon-\kappa}\bigr)$
  outputs $\SK$ with probability $1-\varepsilon_2$.\vspace*{-2ex}
\end{quote}
and we \emph{believe}:\vspace*{-2ex}
\begin{quote}
  No human can create a $\tau(\varepsilon_1)$-time algorithm that
  computes $\SK$ from $\PK$ with success probability at least
  $1-\varepsilon_2$.\vspace*{-2ex}
\end{quote}
then it is \emph{rational} to assume that:\vspace*{-2ex} 
\begin{quote}
  No human without the knowledge of $\SK$ can create a algorithm
  $\PROVER_*$ that succeeds in the proof of knowledge with probability
  at least $\varepsilon_1$.
\end{quote}

\textbf{Caveat:} For each fixed $\PK$, there exists a trivial
algorithm that prints out $\SK$. Hence, we cannot get objective
security guarantees.


\foilhead[-1.0cm]{Soundness and objective security}

Assume that we know a constructive proof:\vspace*{-2ex}
\begin{quote}
  If for a fixed $\PK$ a potentially malicious $t$-time prover
  $\PROVER_*$ succeeds with probability $\varepsilon>\kappa$, then a
  knowledge extractor $\KEXTR^{\PROVER}$ that runs in time
  $\tau(\varepsilon)=\Oh\bigl(\frac{t}{\varepsilon-\kappa}\bigr)$
  outputs $\SK$ with probability $1-\varepsilon_2$.\vspace*{-2ex}
\end{quote}
and know a mathematical fact that  any $\tau(2\varepsilon_1)$-time algorithm $\AD$
\begin{align*}
\pr{(\PK,\SK)\gets\GEN:\AD(\PK)=\SK}\leq\varepsilon_1(1-\varepsilon_2)  
\end{align*}
then we can prove an average-case security guarantee:\vspace*{-2ex} 
\begin{quote}
  For any $t$-time prover $\PROVER_*$ that does not know the secret key 
  \begin{align*}
    \advEAUTHXX{}{\AD}=\pr{(\PK,\SK)\gets\GEN:
      \VERIFIER^{\PROVER_*(\PK)}=1}\leq 2\varepsilon_1\enspace.
  \end{align*}\ \vspace*{-6ex}
\end{quote}

\foilhead[-1.0cm]{Objective security guarantees}

\textbf{Schnorr identification scheme}

If $\GG$ is a DL group, then the Schnorr identification scheme is
secure, where the success probability is averaged over all possible
runs of the setup $\GEN$.

\vskip 2cm
\textbf{Fiat-Shamir identification scheme}

Assume that modulus $n$ is chosen form a distribution $\NNN$ of RSA
moduli such that on average factoring is hard over $\NNN$. Then the
Fiat-Shamir identification scheme is secure, where the success
probability is averaged over all possible runs of the setup $\GEN$ and
over all choices of modulus $n$.



\end{document}