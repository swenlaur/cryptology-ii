\documentclass{llncs}
\usepackage{crysymb}
\usepackage[draft]{crygame}
\usepackage{crypto-ii}
\bibliographystyle{alpha} 

\title{A Connection between Indistinguishability and Semantic Security}
\author{Sven Laur}
\institute{University of Tartu}

\begin{document}
\maketitle

\section{Formal Definitions}

Intuitively, if objects are indistinguishable then we cannot determine
their personal properties which vary among the population. Indeed, if
we could reliably determine wheter an object is green or not then we
could easily distinguish green objects from yellow ones. Hence,
indistinguishability indirectly implies that one can reliably detect
only trivial properties that either hold or do not hold for the entire
population. This basic argumentation template forms the cornerstone of
contemporary cryptography. In the following, we fill out all details
that are needed to convert this informal argumentation to a formal
proof.

Let $\SSS$ be a distribution of secret values $s$ and let
$\supp(\SSS)$ be the corresponding support.  Then we can define
indistinguishability of states $\supp(\SSS)$ w.r.t. a function
$f:\SSS\to\XXX$. We say that states from $\supp(\SSS)$ are
\emph{$(t,\varepsilon)$-indistinguishable} if for any
$s_0,s_1\in\supp(\SSS)$ and for any $t$-time algorithm $\AD$:
\begin{align*}
  \advIndXX{s_0,s_1}{\AD}%
  =\abs{\pr{x\gets f(s_0):\AD(x)=1}-\pr{x\gets f(s_1):\AD(x)=1}}%
  \leq \varepsilon\enspace.
\end{align*}

To define semantic security, we have to formalise which properties of
a hidden state are plausible or relevant. Essentially, we can talk
about semantic security w.r.t. \emph{computable} functions
$g:\SSS\to\YYY$. Of course, for fixed input and output domains
$\SSS$ and $\YYY$ all functions are computable and thus this
restriction is only a cosmetic addition that represents our intent. In
some scenarios, there are obvious restrictions to the function
$g$. For instance, we might require that the output of $g$ must be
computable during the next one hundred years. 

Now note that for any function $g$ there is a trivial predictor
algorithm $\AD_*$ that outputs the most probable outcome of $g$ over
the distribution $\SSS$ and thus the advantage of an algorithm $\AD$
is defined as the difference
\begin{align*}
  \pr{s\gets\SSS, x\gets f(s): \AD(x)=g(s)}-
  \pr{s\gets\SSS, x\gets f(s): \AD_*(x)=g(s)}\enspace.
\end{align*}
More formally, we say that states $\supp(\SSS)$ are
\emph{$(t,\varepsilon)$-semantically secure} w.r.t. functions $f$ and
$g$ if for any $t$-time adversary $\AD$ the corresponding advantage
 \begin{align*}
   \advSemXX{f,g}{\AD}%
   &=\pr{ \AD(x)=g(s)}
   -\max_{y_*\in\YYY}\pr{s\gets\SSS:g(s)=y_*}\leq\varepsilon\enspace.
\end{align*}

\section{Indistinguishability Implies Semantic Security}

The main aim of this section is to prove the classical theorem about
semantic security and introduce basic concepts of game-playing
proofs.

\begin{theorem}\label{thm:ind-sem}
  If states from $\supp(\SSS)$ are
  \emph{$(2t,\varepsilon)$-indistinguishable} w.r.t. the function $f$,
  then states $\supp(\SSS)$ are also $(t,\varepsilon)$-semantically
  for all functions $g$, i.e., $\advSemXX{f,g}{\AD}\leq\varepsilon$
  for all $t$-time adversaries $\AD$.
\end{theorem}

\begin{proof}
In order to present the proof modularly in easily understandable
steps, we start from the basic semantic security game
\begin{align*}
  \begin{game}{\GAME^\AD}
    &s\gets\SSS\\
    &x\gets f(s)\\
    &y\gets\AD(x)\\
    &\RETURN [y\iseq g(s)]
  \end{game}
\end{align*}
and then gradually rewrite the game until we obtain the desired bound
on the success probability 
\begin{align*}
\pr{\GAME^\AD=1}\leq\pr{\smash{\GAME^{\AD_*}=1}}+\varepsilon\enspace.  
\end{align*}

\paragraph{\textsc{Coin fixing}.} Note that $g$ does not have to be a
deterministic function. However, if $\advSemXX{f,g}{\AD}>\varepsilon$
for some randomised function $g:\SSS\times\Omega\to\YYY$, then there
exists a deterministic function $g_*$ such that
$\advSemXX{f,g_*}{\AD}>\varepsilon$. Indeed,  by definition
\begin{align*}
  \advSemXX{f,g}{\AD}%
  &=\pr{\GAME^{\AD}=1}-\pr{\smash{\GAME^{\AD_*}=1}}\\
  &=\sum_{\omega_0\in\Omega}\pr{\omega\gets\Omega:\omega=\omega_0}
  \cdot\brak{\pr{\GAME_{\omega}^\AD=1}-\pr{\smash{\GAME_{\omega}^{\AD_*}=1}}}\\
  &\leq\max_{\omega\in\Omega}\set{\pr{\GAME_{\omega}^\AD=1}-\pr{\smash{\GAME_{\omega}^{\AD_*}=1}}}
\end{align*}
where the game $\GAME_{\omega}$ is defined as follows
\begin{align*}
  \begin{game}{\GAME_{\omega}^\AD}
    &s\gets\SSS\\
    &x\gets f(s)\\
    &y\gets\AD(x)\\
    &\RETURN [y\iseq g(s;\omega)]
  \end{game}
\end{align*}
Since $g_\omega(\cdot)=g(\cdot;\omega)$ is a deterministic function, we have obtained
\begin{align*}
  \advSemXX{f,g}{\AD}\leq\max_{\omega\in\Omega}\advSemXX{f,g_\omega}{\AD}
\end{align*}
and without loss of generality we can assume that $g$ is
deterministic.  This kind of coin-fixing argument is common in many
cryptographic proofs.

\paragraph{\textsc{Sampling Idiom}.}
As a next step, we will split the domain of $\SSS$ into a set of
disjoint sub-domains that cover the entire domain
\begin{align*}
  \SSS_y\doteq\set{s\in\SSS: g(s)=y}\enspace.
\end{align*}
Since $g$ is deterministic the partition is indeed a well defined. Now
a distribution $\SSS$ naturally defines a distribution over indices
$y$. Let $\YYY_0$ be the distribution over elements of $\YYY$ such that
for all $y_0\in\YYY$
\begin{align*}
  \pr{y\gets\YYY_0:y=y_0}=\pr{s\gets\SSS:s\in\SSS_{y_0}}\enspace.
\end{align*}
Secondly, let $\SSS_y$ denote also a conditional distribution that
emerges if we restrict the set of outputs to the domain $\SSS_y$, that
is, for all $s_0\in\SSS$ and $y\in\YYY$:
\begin{align*}
  \pr{s\gets\SSS_y:s=s_0}=\pr{s\gets \SSS: s=s_0|g(s)=y}\enspace.
\end{align*}
Then the sampling procedure $s\gets\SSS$ can be rewritten in two steps
\begin{align*}
  \begin{game}{\GAME_{*}^\AD}
    &y\gets\YYY_0\\
    &s\gets\SSS_y\\
    &x\gets f(s)\\
    &y\gets\AD(x)\\
    &\RETURN [y\iseq g(s)]
  \end{game}
\end{align*}
so that the success probability does not change for any adversary
$\AD$. Indeed
\begin{align*}
  \pr{\GAME^{\AD}=1}%
  &= \sum_{y\in\YYY}\pr{s\gets\SSS: g(s)=y_0}\pr{\GAME^{\AD}=1|g(s)=y_0}\\
  &= \sum_{y\in\YYY}\pr{y\gets\YYY_0: y=y_0}\pr{\GAME_{y_0}^{\AD}=1}
\end{align*}
where the game $\GAME_{y_0}$ is defined as follows
\begin{align*}
  \begin{game}{\GAME_{y_0}^\AD}
    &s\gets\SSS_{y_0}\\
    &x\gets f(s)\\
    &y\gets\AD(x)\\
    &\RETURN [y\iseq g(s)]
  \end{game}  
\end{align*}
To be punctual, one has to use the total probability formula and the
definition of conditional probabilities to formally prove
\begin{align*}
 \pr{\GAME^{\AD}=1|g(s)=y_0}\equiv  \pr{\GAME_{y_0}^{\AD}=1}\enspace
\end{align*}
but this is a trivial exercise that is left to the reader. 

\paragraph{\textsc{Choosing between several simple hypothesis}.}
Already a superficial inspection of the game $\GAME_*$ reveals that an
adversary $\AD$ must choose between multiple simple hypotheses
$\HHH_{y_0}=[y\iseq y_0]$. As a next step, we can express 
\begin{align*}
  \pr{\GAME_*^\AD=1}%
  &=\pr{y\gets \YYY_0:y=y_1}\pr{\GAME_{y_1}^\AD=1}\\
  &\quad+\sum_{y_0\in\YYY\setminus y_1}\pr{y\gets \YYY_0:y=y_0}\pr{\GAME_{y_0}^\AD=1}
\end{align*}
where $y_1$ is the most probable index element. Now note that
\begin{align*}
  \pr{y_1}\pr{\GAME_{y_1}^\AD=1}&=
  \pr{y_1}-\pr{y_1}\cdot\sum_{y_0\in\YYY\setminus y_1}\pr{s\gets\SSS_{y_1}:\AD(f(s))=y_0}\\
  &\leq \pr{y_1}-\sum_{y_0\in\YYY\setminus y_1}\pr{y_0}\pr{s\gets\SSS_{y_1}:\AD(f(s))=y_0}
\end{align*}
since $\pr{y_0}\leq\pr{y_1}$ by the choice of $y_1$.
Consequently,
\begin{align*}
  \pr{\GAME_*^\AD=1}%
  &\leq \pr{y_1}+\sum_{y_0\in \YYY\setminus y_1}%
  \pr{y_0}
    \abs{\pr{
    \begin{aligned}
      &s\gets\SSS_{y_0}:\\
      &\AD(\cdot)=y_0
    \end{aligned}}%
  -\pr{
    \begin{aligned}
      &s\gets\SSS_{y_1}:\\
      &\AD(\cdot)=y_0
    \end{aligned}}}\\
   &\leq \pr{y_1}+
    \max_{y_0\in\YYY}
    \abs{\pr{
    \begin{aligned}
      &s\gets\SSS_{y_0}:\\
      &\AD(\cdot)=y_0
    \end{aligned}}%
  -\pr{
    \begin{aligned}
      &s\gets\SSS_{y_1}:\\
      &\AD(\cdot)=y_0
    \end{aligned}}}\enspace.
\end{align*}

\paragraph{\textsc{From complex hypotheses to simple hypotheses}.}
Note that in terms of $\SSS$ the last term in the upper bound
obtained above is very close to a computational distance between
complex hypotheses $[s\gets\SSS_{y_0}]$ and $[s\gets\SSS_{y_1}]$. 

In fact, if we know the maximising value $y_0$, then we can convert a
$t$-time algorithm $\AD$ that maximises the term into $2t$-time
distinguisher $\ADB:\XXX\to\set{0,1}$. As $y_0$ can be at most
$t$-bits long, we can test $[\AD(f(s))\iseq y_0]$ in $t$-time and
consequently we can build $2t$-time algorithm $\ADB$ such that
\begin{align*}
  &p_0\doteq \pr{s\gets\SSS_{y_0}:\ADB(f(s))=1}=\pr{s\gets\SSS_{y_0}:
    \AD(f(s))=y_0}\enspace,\\
  &p_1\doteq\pr{s\gets\SSS_{y_1}:\ADB(f(s))=1}=\pr{s\gets\SSS_{y_1}:
    \AD(f(s))=y_0}\enspace.
\end{align*}
Since
\begin{align*}
  p_0%
  &=\sum_{s_0\in S_{y_0}\atop s_1\in S_{y_1}}\pr{s\gets\SSS_{y_0}:s=s_0}\pr{s\gets\SSS_{y_1}:s=s_1}\pr{\ADB(f(s_0))=1}\\
  p_1 %
  &=\sum_{s_0\in S_{y_0}\atop s_1\in
    S_{y_1}}\pr{s\gets\SSS_{y_0}:s=s_0}\pr{s\gets\SSS_{y_1}:s=s_1}\pr{\ADB(f(s_1))=1}
\end{align*}
we can estimate
\begin{align*}
  \abs{p_0-p_1}
  &=\Biggl|\sum_{s_0\in S_{y_0}\atop s_1\in S_{y_1}}\pr{s_0}\pr{s_1}\brak{\pr{\ADB(f(s_0))=1}-\pr{\ADB(f(s_1))=1}}\Biggr|\\
  &\leq\sum_{s_0\in S_{y_0}\atop s_1\in S_{y_1}}\pr{s_0}\pr{s_1}\abs{\pr{\ADB(f(s_0))=1}-\pr{\ADB(f(s_1))=1}}\\
  &\leq\max_{s_0\in S_{y_0}\atop s_1\in S_{y_1}}\abs{\pr{\ADB(f(s_0))=1}-\pr{\ADB(f(s_1))=1}}\\
  &\leq\max_{s_0,s_1\in\SSS}\CD_x^{2t}([s\iseq s_0],[s\iseq s_1])\leq \varepsilon\enspace.
\end{align*}

\paragraph{\textsc{The final step}.}
To summarise, we have obtained
\begin{align*}
  \pr{\GAME^\AD=1}&=\pr{\GAME_*^\AD=1}\leq \max_{y_0\in\YYY}\pr{y\gets\YYY_0:y=y_0}+\varepsilon\\
  &\leq\max_{y_0\in\YYY}\pr{s\gets \SSS:g(s)=y_0}+\varepsilon\enspace. 
\end{align*}
\qed
\end{proof}

\section{Final Remarks}
Note that the proof given above is strictly non-constructive and does
not show how to convert a good predictor of $g$-values into a good
distinguisher of hidden states $s_0,s_1\in\SSS$. Non-constructivity
of the proof is simultaneously the main strength and weakness of this
approach. Theorem~\ref{thm:ind-sem} is extremely powerful as a
mathematical claim, since it assumes nothing from the sample
distribution $\SSS$ and holds for all functions $g$. As a result,
non-constructivity is essential in the proof, since we cannot assume
that elements of $\SSS$ can be \emph{efficiently} sampled neither we
can assume that the function $g$ is \emph{efficiently} computable.

To be precise, non-constructivity enters into the proof in three
places. First, we fix random coins $\omega$ so that $g(\cdot,\omega)$
behaves better than $g(\cdot)$.  Second, we fix a most probable output
$y_1$ and sub-distributions $\SSS_{y}$ for $y\in\YYY$. Third, we use
two proof steps for finding an output $y_0\in\YYY$ and two states
$s_0\in\SSS_{y_0}$, $s_1\in\SSS_{y_1}$ that maximises the difference
$\abs{\pr{\AD(f(s_0))=y_0}-\pr{\AD(f(s_1))=y_0}}.$

For a fixed distribution $\SSS$ and a fixed function $g(\cdot)$, the
complexity of these non-constructive steps is irrelevant. For a
hypothetical $t$-time algorithm $\AD$ that achieves
$\advSemXX{f,g}{\AD}>\varepsilon$, there exist states
$s_0,s_1\in\SSS$ and a corresponding $2t$-time algorithm $\ADB$ such
that $\advIndXX{s_0,s_1}{\ADB}\geq \varepsilon$ and consequently the
existence of such a $t$-time algorithm $\AD$ directly contradicts the
security premise.

However, the efficiency is important if the distribution $\SSS$ is
not fixed beforehand and a contradiction triple $(\ADB, s_0,s_1)$ must
be discovered on the fly. For instance, Theorem~\ref{thm:ind-sem} does
not cover semantic security of ciphertexts in the settings, where an
adversary can influence which messages are encrypted. The latter is
not so far-fetched assumption. For instance, communication in a war is
mostly about the adversarial behaviour and thus clearly controllable
by adversary.

To prove security in such settings, we have to weaken the claim so
that the construction in the proof would become efficiently
constructable. In particular, the distribution $\SSS$ must be
efficiently samplable and $g$ efficiently computable. Then we can
convert the original non-constructive proof into a constructive
reduction. The details of this approach are thoroughly discussed
in~\cite{FOCS1997EXT:Bellare1997}.

\bibliography{references}

\end{document}
