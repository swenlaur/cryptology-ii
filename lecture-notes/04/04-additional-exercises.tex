\documentclass{article}
\usepackage{crysymb}
\usepackage[draft]{crygame}
\usepackage{crypto-ii}
\usepackage{graphics}
\usepackage{graphicx}

\newcommand{\PKC}{\mathscr{C}}

\begin{document}


\noindent	
MTAT.07.003 Cryptology II\\
Spring 2010 / Exercise Session IV / Additional Exercises 

\begin{enumerate}

\item Pseudorandom permutation family $\FFF$ can be converted into a
  pseudorandom generator by choosing a function $f\getsu \FFF$ and
  then using the counter scheme
  $\textsc{Ctr}_f(n)=f(0)\|f(1)\|\ldots\|f(n)$. Alternatively, we can
  use the following iterative output feedback $\textsc{Ofb}_f(n)$
  scheme
  \begin{align*}
    c_1\gets f(0),c_2\gets f(c_1),\ldots, c_n\gets f(c_{n-1})\enspace,
  \end{align*}
  where $c_1,\ldots,c_n$ is the corresponding output. In both cases,
  the function $f$ is the seed of the pseudorandom function.  Compare
  the corresponding security guarantees. Which of them is better if we
  assume that $\FFF$ is $(n,t,\varepsilon)$-pseudorandom permutation
  family?

  \textbf{Hint:} To carry out the security analysis, formalise the
  hypothesis testing scenario as a game pair and then gradually
  convert one game to another by using the techniques introduced in
  Exercise Session IV.  Pay a specific attention to the cases when
  $c_i=c_{i+k}$ for some $k>0$.



\item [$(\star)$] The counter mode converts any pseudorandom function
  into a pseudorandom generator. Give a converse construction that
  converts any pseudorandom generator into a pseudorandom
  function. Give the corresponding security proof together with
  precise security guarantees.

  \textbf{Hint:} Use a stretching function
  $f:\set{0,1}^n\to\set{0,1}^{2n}$ to fill a complete binary tree with
  $n$-bit values.





\item Let $\FFF$ be a $(t,q,\varepsilon)$-pseudorandom function family
  that maps a domain $\MSPACE$ to the range $\CSPACE$. Let
  $g:\MSPACE\to\set{0,1}$ be an arbitrary predicate. What is the
  success probability of a $t$-time adversary $\AD$ in the following
  games?
  \begin{align*}
    \begin{game}{\GAME_0^\AD}
      &m\getsu \MSPACE\\
      &f\getsu \FFF\\
      &c\gets f(m)\\
      &\RETURN [\AD(c)\iseq m]
    \end{game}
    &&
    \begin{game}{\GAME_1^\AD}
      &m\getsu \MSPACE\\
      &f\getsu \FFF\\
      &c\gets f(m)\\
      &\RETURN [\AD(c)\iseq g(m)]
    \end{game}
 \end{align*}
 Establish the same result by using the IND-SEM theorem. More
 precisely, show that the hypothesis testing games
 \begin{align*}
    \begin{game}{\GAME_{m_0}^\AD}
      &f\getsu \FFF\\ 
      &c\gets f(m_0)\\
      &\RETURN \AD(c)
    \end{game} 
     &&
    \begin{game}{\GAME_{m_1}^\AD}
      &f\getsu \FFF\\ 
      &c\gets f(m_1)\\
      &\RETURN \AD(c)
    \end{game} 
 \end{align*}
 are $(t,2\varepsilon)$-indistinguishable for all
 $m_0,m_1\in\MSPACE$.

\item Feistel cipher
  $\textsc{Feistel}_{f_1,\ldots,f_k}:\set{0,1}^{2n}\to\set{0,1}^{2n}$
  is a classical block cipher construction that consists of many
  rounds.  In the beginning of the first round, the input $x$ is split
  into two halves such that $L_0\|R_0=x$. Next, each round uses a
  random function $f_i\gets\FFFALL$ to update both halves:
  \begin{align*}
    L_{i+1}\gets R_i\qquad\text{and}\qquad R_{i+1}\gets L_i\oplus f_i(R_i)\enspace.
  \end{align*}
  The output of the Feistel cipher $
  \textsc{Feistel}_{f_1,\ldots,f_k}(L_0\|R_0)= L_k\|R_k$.
  \begin{enumerate}
  \item Show that the Feistel cipher is indeed a permutation.
  \item Show that the two-round Feistel cipher
    $\textsc{Feistel}_{f_1,f_2}(L_0\|R_0)$ where $f_1,f_2\gets\FFFALL$
    is not a pseudorandom permutation. Give a corresponding
    distinguisher that uses two encryption queries.
  \item Show the three-round Feistel cipher
    $\textsc{Feistel}_{f_1,f_2,f_3}(L_0\|R_0)$ where
    $f_1,f_2,f_3\gets\FFFALL$ is a pseudorandom permutation. For the
    proof, note that the output of the three round Feistel cipher can
    be replaced with uniform distribution if $f_2$ and $f_3$ are
    always evaluated at distinct inputs. Estimate the probability that
    the $i$th encryption query creates the corresponding input
    collision for $f_2$.  Estimate the probability that the $i$th
    encryption query creates an input collision for $f_3$.
 
  \item[($\bullet$)] Show that the tree-round Feistel cipher
    $\textsc{Feistel}_{f_1,f_2,f_3}(L_0\|R_0)$ is not pseudorandom if
    the adversary can also make decryption queries.
  \item [($\star$)] Show that the four-round Feistel cipher
    $\textsc{Feistel}_{f_1,f_2,f_3,f_4}(L_0\|R_0)$ where
    $f_1,f_2,f_3,f_4\gets\FFFALL$ is indistinguishable from $\FFFPERM$
    even if the adversary can make also decryption calls.
  \end{enumerate}
\item[$(\star)$] Note that exercises above and the PRP/PRF swithing
  lemme give a circular constructions: PRP $\Rightarrow$ PRF
  $\Rightarrow$ PRF, PRF $\Rightarrow$ PRG $\Rightarrow$
  PRF. Consequently, the existence assumptions for pseudorandom
  permutations, pseudorandom functions and pseudorandom generators are
  equivalent. However, the equivalence of existence assumptions is
  only quantitative.
  \begin{enumerate}
  \item Analyse the tightness of all constructions. More precisely,
    start with a certain primitive, do the full cycle and analyse how
    much the resulting degradation of efficiency and security
    guarantees. Interpret the results: which existence assumptions is
    the must powerful.
  \item Give a direct circular construction: PRP $\Rightarrow$ PRG
    $\Rightarrow$ PRG that is better than combined construction over
    PRF or show that both combined construction are optimal.
  \end{enumerate}
\end{enumerate}

\end{document}
