\documentclass{article}
\usepackage{crysymb}
\usepackage[draft]{crygame}
\usepackage{crypto-ii}
\usepackage{graphics}
\usepackage{graphicx}

\newcommand{\PKC}{\mathscr{C}}

\begin{document}


\noindent	
MTAT.07.003 Cryptology II\\
Spring 2010 / Exercise Session V 

\begin{enumerate}

\item Recall that the message space of the ElGamal cryptosystem is a
  $(t,\varepsilon_1)$-DDH group $\GG$. The latter is rather limiting,
  since normally one needs to encrypt $n$-bit messages and not the
  group elements. The simplified ElGamal cryptosystem is defined as
  follows:
  \begin{itemize}
  \item $\GEN$ returns $\SK=x$ and $\PK=y=g^x$ for $x\getsu\ZZ_{\abs{\GG}}$;
  \item $\ENC_\PK(m)=(g^k,h(y^k)\oplus m)$; 
  \item $\DEC_\SK(c_1,c_2)=c_2\oplus h(c_1^x)$;
  \end{itemize}
  where $h:\GG\to\set{0,1}^n$ is a almost regular hash function. That
  is, the distribution $h(y)$ for $y\getsu\GG$ is statistically
  $\varepsilon_2$-close to the uniform distribution over
  $\set{0,1}^n$. Prove that the simplified ElGamal cryptosystem is
  also IND-CPA secure and give the corresponding security bounds.


  \textbf{Hint:} Modify the security proof for the ElGamal
  cryptosystem to accommodate the change. Where do you need almost
  regularity?
  \begin{itemize}
  \item[$(\star)$] In practice, it is difficult if not impossible to
    define almost regular hash function $h:\GG\to\set{0,1}^n$. Relax
    the security requirements even further so that the corresponding
    construction is also practical.
  \end{itemize}




\item Prove the only non-trivial inclusion result for homological
  classification of public key cryptosystems. What about symmetric key
  cryptosystems?  
 
  \textbf{Theorem}. Assume that $\pi(\cdot)$ is always a $t_\pi$-time
  predicate and it is always possible to obtain a sample from
  $\MSPACE_0$ in time $t_m$. If the cryptosystem $\PKC$ is
  $(t,\varepsilon)$-IND-CCA2 secure, then for all $(t-t_g-2t_m)$-time
  adversaries $\AD$:
  \begin{align*}
  \advNMCCAIIXX{\PKC}{\AD}\leq\varepsilon\enspace.
  \end{align*}
  Why does not a similar proof exists for
  IND-CCA1$\Rightarrow$NM-CCA1?

\item[$(\star)$] In a fixed plaintext attack (FPA), an adversary has
  to fix the queried messages ahead of other
  interactions. Consequently, it might be possible to achieve a
  security goal against fixed plaintext attacks (CPA) that is
  infeasible against chosen ciphertext attacks. This separation
  manifests already if we consider indistinguishability as a security
  goal. Recall that a cryptosystem is $(t,\varepsilon)$-IND-FPA if for
  all $t$-time adversaries
  \begin{align*}
    \advINDFPAXX{}{\AD}=\abs{\pr{\smash{\GAME_0^\AD=1}}-\pr{\smash{\GAME_1^\AD=1}}}\leq\varepsilon
  \end{align*}
  where
  \begin{align*}
    \begin{game}{\GAME_0^\AD}
      &(m_0,m_1)\gets\AD\\
      &(\SK,\PK)\gets\GEN\\
      &\RETURN \AD(\ENC_\PK(m_0))
    \end{game}
    \qquad
    \begin{game}{\GAME_1^\AD}
      &(m_0,m_1)\gets\AD\\
      &(\SK,\PK)\gets\GEN\\
      &\RETURN \AD(\ENC_\PK(m_1))
    \end{game}
  \end{align*}
  \begin{enumerate}
  \item Prove that if the message space $\MSPACE=\set{0,1}$ then
    IND-FPA security implies IND-CPA security. Generalise the proof
    and show that IND-FPA security also implies IND-CPA security of
    encryption tuples
    \begin{align*}
      \overline{\ENC}_\PK(x_1,\ldots,x_n)=\bigl(\ENC_\PK(x_1),\ldots,\ENC_\PK(x_n)\bigr)
      \enspace.
    \end{align*}
  \item Give a corresponding construction that converts any IND-FPA
    secure encryption scheme to the IND-CPA secure encryption
    scheme. What is the corresponding overhead in communication and
    computation if we want to preserve the size of the message space?
  \item Prove that IND-FPA security still implies IND-CPA security for
    larger message spaces; however, the IND-CPA advantage can be
    $\Oh(\abs{\MSPACE})$ times larger than the IND-FPA advantage. What
    is the optimal trade-off point for construction given in~(b).
  \item Finally, prove that the reduction result obtained in~(c) is
    tight. For that, give a construction that takes in an IND-CPA
    secure cryptosystem and creates a new cryptosystem that is
    functional and IND-FPA secure but the IND-CPA advantage meets the
    bound derived in~(d). If the bounds do not mach exactly, the
    reduction given in~(c) might be non-optimal. Hence, it is
    important to study the relation between the upper and the lower
    bound on advantage obtained in~(c)~and~(d).
  \end{enumerate}


\item Show that the Goldwasser-Micali cryptosystem is IND-CPA secure
  if the Quadratic Residuosity Problem is hard. All necessary concepts
  are defined below. The proof is similar to the analysis of the
  ElGamal cryptosystem.

  \textbf{Number theory.} A prime $p$ is a Blum prime if $p\equiv
  3\mod 4$. Let $N=pq$ where $p,q$ are Blum primes. Then for each
  element $a\in\ZZ_N$, we can efficiently compute the Jacobi symbol
  $(\frac{a}{n})$. One can show that Jacobi symbols satisfies
  following equations
  \begin{align*}
    \brak{\frac{ab}{n}}=\brak{\frac{a}{n}}\cdot\brak{\frac{b}{n}}
    \qquad\text{and}\qquad
    \brak{\frac{a^2}{n}}=1\enspace.
  \end{align*}
  In the following, we also need a set 
  \begin{align*}
   J_N(1)=\set{x\in\ZZ_N:\brak{\frac{x}{n}}=1}\enspace.   
  \end{align*}
  Finally, recall that an element $b$ is a quadratic residue if there
  exists $a$ such that $b=a^2\mod N$.  The set of quadratic residues
  is denoted by $QR_N$.

  \textbf{Quadratic residuosity problem.}  Let $\PP_n$ denote uniform
  distribution over $n$-bit Blum primes.  We say that the set of
  $n$-bit Blum primes is $(t,\varepsilon)$-secure with respect to
  quadratic residuosity problem if for all $t$-time adversaries $\AD$:
  \begin{align*}
    \advQRPXX{\PP_n}{\AD}=\abs{\pr{\smash{\BGAME_0^\AD=1}}-\pr{\smash{\BGAME_0^\AD=1}}}\leq\varepsilon
  \end{align*}
  where
  \begin{align*}
    \begin{game}{\BGAME_0^\AD}
      &p,q\getsu\PP(n)\\
      &N\gets pq\\ 
      &x\getsu QR_N\\
      &\RETURN \AD(x)
    \end{game}
    \qquad\qquad
       \begin{game}{\BGAME_1^\AD}
      &p,q\getsu\PP(n)\\
      &N\gets pq\\ 
      &x\getsu J_N\setminus QR_N\\
      &\RETURN \AD(x)
    \end{game}
  \end{align*}

   
  \textbf{Goldwasser-Micali cryptosystem.}
  \begin{itemize}
  \item \textbf{Key generation.}  Sample primes $p,q\in\PP(n)$ and
    choose quadratic non-residue $y\in J_N(1)$ modulo $N=pq$. Set
    $\PK=(N,y)$, $\SK=(p,q)$.
  \item\textbf{Encryption.} First choose a random $x\gets\ZZ_N^*$ and then compute
    \begin{align*}
      \enc{0}=x^2\mod N \quad\text{and}\quad \enc{1}=yx^2\mod N.
    \end{align*}
  \item\textbf{Decryption.} Output $0$ if the ciphertext $c$ is
    quadratic residue and $1$ otherwise. The latter is easy if the
    factorisation of $N$ is known.
  \end{itemize}



\item Let $(\GEN,\ENC,\DEC)$ be a public key cryptosystem and
  $\GEN^\circ,\ENC^\circ,\DEC^\circ)$ a symmetric key cryptosystem. Then we can
  define a hybrid cryptosystem.
  \begin{itemize}
  \item \textbf{Key generation.} Run the key generation algorithm
    $\GEN$ and output the corresponding secret and public key pair
    $(\SK,\PK)$.
  \item \textbf{Encryption.} Given a message $m$, generate a session
    key $\SK^\circ\gets\GEN^\circ$ and output a pair
    $c_1\gets\enc{\SK_\circ}$ and $c_2\gets\ENC^\circ_{\SK^\circ}(m)$.
  \item\textbf{Decryption.} To decrypt a ciphertext $(c_1,c_2)$, first
    reconstruct the session key $\SK^\circ\gets\dec{c_1}$ and then
    recover  $m\gets\DEC^\circ_{\SK^\circ}(c_2)$.
  \end{itemize}
  Analyse the IND-CPA security proof for the hybrid encryption scheme.
  \begin{enumerate}
  \item Note that the change in the first proof step does not require
    full IND-CPA security. Derive a minimal reasonable security
    condition for the public key encryption scheme so that the proof
    would still hold. A security condition is reasonable if it
    contains no explicit references to the symmetric cryptosystem and
    would be universal for all symmetric key cryptosystems. To achieve
    that, you might slightly change the construction of hybrid 
    encryption scheme. Interpret the result. 
   
  \item Prove that the same proof construction can be used to shows
    that hybrid encryption scheme preserves IND-CCA1 security.  Derive
    corresponding security guarantees. Generalise results of~(a).

  \item[$(\star)$] Why cannot we use the same proof construction to
    show that hybrid preserves IND-CCA2 security? Give a separation
    that invalidates the first proof step and the entire claim about
    IND-CCA2 security. A separation is construction that takes in
    primitives needed in construct and then modifies them so that new
    primitives still satisfy the premises but the final claim does not
    hold, i.e., no proof can exist.
  \end{enumerate}

\item A cryptosystem is homomorphic if there exists an efficient
  multiplication operation defined over the ciphertext space $\CSPACE$
  such that for any valid encryption $c_1\gets\ENC_\PK(m_1)$ the
  distribution $c_1\cdot \ENC_\PK(m_2)$ coincides with the
  distribution $\ENC_\PK(m_1\otimes m_2)$, where $\otimes$ is a binary
  operation defined over the message space $\MSPACE$. Show that
  \begin{enumerate}
  \item the RSA cryptosystem is multiplicatively homomorphic;
  \item the ElGamal cryptosystem is multiplicatively homomorphic;
  \item the Goldwasser-Micali cryptosystem is XOR homomorphic; 
  \end{enumerate}

\item Prove the following claims about public key cryptosystems
  \begin{enumerate}
  \item A homomorphic cryptosystem cannot be non-malleable.
  \item NM-CPA security implies IND-CPA security.
  \item NM-CCA1 security implies IND-CCA1 security.
  \item NM-CCA2 security implies IND-CCA2 security.
  \end{enumerate}

\item[$(\star)$] Show as many separations among the security
  properties of cryptosystem as you can. For example, show that there
  are IND-CPA secure cryptosystems that are not IND-CCA1 secure.

\end{enumerate}


\end{document}