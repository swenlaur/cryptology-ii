\documentclass{article}
\usepackage{crysymb}
\usepackage[draft]{crygame}
\usepackage{crypto-ii}


\newcommand{\FALL}{\FFF_{\mathrm{all}}}
\begin{document}

\noindent	
MTAT.07.003 Cryptology II\\
Spring 2010 / Exercise session XI


\begin{enumerate}

\item Let $\PROT$ be a sigma protocol such that the challenge $\beta$
  is chosen uniformly from an $n$ element set $\BBB$. Let $\SSS_0$ be
  a $t_0$-time algorithm that simulates perfectly the protocol
  transcript $(\alpha,\beta,\gamma)$ when both parties are honest:
  \begin{align*}
    \pr{(\alpha,\beta,\gamma)\gets\SSS_0}=\pr{\VERIFIER, \PROVER\text{ create } (\alpha,\beta,\gamma)}\enspace.
  \end{align*}
  \begin{enumerate}
  \item Prove that the following black-box simulator
    \begin{align*}
      \begin{fblock}{\SSS^{\VERIFIER_*}}
        &\begin{forblock}{i\in\set{1,\ldots, k}}
        &(\hat{\alpha},\hat{\beta},\hat{\gamma})\gets\SSS_0\\
        &\beta\gets \VERIFIER_*(\alpha)\\        
        &\IF \beta=\hat{\beta}\ \THEN \RETURN\VERIFIER_*(\hat{\gamma})  
        \end{forblock}\\
        &\; \RETURN\mathsf{Failure}
      \end{fblock}
    \end{align*}
    provides perfect simulation when it does not halt with $\bot$:
    \begin{align*}
      \pr{\psi\gets\SSS^{\VERIFIER_*}|\neg \mathsf{Failure}}=\pr{\psi\gets\VERIFIER_*^\PROVER}\enspace.
    \end{align*}
    Establish also the corresponding failure probability
    $\pr{\mathsf{Failure}}$ and compute the total running time of
    $\SSS^{\VERIFIER_*}$ such that $\pr{\mathsf{Failure}}\leq
    \varepsilon$. When are the running times of $\VERIFIER_*$ and
    $\SSS^{\VERIFIER_*}$ comparable?

  \item There is a trade-off between simulation overhead and
    soundness, since the knowledge error can be expressed as
    $\kappa=\frac{1}{\abs{\BBB}}$. Thus, by decreasing the set $\BBB$
    we also increase acceptance probability for malicious provers. To
    compensate the effect, we must sequentially run several instances
    of $\PROT$. Let $\kappa$ be the desired knowledge error and let
    $\varepsilon$ be the desired bound on simulation failure
    $\pr{\mathsf{Failure}}$.  What is the minimal number of rounds we
    need, if we require that $\SSS^{\VERIFIER_*}$ can run only
    $\poly(\log_2 (1/\varepsilon))$ times slower than $\VERIFIER_*$?
  \end{enumerate}
\item One possibility to convert a sigma protocol into a zero
  knowledge proof is to use commitments to fix $\beta$ value before
  $\VERIFIER_*$ receives $\alpha$ value. As a result, we can use the
  following black-box simulation strategy
    \begin{align*}
      \begin{fblock}{\SSS^{\VERIFIER_*}}
        &\PK\gets \GEN\\
        &c\gets\VERIFIER_*(\PK)\\
        &\hat{\beta}\gets\KEXTR^{\VERIFIER_*}(\PK,c)\\ 
        &(\alpha,\hat{\beta},\gamma)\gets\SSS_1(\hat{\beta})\\
        &\RETURN\VERIFIER_*(\alpha,\gamma)\\
      \end{fblock}
      \qquad
      \begin{fblock}{\KEXTR^{\VERIFIER_*}(\PK,c)}
        &\begin{forblock}{i\in\set{1,\ldots,k}}
          &\alpha\gets\SSS_2,d\gets\VERIFIER_*(\alpha)\\ 
          &\beta\gets \OPEN_\PK(c,d)\\
          &\IF\beta \neq \bot \ \THEN\RETURN\beta\\
        \end{forblock}\\
       &\RETURN\bot
      \end{fblock}
    \end{align*}
    where $\SSS_1$ simulates perfectly the protocol messages given
    $\hat{\beta}$ and $\SSS_2$ simulates perfectly the first message
    $\alpha$.
    \begin{enumerate}
    \item Prove that the simulation is perfect when there are no
      simulation failures. The simulation fails if one of two events
      happens: (i) the knowledge extraction fails
      $\KEXTR^{\VERIFIER_*}(\PK,c)=\bot$ and the verifier
      $\VERIFIER_*$ outputs a valid decommitment in the final run;
      (ii) the knowledge extraction succeeds
      $\KEXTR^{\VERIFIER_*}(\PK,c)\neq \bot$ but $\hat{\beta}\neq\OPEN_\PK(c,d)$.
    \item Let $\varepsilon_1=\varepsilon_1(\PK,c)$ be the probability
      that $\VERIFIER_*$ manages to open $c$. Let $\varepsilon$ be the
      desired failure probability for knowledge extraction procedure
      $\KEXTR^{\VERIFIER_*}(\PK,c)$. Find out the corresponding value of $k$.
    \item Let $\Omega_{\mathrm{good}}$ consist of all reachable pairs
      $(\PK,c)$ such that $\varepsilon_1(\PK,c)\geq 2\varepsilon$.
      What is the total simulation failure probability if the value of
      $k$ is chosen according to the bound obtained in the part (b)?
      How many times $\SSS^{\VERIFIER_*}$ is slower than
      $\VERIFIER_*$?
    \item Compare the results with the previous exercise. Can we
      construct a simulator that is only
      $\poly(\log_2(1/\varepsilon))$ slower than $\VERIFIER_*$?

    \end{enumerate}

  \item Many challenge response protocols can be converted to
    zero-knowledge proofs if one can guarantee that the verifier knows
    the final answer.  Witness-indistinguishable sigma protocols can
    be used for that purpose.
    \begin{enumerate} 
    \item As a first example, consider the zero-knowledge proof for
      quadratic non-residuosity presented in the lecture. Let the game
      $\GAME_0$ model the behaviour of honest verifier with $\beta=0$
      and let the game $\GAME_1$ model the behaviour of honest
      verifier with $\beta$. Write down these games under the
      assumption that $v$ is quadratic residue. Simplify the games
      $\GAME_0$ and $\GAME_1$ until you have reached to the same game.

    \item More generally, let $x$ be the public input and $w$ the
      corresponding witness. Also, assume that the challenge $\alpha$
      is computed as $\alpha(\beta;r)$ where $\beta\getsu\set{0,1}$
      and $r\getsu\RSPACE$. Finally, let
      \begin{align*}
        \pok{\beta}{\exists r: \alpha=\alpha(\beta,r)}\equiv
        \pok{r}{\alpha=\alpha(0;r)}\vee \pok{r}{\alpha=\alpha(1;r)}
      \end{align*}
      be the corresponding disjunctive proof. Prove that the
      corresponding protocol transcript coincides for both potential
      witnesses $\beta\in\set{0,1}$ if the input $x$ is incorrectly
      formed:
      \begin{align*}
        \exists r: \alpha=\alpha(0;r)\wedge \exists r: \alpha=\alpha(1;r) \enspace.
      \end{align*}
      Explain also why and unbounded prover can distinguish proves if
      this condition does not hold. What happens if for some incorrect
      input $x$ only one disjunct can be satisfied?
    \end{enumerate}
  \item The construction of sigma protocols for complex relations can
    be cumbersome and inefficient. Hence, it is often advantageous to
    use more simplistic alternatives for tightening challenge-response
    protocols. Let $\alpha=\alpha(\beta;r)$ be the challenge message
    where $\beta\getsu\BBB$ is the expected response and
    $r\getsu\RSPACE$ is the masking randomness. Then verifier can
    prove that he or she knows $\beta$ by revealing $\beta$ and
    $r$. Of course, the prover must commit his or her response before
    receiving $\beta$ and $r$.
    \begin{enumerate}
    \item As an example, consider the challenge-response protocol that
      proves the possession of secret key (ability to decrypt
      messages). Recall that the challenge is computed as
      $\alpha\gets\ENC_\PK(\beta)$ and the response is computed as
      $\beta\gets\DEC_\SK(\alpha)$.  Construct the corresponding
      zero-knowledge proof using the idea explained above.
    \item Construct a simulator $\SSS^{\VERIFIER_*}$ that first
      extracts $\beta$ and then uses this value to pass the proof. Is
      there any conceptual difference compared to the construction
      given in Exercise 2? Can this simulator construction be used for
      any challenge response protocol?
    \item For clarity, assume that the cryptosystem in question is RSA
      with OAEP padding. Would you use the construction based on sigma
      protocol or the construction given above?
    \end{enumerate}

  \item There is an important distinction between the strengthening of
    sigma protocols with commitments and with a coin-flipping protocol
    for $\beta\getsu\BBB$. Namely, the strengthening with
    coin-flipping protocol preserves proof of knowledge property,
    whereas the use of commitments does not. Analyse the soundness
    guarantees of both protocols under the assumption that the
    knowledge bound of a sigma protocol is $\kappa$, i.e., if a prover
    succeeds in the proof with probability strictly more than $\kappa$
    then by generating and analysing all protocol runs, we can extract
    the secret witness $w$.
    \begin{enumerate}
    \item Let $(\GEN,\COM,\OPEN)$ be a perfectly hiding commitment
      scheme such that randomness $r$ used to compute commitments is
      uniformly sampled from a set $\RRR$. Then it is straightforward
      to prove that for any plausible value $\PK$ of public parameters
      sets
      \begin{align*}
        \RRR_{c,m}=\set{r\in\RRR: (c,d)\gets\COM_\PK(m;r)}
      \end{align*}
      have the same number of elements for any $c\in\CCC$ and
      $m\in\MSPACE$. Use this fact to prove that if a malicious prover
      succeeds with probability $\varepsilon$ against the
      zero-knowledge protocol, then there exist possibly inefficient
      prover that succeeds with the same probability against the sigma
      proof. Why is this sufficient for soundness?

      \textbf{Hint:} Consider first the simplest case when
      $\abs{\RRR_{c,m}}=1$.
    \item In brief, a coin-flipping protocol is $\varepsilon$-secure
      against malicious provers if there exists an efficient simulator
      $\SIM$ that given $\beta\getsu\BBB$ simulates the protocol for
      the prover $\PROVER_*$ such that the revealed challenge after
      the coin-flipping protocol is $\beta$ and the output
      distributions of $\PPP_*^{\VERIFIER}$ and $\SIM^{\PPP_*}$ are
      statistically $\varepsilon$-close. Prove that any prover
      $\PROVER_*$ against the zero-knowledge protocol can be converted
      to an adversary $\PROVER_\sigma$ against the sigma protocol so
      that the advantage does not drop and the running time of
      $\PROVER_\sigma$ is comparable to $\PROVER_*$. How is the
      efficiency of $\PROVER_\sigma$ connected to the efficiency of
      the simulator $\SIM$.

      \textbf{Hint:} Note that $\varepsilon$-closeness holds also for
      the provers $\PROVER_{**}$ that run $\PROVER_*$ and then output
      $1$ if the transcript $(\alpha,\beta,\gamma)$ generated by
      $\PROVER_*$ is valid and $0$ otherwise.
    \end{enumerate}



\item Let $\textsc{SquareFree}$ denote all square free integers, i.e.,
  all integers that do not divisible by some prime square. Let
  $\textsc{ProdOfTwoPrimePowers}$ denote all integers in the form
  $p^aq^b$ where $p,q\in\PP$. Then the classical zero-knowledge proof
  $n\in\textsc{RsaModulus}$ consists of following steps.
  \begin{enumerate}
  \item[1.] Prove that $n\in\textsc{SquareFree}$.
  \item[2.] Prove that $n\in\textsc{SquareFree}$ is a product of two prime powers.
  \end{enumerate}
  In the following, we consider only the simplest proof that
  $n\in\textsc{SquareFree}$.
  \begin{enumerate}
  \item Let $\phi(n)=\abs{\ZZ_n^*}$ denote the Euler totient function.
    Prove that if $\gcd(n,\phi(n))= 1$ then $n\in\textsc{SquareFree}$.
    Describe the class of square free numbers such that
    $\gcd(n,\phi(n))\neq 1$.
  \item Let $\textsc{SquareFree}^*$ be the class of all square free
    numbers such that $\gcd(n,\phi(n))= 1$. Construct a
    challenge-response protocol for proving  $n\in\textsc{SquareFree}^*$.
 
    \textbf{Hint:} If $\gcd(n,\phi(n))=1$ and prover knows the
    factorisation, then a prover can efficiently find $n$-th roots.
  \item Let $X_{\mathrm{bad}}=\set{x^n:x\in\ZZ_n^*}$ be the set of
    $n$-th powers. Show that if $\gcd(n,\phi(n))=d>1$ then
    $\abs{X_{\mathrm{bad}}}=\abs{\ZZ_n^*}/d$. Conclude that a
    malicious prover can win the challenge-response protocol with
    probability at most $\frac{1}{d}$.
  \item Construct a corresponding zero-knowledge proof for $n\in\textsc{SquareFree}^*$.
  \end{enumerate}

\end{enumerate}


\end{document}