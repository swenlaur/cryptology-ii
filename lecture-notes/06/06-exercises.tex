\documentclass{article}
\usepackage{crysymb}
\usepackage[draft]{crygame}
\usepackage{crypto-ii}
\usepackage{graphics}
\usepackage{graphicx}


\begin{document}

\noindent	
MTAT.07.003 Cryptology II\\
Spring 2010 / Exercise session VI 


\section*{Formal Security Definition}

Recall that a keyed hash function $h:\MSPACE\times\KSPACE\to\TSPACE$
is a $(t,q,\varepsilon)$-\emph{secure message authentication code} if
any $t$-time adversary $\AD$:
\begin{align*}
  \advMACXX{h}{\AD}=\pr{\GAME^\AD=1}\leq\varepsilon\enspace,
\end{align*}
where the security game is following
 \begin{align*}
  \begin{game}{\GAME^\AD}
    &k\getsu\KSPACE\\
    &\!\!\!\begin{forblock}{i\in\set{1,\ldots,q}}
      &\text{Given } m_i\gets \AD \text{ send }
      t_i\gets h(m_i,k) \text{ back to $\AD$}\\
    \end{forblock}\\
    &(m,t)\gets\AD\\
    &\RETURN [t\iseq h(m,k)]\wedge[(m,t)\notin\set{(m_1,t_1),\ldots,(m_q,t_q)}]
  \end{game}
\end{align*}


\section*{Applications of Message Authetication Codes}

\begin{enumerate}
\item Although a good message authentication code
  $h:\MSPACE\times\KSPACE\to\TSPACE$ protects against impersonation
  and substitution attacks, it does not guarantee security against
  reflection and interleaving attacks.
  \begin{enumerate}
  \item Show that message authentication protocol, where $\PARTY_1$
    sends $m$ and the corresponding authentication tag $t\gets h(m,k)$
    to $\PARTY_2$, is not secure if we want to send several messages.
  \item Construct a protocol for authenticated communication that
    preserves message order and handles bidirectional message
    transfer.
  \item Construct a similar protocol without an internal state. Use
    random nonces $r_i\gets\RSPACE$ to guarantee that messages arrive
    in correct order.
  \item What are the advantages and disadvantages of stateful and
    stateless protocols for authenticated communication?
  \end{enumerate}

\item Let $(\GEN,\ENC,\DEC)$ be a IND-CPA secure symmetric encryption
  scheme and let $h$ be a secure message authentication code with the
  appropriate message and key domains. Show that the following
  protection methods assure IND-CCA2 security:
  \begin{enumerate}
  \item first encrypt and then authenticate
    \begin{align*}
      \begin{fblock}{\mathsf{Auth\text{-}Enc}_{\SK,k}(m)}
      & c_1\gets\ENC_\SK(m)\\
      & c_2\gets h(c_1,k)\\
      & \RETURN (c_1,c_2)  
      \end{fblock}
      \qquad\qquad
      \begin{aligned}       
      \begin{fblock}{\mathsf{Auth\text{-}Dec}_{\SK,k}(c_1,c_2)}
      & \IF c_2\neq h(c_1,k)\ \THEN \RETURN \bot\\ 
      & \ELSE \RETURN \DEC_\SK(c_1)\\
      \end{fblock}\\
      \
      \end{aligned}
    \end{align*}
  \item first authenticate and then encrypt
    \begin{align*}
      \begin{aligned}
      \begin{fblock}{\mathsf{Auth\text{-}Enc}_{\SK,k}(m)}
      & t\gets h(m,k)\\
      & \RETURN \ENC_\SK(m,t)\\
      \end{fblock}\\
      \
      \end{aligned}
      \qquad\qquad
      \begin{aligned}       
      \begin{fblock}{\mathsf{Auth\text{-}Dec}_{\SK,k}(c)}
      & (m,t)\gets\DEC_\SK(c)\\
      & \IF t\neq h(m,k)\ \THEN \RETURN \bot\\ 
      & \ELSE \RETURN m\\
      \end{fblock}\\
      \
      \end{aligned}
    \end{align*}
  \item What are the advantages and drawbacks of both approaches?  Why
    the construction does not generalise to public key cryptosystems?
  \end{enumerate}
\end{enumerate}

\section*{Common Message Authentication Codes}


\begin{enumerate}
\addtocounter{enumi}{2} 


\item A keyed hash function $h:\MSPACE\times\KSPACE\to\TSPACE$ is
  $(t,q,\varepsilon)$-\emph{weakly collision resistant} if any
  $t$-time adversary $\AD$ that makes at most $q$ oracle queries finds
  a collision with probability
  \begin{align*}
    \advWCRXX{h}{\AD}=\pr{\GAME^\AD=1}\leq\varepsilon\enspace
  \end{align*}
  where the security game is defined as follows
  \begin{align*}
    \begin{game}{\GAME^\AD}
      &k\getsu\KSPACE\\
      &\!\!\!\begin{forblock}{i\in\set{1,\ldots,q}}
        & \text{Given }m_i\gets \AD \text{ send } t_i\gets h(m_i,k)
          \text{ back to }\AD.\\
      \end{forblock}\\
      &(m_0,m_1)\gets\AD\\
      &\RETURN [m_0\neq m_1]\wedge[h(m_0,k)=h(m_1,k)]\\
    \end{game}
  \end{align*}

  \begin{enumerate}
  \item Let $h:\MSPACE^*\times\KSPACE_1\to\MSPACE_2$ and
    $f:\MSPACE_2\times\KSPACE_2\to\TSPACE$ be keyed hash functions
    such that $h$ is $(t,q_1,\varepsilon_1)$-weakly collision
    resistant and $f$ is $(t,q_2,\varepsilon_2)$-secure message
    authentication code. Show that the $\textsc{Nmac}$ construction
    \begin{align*}
      \textsc{Nmac}_{f,h}(m,k_1,k_2)=f(h(m,k_1),k_2)
    \end{align*}
    is secure message authentication code.

  \item Analyse the $\textsc{Nmac}$ construction under the assumption
    that that $h$ is $(t,q_1,\varepsilon_1)$-weakly collision
    resistant and $\FFF=\set{f_k}$ where $f_k(x)=f(x,k)$ is
    $(t,q_2,\varepsilon_2)$-pseudorandom function family.


  \item[$(?)$] The $\textsc{Nmac}$ construction is often instantiated
    with a single cryptographic hash function
    $h:\set{0,1}^*\to\set{0,1}^{256}$ by defining
    $f(m,k_1)=h(k_1\|\texttt{42}\|m)$ and
    $g(m,k_2)=h(k_2\|\texttt{13}\|m)$. Is this construction secure?
  \end{enumerate}

  \textbf{Hint:} Write down the corresponding security game. What
  happens if the adversary provides a message $m$ that creates a
  collision $h(m,k)=h(m_i,k)$ as an answer? How probable this event can be?


\item A keyed hash function $h:\MSPACE\times\KSPACE\to\TSPACE$ is
  $\varepsilon_1$-almost universal if for all distinct message pairs
  $m_0\neq m_1$ the collision probability is bounded
  \begin{align*}
    \pr{k\getsu\KSPACE: h(m_0,k)=h(m_1,k)} \leq \varepsilon_1\enspace.
  \end{align*}
  Prove that hybrid-MAC construction
  \begin{align*}
    \textsc{Hyb-Mac}_{f,h}(m,k_1,k_2)= f(h(m,k_1),k_2)
  \end{align*}
  is secure message authentication code if
  $\FFF=\set{f_{k_2}}_{k_2\in\KSPACE_2}$ is
  $(t,q,\varepsilon_2)$-pseudo\-random function family and
  $h:\MSPACE\times\KSPACE_2\to\TSPACE$ is $\varepsilon_1$-almost
  universal. What are the corresponding security guarantees?
    
  \textbf{Hints:} Write down the corresponding security game. Unroll
  the for cycle. Replace $f$ with a random function. Replace $t_i$
  with randomly chosen element of $\TSPACE$ when possible. Most
  importantly, treat the cases when $f$ is evaluated several times at
  the same argument correctly. What is the main difference in the
  security analysis compared to the previous exercise?



\item The polynomial message authentication code is secure only if we
  do not reuse the authentication key. Construct a modified stateful
  authentication code that allows us to use the same key for many
  messages. You can use the AES block cipher as a
  $(t,\varepsilon)$-pseudorandom permutation:
  \begin{enumerate}
  \item use the AES cipher to build hybrid-MAC; 
  \item use the AES cipher to stretch the initial key.
  \end{enumerate}
  Give the corresponding security proofs.

\item Let $\FFF\subseteq\set{f:\MSPACE\to\MSPACE}$ be a pseudorandom
  function family. Then we can use the $\textsc{Cbc-Mac}$ construction
  to stretch the input domain:
  \begin{align*}
    f^{(k)}(m_1,\ldots,m_k)=f(f(\cdots f(f(m_1)+ m_2)+\cdots+m_{k-1})+m_k)\enspace,
  \end{align*}
  provided that $(\MSPACE,+)$ is a commutative group. Prove the
  following facts about $\textsc{Cbc-Mac}$ construction.
  \begin{enumerate}
  \item If $f$ is $(t,q,\varepsilon)$-pseudorandom function, then
    $f^{(k)}:\MSPACE^k\to\MSPACE$ is also pseudorandom function. Find
    the corresponding security guarantees.

    \textbf{Hint:} Write down the corresponding security game and
    simplify the evaluation of $f^{(k)}$ until all intermediate values
    are chosen uniformly from $\MSPACE$. Compute the probability of
    collisions.

  \item Let $f^{(*)}:\MSPACE^*\to\MSPACE$ be a natural extension for
    variable input lengths, i.e.,
    $f^{(*)}(m_1,\ldots,m_k)=f^{(k)}(m_1,\ldots,m_k)$ for any
    $k\in\NN$. Prove that $f^{(*)}$ is not a pseudorandom
    function. Give a corresponding distinguisher that makes only $3$
    oracle calls.

  \item Can we use $\textsc{Cbc-Mac}$ as an message authentication code?
  \end{enumerate}

  \item The hybrid hybrid CBC-MAC construction is following
  \begin{align*}
    \textsc{Hyb-Cbc-Mac}(m,f_1,f_2)=f_2\bigl(f_1^{(*)}(m)\bigr)\quad\text{for}\quad
    f_1\in\FFF_1, f_2\in\FFF_2\enspace,
  \end{align*}
  where $\FFF_1$ and $\FFF_2$ be a pseudorandom permutation
  families. Show that the $\textsc{Hyb-Cbc-Mac}$ construction is
  secure message authentication code even for variable input lengths.
  What is the role of $f_2$?



\end{enumerate}


\end{document}