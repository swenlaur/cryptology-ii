\documentclass{article}
\usepackage{crysymb}
\usepackage[draft]{crygame}
\usepackage{crypto-ii}
\usepackage{graphics}
\usepackage{graphicx}

\renewcommand{\vec}[1]{\boldsymbol{#1}}

\begin{document}


\noindent	
MTAT.07.003 Cryptology II\\
Spring 2010 / Exercise Session VIII



\begin{enumerate}

\item Entity authentication protocols are often used to prove liveness
  of a device or a person. For instance, ATM machines normally ask PIN
  codes several times during long transactions to assure that the
  person is still present. Such liveness proofs can be implemented
  with one-way functions.
  \begin{enumerate}
  \item Let $f:\XXX\to\YYY$ be a one-way function and let $n$ be the
    maximal number of protocol invocations. Then a secret key $\SK$
    can be chosen as a tuple of random values $x_{1},\ldots
    x_{n}\getsu \XXX$ and the corresponding public key $\PK$ as a
    tuple of hash values $f(x_1),\ldots,f(x_n)$.

    Each time when a party wants to prove liveness he or she will
    release non-published sub-key $x_i$.  The proof is successful if
    $f(x_i)\iseq y_i$ where $y_i$ is the $i$th component of the public
    key $\PK$.
  
    Prove that if $f$ is $(t,\varepsilon_1)$-secure one-way function
    and protocols are executed sequentially, then the probability that
    a $t$-time adversary succeeds in the $i$th authentication without
    seeing $x_i$ is at most $\varepsilon$.
  \item Consider a modification of the scheme described above, where
    $f:\XXX\to\XXX$ is a permutation and in the setup phase the party
    computes an hash chain $x_n\getsu\XXX$, $x_i\gets f(x_{i+1})$ for
    $i\in\set{1,\ldots,n-1}$ and publishes $x_0\gets f(x_1)$ as a
    public key.  In liveness proofs, sub-keys $x_i$ are released
    one-by-one as before. The proof is valid if $x_{i-1}=f(x_i)$.

    Prove that $(t,\varepsilon)$-secure one-way permutation protocols
    are executed sequentially, then the probability that a $t$-time
    adversary succeeds in the $i$th authentication without seeing
    $x_i$ is at most $\varepsilon$.
\end{enumerate}
 How large can be the success probability of a $t$-time adversary that
 can attack any of these liveness proofs?
  

\item The sizes of the public and private of the liveness proofs
  described in (a) part of the previous exercise grow linearly wrt to
  the maximal number of invocations. The solution (b) requires a
  linear amount of work wrt $n$ if we do not store the intermediate
  states $x_i$ or some liveness proofs are missing.
  \begin{enumerate}
  \item Show that we can use a $(t,\varepsilon_2)$-collision resistant
    hash function family to compact the public key. Describe the
    corresponding compaction procedure and the resulting
    proof. Estimate the size of proofs and the amount of time needed
    to compute them.

   \textbf{Hint:} Binary trees provide an optimal hashing scheme.
 
 \item Show that we can use $(t,\varepsilon_3)$-pseudorandom function
   family $\FFF$ to compact also the private key $\SK$. Describe the
   corresponding scheme and recompute the security guarantees.
   
   \textbf{Hint:} How to stretch randomness in a most optimal way?
   \end{enumerate}

 \item Consider the following entity authentication protocol proposed
   by Bellare and Rogaway. In the \textsc{Map-1} protocol, parties
   $\PARTY_1$ and $\PARTY_2$ share the secret key $k\getsu\KSPACE$ of
   a $(t,\varepsilon)$-pseudorandom function
   $f:\set{0,1}^*\times\KSPACE\to\TSPACE$. More formally, the induced
   function family $\FFF\doteq\set{f_k}$ is
   $(t,\varepsilon)$-pseudorandom.
  \begin{enumerate}
  \item[1.] $\PARTY_1$ sends a random nonce $r_1\getsu\RSPACE$ to $\PARTY_2$.
  \item[2.] $\PARTY_2$ generates a random nonce $r_2\getsu\RSPACE$ and
    sends the identities $\ID_1,\ID_2$, nonces $r_1,r_2$ and the
    authentication tag $f(\ID_1\|\ID_2\|r_1\|r_2,k)$ to $\PARTY_1$.
  \item[3.] $\PARTY_1$ replies $\ID_1,r_2$ and the authentication tag
    $f(\ID_1\|r_2,k)$ to $\PARTY_2$.
  \end{enumerate}
  Parties $\PARTY_1$ and $\PARTY_2$ halt if the received messages are
  not in correct form. Otherwise, both parties are convinced that they
  are indeed talking with each other. Analyse the security of
  $\textsc{Map-1}$ protocol in the standalone setting, where
  $\PARTY_1$ and $\PARTY_2$ run a single instance of the protocol by
  sending messages through the adversary $\AD$ who can alter, drop or
  insert messages into the conversation. The adversary $\AD$ succeeds
  in deception if both parties reach accepting state but the adversary
  has altered some messages.
  \begin{enumerate}
  \item Formalise the execution of $\textsc{Map-1}$ protocol as a game
    that ends with $1$ iff $\AD$ succeeds in deception. Note that
    $\AD$ does not have o respect the temporal order. For example,
    $\AD$ can transfer $\hat{r}_1$ to $\PARTY_1$ before $\PARTY_1$ has
    released $r_1$.
    Estimate the probability that the adversary $\AD$ sends
    $\hat{r}_1\neq r_i$ to $\PARTY_2$ and still succeeds in deception.
  \item Estimate the probability that the adversary $\AD$ sends
    $(\hat{\ID}_1,\hat{\ID}_2,\hat{r}_1,\hat{r}_2)\neq
    (\ID_1,\ID_2,r_1,r_2)$ to $\PARTY_1$ and still succeeds in
    deception.
  \item Estimate the probability that $\AD$ sends
    $(\hat{\ID}_1,\hat{r}_2)\neq(\ID_1,r_2)$ to $\PARTY_1$ and still
    succeeds in deception.
  \item Summarise the results and give the final bound on deception.
  \end{enumerate}


\item The Kerberos protocol is uses a trusted key generation server
  $\TTP$ to set up shared keys between participants
  $\PARTY_1,\ldots,\PARTY_n$. Let $(\GEN,\ENC,\DEC)$ be a IND-CCA2
  secure symmetric cryptosystem. Then in a setup phase, each party
  $\PARTY_i$ shares a secret key $\SK_i\gets\GEN$ with the trusted
  server $\TTP$. To set up a new session key $\SK_{ij}\gets\GEN$
  between $\PARTY_i$ and $\PARTY_j$, the parties $\PARTY_1$,
  $\PARTY_2$ and $\TTP$ execute the following protocol.
  \begin{enumerate}
  \item[1.] $\PARTY_i$ sends $\ID_i,\ID_j$ and a random nonce
    $r_1\getsu\RSPACE$ to the server $\TTP$.
  \item[2.] $\TTP$ generates a new session key $\SK_{ij}\gets\GEN$ and
    sends back:
    \begin{align*}
      \mathsf{ticket}&\gets\ENC_{\SK_j}(\SK_{ij},\ID_i,\text{expiration
        time})\enspace,\\
      \mathsf{enc\text{-}info}&\gets\ENC_{\SK_i}(\SK_{ij},r_1,\text{expiration
      time},\ID_j)\enspace.
    \end{align*}
  \item[3.] $\PARTY_i$ decrypts $\mathsf{enc\text{-}info}$ creates
    another nonce $r_2\getsu\RSPACE$ and sends $\mathsf{ticket}$ and
    $\ENC_{\SK_{ij}}(\ID_i,r_2)$ to $\PARTY_j$, who replies
    $\ENC_{\SK_{ij}}(r_2)$.
  \end{enumerate}
  Participants halt if some messages are not in expected form.  An
  adversary $\AD$ succeeds in deception if either $\PARTY_1$ or
  $\PARTY_2$ reach the accepting state but one of them has a
  fraudulent output.
  \begin{enumerate}
  \item Estimate the probability that $\PARTY_i$ accepts altered
    $\mathsf{enc\text{-}info}$.
  \item Estimate the probability that $\PARTY_j$ accepts altered $\mathsf{ticket}$.
  \item Estimate the probability that $\PARTY_j$ halts but $\PARTY_i$
    accepts.
  \item Give the final bound on the deception probability. 
  \end{enumerate}

\item Let $(\GEN,\ENC,\DEC)$ be $(t,\varepsilon)$-IND-CCA2 secure
  cryptosystem such that the message space $\MSPACE$ is an additive
  group. Then the classical challenge-response protocol for proving
  the possession of  $\SK$ is following:
  \begin{enumerate}
  \item[1.] The verifier $\VERIFIER$ chooses $m\getsu\MSPACE$ and sends
    $\ENC_\PK(m)$ to the prover $\PROVER$.
  \item[2.] Given a challenge $c$, the prover $\PROVER$ replies
    $\overline{m}\gets\DEC_\SK(c)$.
  \item[3.] The verifier $\VERIFIER$ accepts if $m=\overline{m}$ to
    $\VERIFIER$.
  \end{enumerate}
  However, sometimes one needs to prove that he or she possesses two
  different secret keys or only one of them. The corresponding proofs
  are known as conjunctive and disjunctive proofs.

  \textbf{Conjunctive proof for secret keys $\SK_0$ and $\SK_1$:}
  \begin{enumerate}
  \item[1.] The verifier $\VERIFIER$ chooses $m_0,m_1\getsu\MSPACE$
    and sends challenge $\ENC_{\PK_0}(m_0)$ and $\ENC_{\PK_1}(m_1)$ to
    the prover $\PROVER$.
  \item[2.] Given challenge ciphertexts $c_0,c_1$, the prover $\PROVER$
    uses both secret keys $\SK_0$ and $\SK_1$ and replies
    $\overline{m}_0\gets\DEC_{\SK_0}(c_0)$ and
    $\overline{m}_1\gets\DEC_{\SK_1}(c_1)$ to $\VERIFIER$.
  \item[3.] The verifier $\VERIFIER$ accepts if $m_0=\overline{m}_0$
    and $m_1=\overline{m}_1$.
  \end{enumerate}

  \textbf{Disjunctive proof for secret keys $\SK_0$ and $\SK_1$:}
  \begin{enumerate}
  \item[1.] The verifier $\VERIFIER$ chooses $m\getsu\MSPACE$ and
    sends the corresponding challenge $\ENC_{\SK_0}(m;r_0)$ for
    $r_0\gets\RSPACE$, and $\ENC_{\SK_1}(m;r_1)$ for $r_1\gets\RSPACE$
    together with encryptions of random nonces $\ENC_{\SK_0}(r_1)$ and
    $\ENC_{\SK_1}(r_0)$ to $\PROVER$.
  \item[2.]  Given challenge ciphertexts $c_1,c_2,c_3,c_4$, the prover
    $\PROVER$ uses one of the secret keys $\SK_i$ to decrypt a
    challenge $\overline{m}$ and the nonce $r_{\neg i}$ used to
    randomise the other encryption $c_{\neg i}$. If $c_{\neg i}=
    \ENC_{\PK_{\neg i}}(\overline{m};r_{\neg i})$, the prover
    $\PROVER$ sends $\overline{m}$ to $\VERIFIER$, otherwise $\PROVER$
    can halt as $\VERIFIER$ cheats.
  \item[3.] The verifier $\VERIFIER$ accepts if $\overline{m}=m$.
  \end{enumerate}
  Consider a simple standalone setting, where a prover and a verifier
  execute a protocol to be analysed only once in isolation and prove
  the following facts about conjunctive and disjunctive proofs.
  \begin{enumerate}
  \item A prover can succeed in conjunctive proof only if he or she
    knows both secret keys and a prover fails in disjunctive proof if
    he or she does not know neither of the secret keys.
  \item Even a malicious verifier cannot reliably detect which secret
    key is known by the honest prover.
  \end{enumerate}
\item Conjunctive and disjunctive proofs of possession can be used as
  a building-blocks for more complex relations. 
  \begin{enumerate}
  \item Construct a proof of possession, where the prover has to have
    at least two secret keys out of three.
  \item Generalise this construction and show that for any monotone
    formula $\phi:\set{0,1}^n\to\set{0,1}$. More precisely, let the
    $i$th bit of $x\in\set{0,1}^n$ denote whether a prover has a
    secret key $\SK_i$. Show that there exists a proof of possession,
    where a prover succeeds only if $\phi(x)=1$.
  \end{enumerate}
\end{enumerate}
\end{document}
