\documentclass{article}
\usepackage{crysymb}
\usepackage[draft]{crygame}
\usepackage{crypto-ii}


\newcommand{\FALL}{\FFF_{\mathrm{all}}}
\begin{document}

\noindent	
MTAT.07.003 Cryptology II\\
Spring 2010 / Exercise session X


\section*{Examples of Sigma Protocols}

\begin{enumerate}

\item The Guillou-Quisquater identification scheme (GQ scheme) is
  directly based on the RSA problem. The identification scheme is a
  honest verifier zero-knowledge proof that the prover knows $x$ such
  that $x^e=y\mod n$ where $n$ is an RSA modulus, i.e., the public
  information $\PK=(n,e,y)$ and the secret is $x$. The protocol itself
  is following:
  \begin{enumerate}
  \item[1.] $\PROVER$ chooses $r\getsu\ZZ_n^*$ and sends $\alpha\gets r^e$ to $\VERIFIER$.
  \item[2.] $\VERIFIER$ chooses $\beta\getsu\set{0,1}$ and sends it to $\PROVER$.
  \item[3.] $\PROVER$ computes $\gamma\gets r x^\beta$ and sends it to $\VERIFIER$.
  \item[4.] $\VERIFIER$ accepts the proof if $\gamma^e=\alpha y^\beta$.    
  \end{enumerate}
  Prove that the Guillou-Quisquater identification scheme is sigma
  protocol.
  \begin{enumerate}
  \item The GQ identification scheme is functional.
  \item The GQ identification scheme has the zero-knowledge property.
  \item The GQ identification protocol is specially sound. 
  \item Amplify soundness guarantees with parallel and sequential
    composition and derive the corresponding knowledge bounds.
  \end{enumerate}

\item Let $\GG$ be a cyclic group with prime number of elements $q$
  and let $g_1$ and $g_2$ be generators of the group. Now consider a
  sigma protocol for proving the knowledge of $x$ such that
  $g_1^x=y_1$ and $g_2^x=y_2$, i.e., the public information is
  $(g_1,g_2,y_1,y_2)$ and the secret knowledge is $x$. The protocol is
  following:
  \begin{enumerate}
  \item[1.] $\PROVER$ chooses $r\getsu\ZZ_q$ and sends $\alpha_1\gets
    g_1^r$ and $\alpha_2\gets g_2^r$ to $\VERIFIER$.
  \item[2.] $\VERIFIER$ chooses $\beta\getsu\ZZ_q$ and sends it to $\PROVER$.
  \item[3.] $\PROVER$ computes $\gamma\gets x\beta+r$ and sends it to
    the verifier $\VERIFIER$.
  \item[4.] $\VERIFIER$ accepts the proof if $g_1^{\gamma}=\alpha_1
    y_1^\beta$ and $g_2^{\gamma}=\alpha_2 y_2^\beta$.
  \end{enumerate}
  Prove that the protocol is indeed a sigma protocol.
  \begin{enumerate}
  \item The protocol is functional and has the zero-knowledge
    property.
  \item The protocol is specially sound and two colliding transcripts
    indeed reveal $x$ such that $g_1^x=y_1$ and $g_2^x=y_2$.
  \end{enumerate}
  As a concrete application of this protocol construct a proof that
  the ElGamal encryption $(c_1,c_2)$ is an encryption of $\ENC_\PK(1)$.
\end{enumerate}


\section*{Applications of Sigma Protocols}

\begin{enumerate}
  \addtocounter{enumi}{2}
\item Recall that in the first step of certified computations the
  prover $\PROVER$ commits bit by bit to his or inputs $x_1,\ldots,
  x_n$ and uses sigma protocol to prove the validity of
  commitments. Use Pedersen commitments and the Schnorr protocol
  $\pok{x}{y=g^x}$ to implement this strategy.
  \begin{enumerate}
  \item Construct a sigma protocol $\pok{c,g}{\exists r: c=y^xg^r}$.
  \item Construct a sigma protocol $\pok{c,g}{\exists d: \OPEN(c,d)\in\set{0,1}}$.
  \item Use homomorphic properties of the Pedersen commitment to
    construct a sigma protocol for proving $x_1+\cdots+x_n=1$ and
    $x_i\in\set{0,1}$.
  \end{enumerate}

\item In the second phase of certified computations the prover reveals
  commitments to all intermediate values in the Boolean circuit. As in
  the previous exercise use Pedersen commitments and the Schnorr
  protocol $\pok{x}{y=g^x}$ to construct sigma protocols to prove the following facts
  \begin{enumerate}
  \item Values $c_u$ and $c_v$ are the commitments of $u$ and $v$ such that $v=\neg u$.
  \item Commitments $c_u$, $c_v$ $c_w$ of $u$, $v$ and $w$ are such that $w=u\wedge v$.
  \item Commitment $c_f$ of $f$ is such that $f=x_0\wedge \neg x_1 \vee \neg x_3\wedge x_4$. 
  \end{enumerate}

\item Many e-voting protocols use sigma protocols to prove the
  correctness of several crucial steps. In particular, one often needs
  to prove
  \begin{enumerate}
  \item $c$ is a lifted ElGamal encryption of $0$ or $1$;
  \item $c$ is a lifted ElGamal encryption of $x\in\set{0,\ldots 2^\ell}$;
  \item $(c_{ij})_{i,j=1}^n$ is an Pedersen commitment to a permutation matrix. 
  \end{enumerate}
  Use the Schnorr protocol $\pok{x}{y=g^x}$ and properties of lifted ElGamal
  and Pedersen commitments to construct the corresponding sigma
  protocols.


\item[$(\star)$] Let $\GG$ be a cyclic group with prime number of
  elements $q$ as in the previous exercise. Design a sigma proof that
  the prover knows $x_1$ and $x_2$ such that
  $y=g_1^{x_1}g_2^{x_2}$. The latter is often used together with the
  lifted ElGamal encryption $\overline{\ENC}_\PK(x)=\ENC(g^x)$ that is
  additively homomorphic. Construct sigma protocols for the following
  statements.
  \begin{enumerate}
  \item An encryption $c$  is $\overline{\ENC}_\PK(m)$ and $m$ is known or publicly fixed.
  \item An encryption $c_2$ is computed as $c\cdot\ENC_\PK(1)$.
  \item An encryption $c_2$ is computed as $c_1^y\cdot \ENC_\PK(1)$.
  \item An encryption $c_3$ is computed as $c_1\cdot c_2\cdot \ENC_\PK(1)$.
  \end{enumerate}

\item Recall that a generic Schnorr signature
  $(m,\alpha,\beta,\gamma)$ is defined as follows $\alpha\gets g^r$
  for $r\getsu \ZZ_q$, $\beta\gets h(m,\alpha)$ and $\gamma=x\beta+r$
  where $y=g^x$ is the public key of a signer and $x$ is the secret
  key. Consider the security of the Schnorr signature scheme against
  existential forgeries, where the function $h$ is replaced with a
  random oracle $\ORA_h(\cdot)$ that computes uniformly chosen
  function from $\FALL=\set{h:\GG\times\MSPACE\to\ZZ_q}$.  
  \begin{enumerate}
  \item Convert an adversary that makes at most $q_h$ queries to
    random oracle $\ORA_h(\cdot)$ and succeeds with the probability
    $\varepsilon$ in the key only model can be converted to an
    adversary $\AD_*$, which queries each message only once from
    $\ORA_h$ and returns only valid signatures or halts. Show that the
    running times of $\AD$ and $\AD_*$ are comparable and $\AD_*$
    makes at most $q_h+1$ queries.
  \item Convert $\AD_*$ to an adversary $\ADB$ that initiates up to
    $q_h+1$ Schnorr identification protocols and then finishes
    successfully one these identification protocols with the same
    probability than $\AD_*$ succeeds in existential forgery.
  \item Look at the second type of matrix games we considered in the
    lectures and provide the expected number of probes needed to
    extract the secret key from $\ADB$ and $\AD$.

  \item It is common to consider security in the model where adversary
    can use signing oracle up to $g_s$ times. Show that each of the
    queries $\SIGN(m)$ can be simulated by choosing
    $\beta,\gamma\gets\ZZ_q$ and computing $\alpha\gets g^\gamma
    y^{-\beta}$ and then defining $\ORA_h(m,\alpha)=\beta$. Why and
    when is this assignment consistent with the definition of random
    oracle?
 \end{enumerate}

\end{enumerate}

\end{document}
