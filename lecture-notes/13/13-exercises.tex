\documentclass{article}
\usepackage{crysymb}
\usepackage[draft]{crygame}
\usepackage{crypto-ii}


\newcommand{\FFFTP}{\FFF_{\mathrm{tp}}}
\newcommand{\PPPHC}{\PPP_{\mathrm{hc}}}


\begin{document}

\noindent	
MTAT.07.003 Cryptology II\\
Spring 2009 / Exercise session XIII


\begin{enumerate}

\item In the following, we are going to consider the simplified
  version of the oblivious transfer protocol that was proposed by
  Even, Goldreich and Lempel. As usual, let $\PARTY_1$ be the receiver
  with input $b\in\set{0,1}$ and let $\PARTY_2$ be the sender with
  input $x_0,x_1\in\MSPACE$. Additionally, assume that participants
  can use oracle $\ORA_{\mathrm{map}}:\MSPACE\to\MSPACE$ to evaluate a
  random permutation over $\MSPACE$ and the sender $\PARTY_2$ can use
  the corresponding inversion oracle $\ORA_{\mathrm{inv}}$:
  \begin{align*}
    \forall m\in\MSPACE:\quad
    \ORA_{\mathsf{inv}}(\ORA_{\mathrm{map}}(m))\equiv m\enspace.
  \end{align*}
  To transfer the message $x_b$ participants carry out the following
  steps:
  \begin{enumerate}
  \item[1.] $\PARTY_1$ sends $c_0,c_1$ where $c_{1-b}\getsu\MSPACE$ and
    $c_{b}\gets \ORA_{\mathrm{map}}(r)$ for $r\getsu\MSPACE$.
  \item[2.] $\PARTY_2$ replies $d_0\gets\ORA_{\mathrm{inv}}(c_0)\oplus
    x_0$ and $d_1\gets\ORA_{\mathrm{inv}}(c_1)\oplus x_1$.
  \item[3.] $\PARTY_1$ reconstructs $x_b\gets r\oplus d_b$.
  \end{enumerate}
  Prove the protocol is secure in the semi-honest model.
  \begin{enumerate}
  \item The protocol is functional and receiver indeed recovers $x_b$. 
  \item Construct a simulator for semi-honest sender and show
    that the output distributions in the real an ideal world coincide.
  \item Construct a simulator for semi-honest receiver and show that
    the output distributions in the real an ideal world are
    computationally indistinguishable. More precisely, give the bound
    on distinguishing advantage in terms of $\ORA_{\mathrm{map}}$
    queries made by the distinguisher.

    \textbf{Hint:} What is the distinguishing advantage if the
    distinguisher makes no calls to $\ORA_{\mathrm{map}}$ and the
    receiver makes only a single $\ORA_{\mathrm{map}}$ call?
  \end{enumerate}

\item[$(\star)$] The full construction of the oblivious transfer
  proposed by Even, Goldreich and Lempel uses hard-core
  predicates. Let $\FFFTP$ be a collection of trapdoor permutations
  and $\PPPHC$ be the predicate collection with matching domain
  $\MSPACE_\PK$. Then the family $\PPPHC$ forms a
  $(t,\varepsilon)$-hard-core predicate set with respect to $\FFFTP$,
  if for any $t$-time algorithm $\AD$ the corresponding advantage
  \begin{align*}
    \advHCPredXX{\FFFTP\Join\PPPHC}{\AD}=\abs{\pr{\GAME_0^\AD=1}-\pr{\GAME_1^\AD=1}}\leq\varepsilon
  \end{align*}
  where 
  \begin{align*}
    \begin{game}{\GAME_0^\AD}
      &(\PK,\SK)\gets\GEN\\
      &m\getsu\MSPACE_\PK\\
      &b\getsu\set{0,1}\\
      &\RETURN \AD(\PK,m,b) 
    \end{game}
    \qquad
    \begin{game}{\GAME_1^\AD}
      &(\PK,\SK)\gets\GEN\\
      &m\getsu\MSPACE_\PK\\
      &b\gets \mathsf{Pred}_\PK(\INV_\SK(m))\\
      &\RETURN \AD(\PK,m,b) 
    \end{game}
  \end{align*}
  Prove that the standard oblivious transfer protocols for $1$-bit
  messages, where $d_i\gets\mathsf{Pred}_\PK(\INV_\SK(c_i))\oplus
  x_b$, is secure in the standard model. Also, find a reference that
  shows that hard core predicates exist if one-way functions
  exist. Compute the corresponding security guarantees.




\item Recall that the Goldwasser-Micali cryptosystem is additively
  homomorphic over $\ZZ_2$ construct a corresponding oblivious
  transfer protocol based on the Aiello-Ishai-Reingold generic
  construction that was presented in the lecture. Additionally, assume
  that RSA modulus is correctly generated from two $n$-bit Blum primes
  such that the quadratic residuosity problem is
  $(t,\varepsilon)$-hard. That is, for all $t$-time adversaries $\AD$:
  \begin{align*}
    \advQRPXX{\PP_n}{\AD}=\abs{\pr{\smash{\BGAME_0^\AD=1}}-\pr{\smash{\BGAME_0^\AD=1}}}\leq\varepsilon
  \end{align*}
  where
  \begin{align*}
    \begin{game}{\BGAME_0^\AD}
      &p,q\getsu\PP(n)\\
      &N\gets pq\\ 
      &x\getsu QR_N\\
      &\RETURN \AD(x)
    \end{game}
    \qquad\qquad
       \begin{game}{\BGAME_1^\AD}
      &p,q\getsu\PP(n)\\
      &N\gets pq\\ 
      &x\getsu J_N\setminus QR_N\\
      &\RETURN \AD(x)
    \end{game}
  \end{align*}
  Now prove the security of the corresponding oblivious transfer
  protocol directly from the indistinguishability property.

\item The construction of the one out of two oblivious transfer
  protocol proposed by Aiello, Ishai and Reingold can be extended in
  several ways.
  \begin{enumerate}
  \item Construct one out of $n$ oblivious transfer protocol.

    \textbf{Hint:} Construct a protocol that reveals $x$ only if
    $\DEC_\SK(c)=2$.
  
  \item Explain how to transfer long messages that do not fit into the
    message space of the homomorphic encryption by repeating the
    transfer phase. What happens with the security guarantees?
  \item Explain how we can use the underlying principle of hybrid
    encryption and make the protocol more efficient for long
    strings. What happens with the security guarantees?
  \item Describe how we can use the construction given above to
    implement pay-per-view system for sensitive content (satellite pictures).
   \end{enumerate}

 \item Not all protocols used for one out of two oblivious transfer
   are perfect and therefore we need generic techniques for security
   amplification.
   \begin{enumerate}
   \item Consider a leaking ideal oblivious transfer protocol $\PROT$
     for $k$-bit strings, where the trusted third party reveals with
     probability $p$ both inputs $x_0$ and $x_1$ to receiver but with
     probability $1-p$ reveals only $x_b$ as desired. Describe a
     construction that uses several instances of $\PROT$ to securely
     implement one out of two oblivious transfer for $k$-bit strings.

     \textbf{Hint:} Use additive secret sharing to achieve all or
     nothing property.

   \item Consider a defective ideal oblivious transfer protocol
     $\PROT$ for $k$-bit strings, where trusted third party reveals
     $x_b$ with probability $p$ and otherwise halts the protocol
     prematurely. Describe a construction that uses several instances
     of $\PROT$ to securely implement one out of two oblivious
     transfer for $k$-bit strings.

     \textbf{Hint:} What about error correction codes?
   \item Let $\PROT$ be an ideal oblivious transfer protocol for
     $k$-bit strings and let $f:\set{0,1}^k\to\set{0,1}^n$ be a
     $(t,\varepsilon)$-pseudorandom generator.  Construct one out of
     two oblivious transfer protocol for $n$-bit strings.
   \end{enumerate}

 \item Given a protocol $\PROT$ for ideal one out of two oblivious
   transfer, construct a commitment scheme by amplifying security
   properties of the following commitment scheme.
   \begin{enumerate}
   \item[1.] To commit a bit $x$, a sender creates uniformly distributed
     pair $x_0,x_1$ such that $x_0\oplus x_1=x$. To get a commitment
     string $c$, the receiver chooses $b\getsu\set{0,1}$ and uses
     $\PROT$ to retrieve $x_b$.
   \item[2.] To open a commitment, the sender releases both values
     $x_0$ and $x_1$. The commitment is invalid if $c\neq x_b$.
   \end{enumerate}
   \begin{enumerate}
   \item Prove that the commitment scheme is perfectly hiding and $\frac{1}{2}$-binding.
   \item Give a construction that is perfectly hiding and
     $2^{-k}$-binding.
   \item Extend the abovementioned construction for $\ell$-bit
     strings.
   \end{enumerate}
   This result is important as Brassard, Chaum and Cr\'{e}peau proved
   that everything is provable in zero knowledge provided that binding
   and hiding commitment schemes exists. More precisely, let the
   relation between a public parameter $x$ and a secret witness $w$ be
   encoded as a predicate
   \begin{align*}
     \psi(x,w)=1\quad\Leftrightarrow\quad (x,w)\in R\enspace.
   \end{align*}
   Then the corresponding zero-knowledge proof can be computed in time
   $\Theta(\mathrm{size}(\psi))$ where $\mathrm{size}(\phi)$ is the
   the total number of wires, unary and binary logic gates in the
   circuit used to compute the formula $\psi$.
   \begin{enumerate}
   \item[(d)] Conclude that one can implement zero knowledge proofs
     using only the implementations of ideal oblivious transfer. Is
     the latter formally sufficient to conclude that everything can be
     securely computed given a secure implementation of oblivious
     transfer?
   \end{enumerate}
 \item Consider computations in the semi-honest model. Let $\PROT$ be
   a one out of two ideal oblivious transfer protocol for $n$-bit
   strings and let $f:\set{0,1}^n\to\set{0,1}^{2n}$ be a
   $(t,\varepsilon)$-secure pseudorandom generator.  Construct an
   oblivious tree traversal protocol such that a receiver could
   retrieve all inputs stored in the arcs on the path selected by the
   receiver.
   \begin{enumerate}
   \item Implement non-private but functional protocol, where receiver
     uses oblivious transfer to receive the input stored on the
     outgoing arc.

   \item Extend the tree protocol so that it preserves
     receiver-privacy, i.e., the server cannot detect which path the
     receiver chose.

   \item Use the inputs on the arc to encrypt the inputs of the
     following arcs. Conclude that the modified protocol is also
     computationally secure against semi-honest and malicious
     receivers.

   \item Conclude that any function can be computed by simultaneously
     executing enough oblivious transfer protocols. Is this protocol
     efficient for all functions?
 
   \end{enumerate}

 \item[$(\star)$] Let $\PROT$ be a one out of two oblivious transfer
   protocol that is secure against malicious receivers and semi-honest
   senders. Now consider the following simulator for malicious senders
   $\PARTY_2^*$
   \begin{align*}
     \begin{fblock}{\SSS^{\PARTY_2^*}}
       &\omega_1\getsu\Omega_1, \omega_2\getsu\Omega_2\\
       & \text{Let $\hat{x}_0$ be the output of $\PARTY_1(0;\omega_1)$ interacting with $\PARTY_2(\omega_2)$.}\\
       & \text{Let $\hat{x}_1$ be the output of $\PARTY_1(0;\omega_1)$ interacting with $\PARTY_2(\omega_2)$.}\\
       & \text{Submit $\hat{x}_0$ and $\hat{x}_1$ to the trusted third party $\TTP$.}\\
       & \text{Output the end state of  $\PARTY_2(\omega_2)$ interaction with  $\PARTY_1(0;\omega_1)$.}
     \end{fblock}
   \end{align*}
   where $\omega_1\getsu\Omega_1$ is the randomness used by honest
   receiver $\PARTY_1$ and $\omega_2\getsu\Omega_2$ is the randomness
   used by malicious sender $\PARTY_2^*$. Does this construction lead
   to joint output distribution that is computationally or
   statistically indistinguishable from the real world output
   distribution? Prove it or give a counter-example. Interpret the result.

   \textbf{Implicit assumption:} We assume that receiver output always
   some output even if the sender halts or otherwise behaves
   illogically.

   \textbf{Hint:} Consider first the protocol that has a perfect
   simulator for semi-honest sender.

\end{enumerate}
\end{document}
