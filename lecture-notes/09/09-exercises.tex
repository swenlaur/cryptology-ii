\documentclass{article}
\usepackage{crysymb}
\usepackage[draft]{crygame}
\usepackage{crypto-ii}
\usepackage{graphics}
\usepackage{graphicx}

\renewcommand{\vec}[1]{\boldsymbol{#1}}

\newcommand{\FFFTP}{\FFF_{\mathrm{tp}}}
\newcommand{\HASH}{\mathrm{Hash}}


\newcommand{\dindex}[3]{{#1}_{#2}^{\scriptscriptstyle (#3)}}



\begin{document}


\noindent	
MTAT.07.003 Cryptology II\\
Spring 2010 / Exercise Session IX




\begin{enumerate}





\item Consider a following message transmission protocol. A sender
  $\PARTY_1$ knows the public encryption key $\PK_2$ of a receiver
  $\PARTY_2$ and the receiver $\PARTY_2$ knows the public signing key
  $\PK_1$ of the sender $\PARTY_1$. To encrypt a message $m$ the
  sender $\SK$ computes $c\gets\ENC_{\PK_2}(m)$,
  $s\gets\SIGN_{\SK_1}(c)$ and sends $(c,s)$ over unreliable channel
  to $\PARTY_2$. The receiver $\PARTY_2$ first checks the authenticity
  by computing $\VERIFY_{\PK_1}(c,s)$ and then decrypts the message
  $m\gets\DEC_{\SK_2}(c)$.

  \begin{enumerate}
  \item What are properties of the encryption and the signing scheme
    are needed to guarantee secure message transmission? Compute the
    corresponding security guarantees.
  \item Show that the message transmission protocol may become
    insecure if $\PARTY_1$ uses the signing key $\SK_1$ also for some
    other purposes. Give an explicit attack description under the
    assumption that the secret key $\SK_2$ can be extracted using
    chosen message attacks.

  \item Show that the message transmission protocol can become
    insecure if $\PARTY_2$ uses $\SK_2$ to decrypt messages of several
    senders.
 
  \item Interpret the results.  In which contexts, the this message
    transmission protocol is useful? When is the traditional
    construction based on symmetric encryption and authentication
    primitives better?

  \item[$(\star)$] Give a construction of secure message transmission
    protocol that is still based on signing and asymmetric encryption
    primitives but is significantly more secure against malicious
    behaviour.
 
  \end{enumerate}

\item Construct an identification scheme that is based on a signature
  scheme. Prove that the corresponding identification scheme is secure
  in the most powerful setting, where the adversary can run several
  identification protocols concurrently in order to impersonate true
  signer.

\item Digital signatures are often used in various electronic
  transaction systems. In the simplest setting, sellers issue a bills
  for goods that must be signed by a bank before the transaction
  becomes valid. However, such a setting violates the privacy of
  buyers as the bank will always know what its clients have bought.
  The latter is the main motivation for blind-signatures as they allow
  to hide messages from the signer. More precisely, consider an blind
  signature scheme, which is based on full domain hash and RSA.
  \begin{itemize}
  \item A public key is RSA modulus $n$ and a exponent $e$. The
    corresponding private exponent $d$ such that $ed=1\mod \phi(n)$ is
    the secret key.
  \item To sign a message $m$, the signer must compute $s=h(m)^d\mod
    n$.
  \item A signature $(m,s)$ is valid if $h(m)=s^e\mod n$. 
  \item To get a blind signature, a client must compute $h(m)$ and
    then choose $r\getsu\ZZ_n^*$ and sent $\overline{m}=h(m)r^e\mod n$
    to the signer. When the signer replies
    $\overline{s}=\overline{m}^d\mod n$, the client computes
    $s=\overline{s}r^{-1}\mod n$.
  \end{itemize}
  \begin{enumerate}
  \item Show that the signature obtained in the blind signing protocol
    is valid and the signer learns nothing about the message, i.e., we
    can simulate $\overline{m}$ without seeing $m$.
  \item Show that the blind signature scheme can be less secure than
    the underlying RSA signature scheme by providing an explicit
    attack that works against blind signatures and not for the RSA
    signature.
  \item Construct a protocol for selling goods over the internet that
    preserves the privacy of buyers, assures that no client can spend
    more money than he or she owns and guarantees that sellers
    get their money.
  \item Is it possible to design a protocol where buyers can
    anonymously revoke transactions when sellers cheat.
  \end{enumerate}



\item[$(\star)$] Prove that security of a signature scheme can be
  never proved through a reduction that shows how to extract secret
  key from an adversary who is successful in deception. More
  precisely, show that if such a reduction exists then there exists
  also an attack strategy that extracts a secret key using few signing
  queries. Why this impossibility result does not conflict with the
  security proofs in the random oracle model?

\item Let $\GG=\langle g\rangle$ be a $(t,\varepsilon_1)$-secure
  $q$-element discrete logarithm group. Then the Schnorr signature
  scheme is defined as follows.
  \begin{itemize}
  \item  A secret key $x\getsu\ZZ_q$ and the corresponding public key $y\gets g^x$.
  \item To sign a message $m$, generate $r\getsu\ZZ_q$, compute
    $\alpha\gets g^r$, $\beta\gets h(m,\alpha)$ and then compute the
    reply $\gamma$ of the Schnorr identification protocol. The
    resulting signature is a triple $s=(\alpha,\beta,\gamma)$.
  \item A pair $(m,s)$ is a valid signature if $h(m,\alpha)=\beta$ and
    $g^\gamma=y^\beta\alpha$.
  \end{itemize}
  Prove that if $h$ is replaced with a random oracle $\ORA_h(\cdot)$,
  then the Schnorr signature scheme is $(c\cdot
  t\varepsilon_1,\frac{\varepsilon_1}{2})$-secure against existential
  forgeries in the key only attack model for an appropriate constant
  $c$ provided that the adversary makes only single call to
  $\ORA_h(\cdot)$ and that the adversary always controls the validity
  of the proposed signature.
  \begin{enumerate}
  \item Show that an that if an adversary $\AD$ succeeds with
    probability $\varepsilon$, there exists an adversary $\ADB$ that
    succeeds with probability $\varepsilon$ in the Schnorr
    identification protocol and has same time complexity.
  \item Use the theorem about matrix games to estimate the average
    running-time of a knowledge extractor for the Schnorr protocol and
    use Markov inequality to find success probability for a strict
    time bound.
  \item Note that the previous result holds only for the values of
    public keys $y$ such that the success probability is large
    enough. Show that if the average success probability
    $\advEAUTHXX{}{\ADB}> \varepsilon$ then the probability of getting
    a bad key $y$ such that $\varepsilon_y<\frac{\varepsilon}{2}$ is
    low.
  \end{enumerate}

\item Construct a generic signature scheme from the Fiat-Shamir
  identification protocol. Recall that the Fiat-Shamir protocol works
  as follows.
  \begin{itemize}
  \item A secret key $s\getsu\ZZ_n^*$ and the public key $v\gets
    s^2\mod n$.
  \item To authenticate, the prover generates $r\getsu\ZZ_n^*$ and
    sends $\alpha\gets r^2\mod n$ to the verifier. The verifier sends
    $\beta\getsu\set{0,1}$ to the prover who replies $\gamma\gets
    rs^b\mod n$. The prover succeeds if $\gamma^2=\alpha v^b\mod n$.
  \end{itemize}


\item Any instantiation of the full domain hash signature scheme
  defines implicitly a bundle $\HHH\Join\FFFTP$ of function families
  $\HHH$ and $\FFFTP$. Namely, the signature scheme is determined by a
  tuple of algorithms $(\GEN,\MAP,\INV,\HASH)$, where
  $(\GEN,\MAP,\INV)$ determines the collection of trapdoor
  permutations $\FFFTP$ and functions $\INV_\SK:\MSPACE_\PK\to\SSPACE$
  and $\HASH_\PK:\MSPACE\to\TSPACE_\PK$ have matching input and output
  domains $\TSPACE_\PK\subseteq \MSPACE_\PK$ for every
  $(\PK,\SK)\gets\GEN$. The corresponding bundle $\HHH\Join\FFFTP$ of
  function families $\HHH$ and $\FFFTP$ is
  $(t,\varepsilon)$-claw-free if for any $t$-time adversary $\AD$
  the following advantage
  \begin{align*}
    \advCFXX{\FFFTP\Join\HHH}{\AD}=\pr{\GAME^\AD=1}\leq\varepsilon\enspace
  \end{align*}
  where
  \begin{align*}
    \begin{game}{\GAME^\AD}
      &(\PK,\SK)\gets\GEN\\
      &(m,s)\gets\AD(\PK)\\
      &\RETURN [\HASH_\PK(m)\iseq \MAP_\PK(s)]
    \end{game}
  \end{align*}
  Prove the following facts about the full domain hash signature
  scheme.
  \begin{enumerate}
  \item The signature scheme is $(t,\varepsilon)$-secure against
    existential forgeries in the model, where the adversary cannot
    access the signing oracle, if the bundle $\HHH\Join\FFFTP$ is
    $(t,\varepsilon)$-claw-free.
  \item Generalise the notion of claw-free bundles so that
    $(t,\varepsilon)$-security is sufficient for the standard attack
    model.
  \end{enumerate}

\item Consider a full domain hash signature as in the previous
  exercise. Assume that the hash function family $\HHH$ is strongly
  $\varepsilon_1$-regular, i.e., for every key pair
  $(\PK,\SK)\gets\GEN$ and the output distribution of $\HASH_\PK(m)$
  where $m\getsu\MSPACE$ and uniform distribution over $\MSPACE_\PK$
  are $\varepsilon_1$-close. Now consider the security against
  universal forgeries
    \begin{align*}
      \advUFORGEXX{\HHH\Join\FFFTP}{\AD}=\pr{\GAME^\AD=1}
    \end{align*}
   where
   \begin{align*}
     \begin{game}{\GAME^\AD}
       &(\PK,\SK)\gets\GEN\\
       &m\getsu\MSPACE\\
       &s\gets\AD(m,\PK)\\
       &\RETURN \VERIFY_\PK(m,s)
     \end{game}
   \end{align*}
   and prove that $(t,\varepsilon)$-security of trapdoor collection
   $\FFFTP$ is sufficient for security. Generalise the notion of
   one-wayness so that it is also sufficient against chosen message
   attacks.




\end{enumerate}
\end{document}

