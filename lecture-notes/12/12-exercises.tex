\documentclass{article}
\usepackage{crysymb}
\usepackage[draft]{crygame}
\usepackage{crypto-ii}



\begin{document}

\noindent	
MTAT.07.003 Cryptology II\\
Spring 2009 / Exercise session XII



\begin{enumerate}

\item Sometimes it is inherently possible to implement the desired
  functionality without breaching desired security. For instance,
  consider a two-party protocol for addition, where at the end both
  parties obtain $x_1+x_2$.
  \begin{enumerate}
  \item Show that any protocol that implements addition also reveals
    the input of the opponent. Generalise the result and show that it
    is always possible to deduce something about the opponents input
    unless the desired output is not a constant (protocol is actually
    redundant).
  \item Show that if a malicious party knows the parity of the
    opponents input before the protocol, then he or she can control
    the output parity in the addition protocol. Generalise the result
    and show that given enough information the adversary can always
    partially control the opponents output unless the output is
    constant.
  \end{enumerate}

\item To illustrate some aspects of secure computations, consider a
  simultaneous broadcast protocol used for the rock-paper-scissors
  game:
  \begin{itemize}
  \item $\PARTY_1$ computes $\PK\gets\GEN$.
    $(c,d)\gets\COM_\PK(x_1)$ and sends $\PK,c$ to $\PARTY_2$.
  \item $\PARTY_2$ replies $x_2$ to $\PARTY_1$ who after that
    releases $d$ to $\PARTY_1$,
  \item $\PARTY_2$ computes $x_1\gets\OPEN_\PK(c,d)$ and both parties output $x_1,x_2$. 
  \end{itemize}
  Let $\PARTY_2^*$ be a semi-malicious adversary that chooses
  $\hat{x}_2$ as an efficiently computable randomised function
  $f(c,x_2)$ but outputs $(x_1,\hat{x}_2)$ as an output. For
  simplicity, assume that the input of party $\PARTY_1$ consists only
  form the protocol input $x_1$ and thus $\PARTY_1$ outputs
  $(x_1,\hat{x}_2)$.
  \begin{enumerate}
  \item Compute the corresponding output distribution if the initial
    input distribution $\INPDIST$ is a uniform distribution over
    $\ZZ_3\times\ZZ_3$ and
    \begin{align*}
      \alpha(x_1,x_2,\hat{x}_2)=\pr{\PK\gets\GEN,(c,d)\gets\COM_\PK(x_1):
        f(c,x_2)=\hat{x}_2}
    \end{align*}
    is independent form $x_2$. To get concrete results, fix some
    concrete values for the $\alpha$ table.
  \item Compute the output distribution in the ideal world under the
    assumption that the commitment scheme is perfectly hiding and we
    use a naive simulation construction
    \begin{align*}
    \begin{fblock}{\SSS^{\PARTY_2^*}(x_2)}
      &\PK\gets\GEN\\
      & (c,d)\gets\COM_\PK(0)\\
      & \text{Send }\hat{x}_2\gets\PARTY_2^*(\sigma_2,x_2,c)\text{ to }\TTP.\\
      & \text{Given $x_1,\hat{x}_2$ from $\TTP$ set $(c,d)\gets\COM_\PK(x_1)$.}\\
      & \text{Rewind $\PARTY_2^*$ and output whatever $\PARTY_2^*$ on
        $\PK,c,d$ does.}
  \end{fblock}
  \end{align*}
  What is the corresponding statistical difference between
  real and ideal world implementations?
\end{enumerate}
\item Consider the implementation of simultaneous broadcast primitive
  detailed in the previous exercise. For simplicity and concreteness,
  let the input distribution and $\PARTY_2^*$ be the same as in the
  previous exercise.
  \begin{enumerate} 
  \item Compute the distance between ideal and real world
    distributions if we use an inefficient but perfect simulator
  \begin{align*}
  \begin{fblock}{\SSS_\circ^{\PARTY_2^*}(x_2)}
    & \PK\gets\GEN\\
    & (c,d)\gets\COM_\PK(0)\\
    & \text{Send }\hat{x}_2\gets\PARTY_2^*(\sigma_2,x_2,c)\text{ to }\TTP.\\
    & \text{Given $x_1,\hat{x}_2$ from $\TTP$,
      generate all runs between $\PARTY_1(x_2)$ and $\PARTY_2(x_2)$:}\\
    &\begin{cblock}
      &\text{Choose randomness $\omega_1\gets\Omega_1$ and
        $\omega_2\gets\Omega_2$ for $\PARTY_1$ and $\PARTY_2^*$.}\\
      &\text{Run the protocol and store the output $(x_1,\overline{x}_2)$ of $\PARTY_2^*$ into a list $\LLL$.}
   \end{cblock}\\
   &\text{Let $\LLL_{\hat{x}_2}$ be the list of pairs $(x_1,\overline{x}_2)\in\LLL$
   such that $\hat{x}_2=\overline{x}_2$.}\\
   &\text{Choose uniformly a pair for the list $\LLL_{\hat{x}_2}$ and output it.}
  \end{fblock}
  \end{align*}
  Show that if the commitment scheme is perfectly hiding then the
  output distributions in the real and ideal world coincide.
  \item Show that the simulator
  \begin{align*}
  \begin{fblock}{\SSS^{\PARTY_2^*}(x_2)}
   &\PK\gets\GEN\\ 
   & (c,d)\gets\COM_\PK(0)\\
   & \text{Send }\hat{x}_2\gets\PARTY_2^*(x_2,c)\text{ to }\TTP.\\
   & \text{Given $x_1,\hat{x}_2$ from $\TTP$ rewind until success.}\\
   &\begin{cblock}
    & (c,d)\gets\COM_\PK(x_1)\\
    & \text{If $\PARTY_2^*(x_2,c)\neq \hat{x}_2$ repeat the cycle.}
   \end{cblock}\\
   &\text{Output whatever $\PARTY_2^*$ does.}
  \end{fblock}
  \end{align*}
  provides same output distribution as the simulator $\SSS_\circ$
  provided that $\SSS$ stops before the time-bound $\tid$. Compute the
  failure probability for a fixed value of $\tid$ and estimate the
  final statistical distance between real and ideal world
  distributions.
  \end{enumerate}

\item The Blum coin-fipping protocol is very similar to the
  simultaneous broadcast protocol described and analysed in previous
  exercises.
  \begin{enumerate}
  \item This resemblance is not a coincidence. Prove that given an
    ideal implementation of addition protocol, it is trivial to
    implement coin-flipping protocol. Also, prove that given an ideal
    addition protocol it is trivial to implement simultaneous
    broadcast protocol, where both parties learn their inputs and vice
    versa.
  \item Construct a coin-flipping protocol from the ideal simultaneous
    broadcast protocol and substitute the ideal implementation with
    the protocol analysed in previous exercises. Compare the end
    result with the description of the Blum protocol:
   \begin{itemize}
   \item $\PARTY_1$ generates $b_1\getsu\set{0,1}$, computes
     $\PK\gets\GEN$, $(c,d)\gets\COM_\PK(b_1)$ and sends $\PK,c$ to
     $\PARTY_2$.
   \item $\PARTY_2$ generates $b_2\getsu\set{0,1}$ to $\PARTY_1$ who after that
    releases $d$ to $\PARTY_1$.
   \item $\PARTY_2$ computes $b_1\gets\OPEN_\PK(c,d)$ and both parties
    output $b_1\oplus b_2$.
   \end{itemize}
\end{enumerate}

\item Let $\PROT_\circledast$ denote the the coin-flipping protocol
  that uses an ideal simultaneous broadcast primitive and $\PROT$ the
  Blum protocol.
\begin{enumerate}
\item Construct a simulator for $\PARTY_1^*$ for the protocol
  $\PROT_\circledast$. Next, modify the simulator so that it works
  with the Blum coin-flipping protocol. For that recall that the
  simulator for $\PARTY_1^*$ for the simultaneous broadcast protocol
  consists of an input extractor block
   \begin{align*}
   \begin{fblock}{\KEXTR^{\PARTY_1^*}(\sigma_1,x_1)}
    & \text{Generate randomness $\omega_1\gets\Omega_1$ for $\PARTY_1^*$.}\\ 
    & (\PK,c)\gets \PARTY_1^*(\sigma_1,x_1;\omega_1)\\
    & \text{Use rewinding to get }\\
    &\begin{cblock}
     &d_0\gets\PARTY_1^*(0),
      d_1\gets\PARTY_1^*(1),
    \end{cblock}\\
   &\text{Reveal the actual inputs:}\\
   &\begin{cblock}
       &\hat{x}_1^0\gets\OPEN_\PK(c,d_0)\\
       &\hat{x}_1^1\gets\OPEN_\PK(c,d_1)\\
     \end{cblock}\\
  & \text{If $\bot\neq\hat{x}_1^0\neq\hat{x}_1^1\neq\bot$ then output $\mathsf{double\text{-}opening}$.}\\
  & \text{If $\hat{x}_1^0\neq\bot$ output $\hat{x}_1^0$ else output $\hat{x}_1^0$.}
  \end{fblock} 
\end{align*}%
followed by the protocol simulation block
\begin{align*}
  \begin{fblock}{\SSS_{\circ}^{\PARTY_1^*}(x_1,\omega_1,x_2)}
    & (\PK,c)\gets \PARTY_1^*(\sigma_1,x_1;\omega_1)\\
    & d\gets\PARTY_1^*(x_2)\\
    & \text{If $\OPEN_\PK(c,d)=\bot$ then order $\TTP$ to  halt the computations.}\\
    & \text{Output whatever $\PARTY_1^*$ outputs.}   
  \end{fblock}
\end{align*}
\item Analyse the quality and running time of these simulators. Show
  that both achieve perfect simulation of the output distributions,
  i.e., the real and ideal world distributions coincide.
\end{enumerate}

\item Let $\PROT_\circledast$ denote the the coin-flipping protocol
  that uses an ideal simultaneous broadcast primitive and $\PROT$ the
  Blum protocol.
\begin{enumerate}
\item Construct a simulator for $\PARTY_2^*$ for the protocol
  $\PROT_\circledast$. Next, modify the simulator so that it works
  with the Blum coin-flipping protocol.  For that recall that the
  simulator for $\PARTY_1^*$ for the simultaneous broadcast protocol
  consists of an input extractor block
   \begin{align*}
   \begin{fblock}{\KEXTR^{\PARTY_2^*}(\sigma_2,x_2)}
     & \text{Generate randomness $\omega_2\gets\Omega_2$ for $\PARTY_2^*$.}\\
     &\PK\gets\GEN\\
     & (c,d)\gets\COM_\PK(0)\\
     & \text{Return }\hat{x}_2\gets\PARTY_2^*(x_2,c).\\
  \end{fblock} 
\end{align*}%
followed by the protocol simulation block
\begin{align*}
  \begin{fblock}{\SSS_{\circ}^{\PARTY_2^*}(x_2,\omega_2,x_1,\hat{x}_2)}
   & \text{Rewind until success.}\\
   &\begin{cblock}
   & (c,d)\gets\COM_\PK(x_1)\\
   & \text{If $\PARTY_2^*(x_2,c;\omega_2)\neq \hat{x}_2$ repeat the cycle.}
   \end{cblock}\\
   &\text{Output whatever $\PARTY_2^*$ does.}
  \end{fblock}
\end{align*}
\item Analyse the quality and running time of these simulators. Show
  that the real and ideal world distributions are statistically close
  if the number of rewinds is high enough.
\end{enumerate}

\end{enumerate}



\end{document}