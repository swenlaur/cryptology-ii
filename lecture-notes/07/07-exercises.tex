\documentclass{article}
\usepackage{crysymb}
\usepackage[draft]{crygame}
\usepackage{crypto-ii}
\usepackage{graphics}
\usepackage{graphicx}

\renewcommand{\vec}[1]{\boldsymbol{#1}}

\begin{document}


\noindent	
MTAT.07.003 Cryptology II\\
Spring 2010 / Homework VII
\begin{enumerate}
\item To minimise memory footprint in servers, operational information
  is often stored by clients and provided on demand. Web cookies are
  the most famous example. Such a storage strategy opens up new attack
  vectors, since malicious clients can provide inconsistent data that
  might lead to crashes or code injection attacks.
  \begin{enumerate}
  \item Design simple integrity tests based on collision resistant
    hash function if the stored data is always used as a single unit.
  \item Provide a solution if stored data is structured and only few
    substructures are used in each operation. For example, the entire
    file system is stored at client cite who can potentially alter it.
  \item Design a data protection model for BitTorrent like
    application, where the data is hosted by many potentially
    malicious sub-servers and a client assembles the entire file by
    combining the data streams.
  \item[$(?)$] The MD5 hash function was recently shown to be weak, i.e., it
    is possible to find collisions. However, the attacker cannot
    control the values of colliding messages. Are now all integrity
    protection mechanisms based on MD5 insecure or not?

    \textbf{Clarification:} The MD5 hash function is iterative
    \begin{align*}
      f^*(m_1,\ldots m_n)=f(f(\cdots
      f(f(\mathsf{iv},m_1),m_2),\ldots,m_{n-1}),m_n)
    \end{align*}
    where $f:\TSPACE\times\MSPACE\to\TSPACE$ is a dedicated
    compression function.
  \end{enumerate}
  Compute all corresponding security guarantees provided that the hash
  function is sampled form the $(t,\varepsilon)$-collision-resistant
  function family $\HHH$.

\item There are several other properties that hash function families
  can have besides collision resistance.
  \begin{itemize}
  \item A hash function family $\HHH$ is $(t,\varepsilon)$-secure
    one-way function family if for any $t$-time adversary $\AD$
    \begin{align*}
      \pr{h\getsu\HHH, m_0\getsu\MSPACE, m_1\gets\AD(h,h(m_0)): h(m_0)=h(m_1)}\leq\varepsilon\enspace.
    \end{align*}
  \item A hash function family $\HHH$ is $(t,\varepsilon)$-secure
    against second preimage if for any $t$-time adversary $\AD$
    \begin{align*}
      \pr{
        \begin{aligned}
         & h\getsu\HHH, m_0\getsu\MSPACE, m_1\gets \AD(h,m_0):\\
         & m_0\neq m_1\wedge h(m_0)=h(m_1)
        \end{aligned}}\leq\varepsilon\enspace.
    \end{align*}
  \end{itemize}
  Establish the corresponding homological classification of these
  three properties under the assumption that $\HHH$ is a compressing
  function family. Provide the corresponding reductions.
  \begin{enumerate}
  \item Show that collision resistance implies security against second
    preimage attacks.
  \item Show that security against second preimage attacks implies
    one-wayness.
  \item Give interpretation to all three properties. Is the MD5
    function still secure against second preimage attacks?
  \item[$(\star)$] Give the corresponding separations that show that
    the corresponding inclusions are strict under the assumption that
    $\HHH$ is compressing function family.
  \end{enumerate}



\item The main drawback of the modified Naor commitment scheme is
  message expansion---to commit one bit one must send $n$ bits. One
  possibility is to increase the size of the message space.  Let the
  message space $\MSPACE$ be a subset of a finite field
  $(\FF_{2^n};+,\times)$ such that we can treat all $n$-bit strings as
  elements of $\FF_{2^n}$. Then we can define modified commitment
  scheme:
    \begin{align*}
      \begin{aligned}
      \begin{fblock}{\GEN}
       & \PK\getsu\FF_{2^n}^*\\
       &\RETURN \PK       
      \end{fblock}\\
      \
      \end{aligned}
      \qquad
      \begin{fblock}{\COM_\PK(x)}
        & d\gets\set{0,1}^k\\
        & c\gets f(d)+ x\times \PK\\
        & \RETURN (c,d) 
      \end{fblock}
      \qquad
      \begin{fblock}{\OPEN_\PK(c,d)}
        &y\gets c\oplus f(d)\\
        &\IF y\notin\PK\times\MSPACE\ \THEN \RETURN \bot\\
        &\ELSE \RETURN y\times \PK^{-1}
      \end{fblock}
    \end{align*}
    Establish the corresponding security guarantees under the
    assumption that $f:\set{0,1}^k\to\set{0,1}^n$ is a
    $(t_1,\varepsilon_1)$-pseudorandom generator.

    How big must be the message space $\MSPACE\subseteq \FF_{2^n}$ to
    achieve reasonable security guarantees against double openings?

    \textbf{Hint:} How many decommitment pairs can lead to a double
    opening? How is this number related to the size of $\FF_{2^n}$ and
    $\MSPACE$?


  \item Another way to improve the modified Naor commitment scheme is
    to use a collision resistant hashing to build a list commitment
    scheme on top of the ordinary commitment scheme:
    \begin{align*}
      \begin{fblock}{\GEN}
        &\PK\getsu\set{0,1}^n\\
        &h\getsu\HHH\\
        &\RETURN (\PK,h)
      \end{fblock}
      \qquad
      \begin{fblock}{\COM_{\PK,h}(x_1,\ldots,x_{\ell})}
        &(c_i,d_i)\gets \mathsf{Naor\text{-}Com}_{\PK}(x_i),\ i=1,\ldots,\ell\\ 
        &c_*\gets h(c_1,\ldots,c_{\ell})\\
        &\RETURN (c_*,(c_1,\ldots,c_{\ell},d_1,\ldots,d_{\ell})) 
      \end{fblock}
    \end{align*}
    where the decommitment procedure just verifies
    $c_*=h(c_1,\ldots,c_\ell)$ and restores
    $x_i\gets\OPEN_\PK(c_i,d_i)$ for $i=1,\ldots,\ell$.
    \begin{enumerate}
    \item Establish security guarantees under the assumption that the
      basic commitment scheme is $(t_1,\varepsilon_1)$-hiding and
      $(t_2,\varepsilon_2)$-binding and $\HHH$ is a
      $(t_3,\varepsilon_3)$-collision resistant hash function family.
    \item Modify the compaction strategy so that it is possible to
      open individual bits without leaking information about the
      others.
 
    \item[$(?)$] Can we use a pseudorandom generator $f$ for compacting the
      decommitment? What happens if we generate $d_0,\ldots,d_{\ell-1}$ by
      stretching a single master seed $d_*$? Provide 
      corresponding security guarantees.
    \end{enumerate}
  
  \item One of the most elegant properties of additively homomorphic
    commitments is the ability to do verifiable shuffling.  As an
    example consider the following card shuffling protocol:
    \begin{enumerate}
    \item[$\PARTY_1$] generates a random permutation
      $\pi:\set{1,\ldots,36}\to\set{1,\ldots,36}$. Let $P$ be the
      corresponding $36\times 36$ zero-one matrix such that
      $\pi(\vec{y})=P\vec{y}$ for any $n$-element vector $\vec{y}$ and
      let $(c_{ij},d_{ij})\gets\COM_\PK(p_{ij})$. Next, $\PARTY_1$
      sends the matrix of commitments $c_{ij}$ to $\PARTY_2$.
    \item[$\PARTY_2$] computes randomly shuffled card pack. First
      $\PARTY_1$ chooses a random permutation $x_1,\ldots,x_{36}$ of
      the set $\set{1,\ldots,36}$. Next, $\PARTY_2$ computes
      \begin{align*}
        e_i\gets c_{i1}^{x_1}\cdot c_{i2}^{x_2}\cdots c_{in}^{x_n} c_i^*\enspace,
      \end{align*}
      where $(c_i^*,d_i^*)\gets\COM_\PK(0)$, and sends
      $e_1,\ldots,e_n$ to $\PARTY_2$.
    \end{enumerate}
   \begin{enumerate}
   \item Prove that the values $e_1,\ldots, e_n$ are indeed randomly
     shuffled commitments of $x_1,\ldots, x_n$.
   \item Prove that neither $\PARTY_1$ not $\PARTY_2$ cannot guess
     where is the commitment to $36$ among $e_1,\ldots,e_n$ if
     commitment is $(t,\varepsilon)$-hiding.
   \item Prove that $\PARTY_1$ and $\PARTY_2$ can release cards one by
     one and one can detect cheating in the release phase if
     commitment scheme is $(t,\varepsilon_1)$-binding.
   \item How $\PARTY_1$ can prove that $c_{ij}$ are indeed commitments
     to the permutation matrix under the assumption that $c_{ij}$ are
     guaranteed to be commitments of zeros or ones?
    
     \textbf{Hint:} Can one characterise permutation matrices in terms
     of row and column sums.
   \item[$(\star)$] Use cut-and-choose techniques to make the protocol
     secure against malicious corruption in the dealing phase.

   \end{enumerate}

 \item Consider the following simple user-aided key agreement
   protocol. The public key $\PK$ of a server $\PARTY_1$ is known to
   all participants. If a participant $\PARTY_2$ wants to connect to
   $\PARTY_1$ it generates a random session key $k\getsu\KSPACE$ and a
   short authentication nonce $r\getsu\set{0,\ldots,9999}$ and sends
   $\ENC_\PK(k\|r)$ to $\PARTY_1$. Next $\PARTY_1$ recovers $k$ and
   $r$ and sends $r$ as an SMS back to $\PARTY_2$. The client
   $\PARTY_2$ halts if the SMS does not correspond to his or her
   authentication nonce.
  \begin{enumerate}
  \item Prove that a $t$-time adversary can alter the ciphertext
    without being detected with probability at most
    $10^{-4}+\varepsilon$ provided that the cryptosystem is
    $(t,\varepsilon)$-IND-CCA2 secure and no adversary can alter the
    SMS message.
  \item Provide an explicit ciphertext altering strategy against
    the ElGamal cryptosystem which succeeds with probability $\frac{1}{4}$
    under the assumption that $k||r$ is uniformly distributed over the
    message space $\GG$.
  \end{enumerate}



\end{enumerate}
\end{document}
