\documentclass{crypto-exercise}
\usepackage{amsthm}
\author{Sven Laur}
\contributor{Sven Laur}
\editor{Sven Laur}
\tags{Oblivious transfer, input extraction, output equivocation, stateful protocols}

\newcommand{\REWRITE}{\qquad\scalebox{1.5}{$\rightsquigarrow$}\qquad}

\newcommand{\CS}{\mathcal{C}om}
\newcommand{\IEXTR}{\mathscr{K}}
\newcommand{\OEQUIV}{\mathscr{E}}
\newcommand{\FAIL}{\mathsf{Fail}}
\newcommand{\ELIF}{\mathsf{else\ if}\ }

\makeatletter
%\newcommand{\xleftrightarrow}[2][]{\ext@arrow 3359\leftrightarrowfill@{#1}{#2}}
\newcommand{\xdashrightarrow}[2][]{\ext@arrow 0359\rightarrowfill@@{#1}{#2}}
\newcommand{\xdashleftarrow}[2][]{\ext@arrow 3095\leftarrowfill@@{#1}{#2}}
\newcommand{\xdashleftrightarrow}[2][]{\ext@arrow 3359\leftrightarrowfill@@{#1}{#2}}
\def\rightarrowfill@@{\arrowfill@@\relax\relbar\rightarrow}
\def\leftarrowfill@@{\arrowfill@@\leftarrow\relbar\relax}
\def\leftrightarrowfill@@{\arrowfill@@\leftarrow\relbar\rightarrow}
\def\arrowfill@@#1#2#3#4{%
  $\m@th\thickmuskip0mu\medmuskip\thickmuskip\thinmuskip\thickmuskip
   \relax#4#1
   \xleaders\hbox{$#4#2$}\hfill
   #3$%
}
\makeatother

\begin{document}


\begin{exercise}{Security of against malicious server}
Analyse security of the Aiello-Ishai-Reingold oblivious transfer for additively homomorphic encryption scheme with prime order message space:  
\begin{align*}
&\PARTY_1(b) &&  &&\PARTY_2(x_0,x_1) \\
&&\xdashleftrightarrow[\hspace*{2cm}]{\fbox{$\PK,\SK\gets\GEN$}\atop {\phantom{ \textstyle c} \atop\SK,\PK \hspace{1.3cm}\PK}}\\
&c\gets\ENC_\PK(b)\\
&&\xrightarrow[\hspace*{2cm}]{c}\\
&&&&&r\gets\MMM \\
&&&&&d_0=\ENC_\PK(br+x_0) \\
&&&&&d_1=\ENC_\PK((1-b)r+x_1)\\
&&\xleftarrow[\hspace*{2cm}]{d_0,d_1} && \\
&\RETURN \DEC_\SK(d_b)
&& &&\RETURN \emptyset
\end{align*}
where $b\in\set{0,1}$ and $x_0,x_1\in\MMM$ are private protocol inputs and a triple of algorithms $(\GEN,\ENC,\DEC)$ is an additively homomorphic encryption scheme. The dashed line denotes sub-protocol for fixing the commitment parameters. 
Prove that there exist an efficient simulator for $\PARTY_2$ if the set of plausible attack goals consists of computationally bounded predicates.
\end{exercise}

\begin{solution}
\ \\
\textsc{Right ideal implementation.}
As the party $\PARTY_2$ gets no output fairness is acheivable:
\begin{align*}
&\PARTY_1(b) &&&& \TTP  &&&&\PARTY_2(x_0, x_1) \\
&&&\xrightarrow[\hspace*{2cm}]{b}
&&&\xleftarrow[\hspace*{2cm}]{x_0,x_1}\\
&&&\xleftarrow[\hspace*{2cm}]{x_b}
&&&\xrightarrow[\hspace*{2cm}]{\emptyset}\\
&\RETURN x_b
&&&&&&&&\RETURN \emptyset
\end{align*}

\noindent
\textsc{\textcolor{red}{(a)} Input extractor for $\PARTY_2$.}
\textcolor{red}{
\begin{itemize}
\item Construct input extractor for $\PARTY_2^*$ 
\item Show that outputs of $\PARTY_1$ in real and ideal world coincide. 
\end{itemize}}

\noindent
\textsc{\textcolor{red}{(b)} Output extractor for $\PARTY_2$.}
\textcolor{red}{
\begin{itemize}
\item Construct output equivocator for $\PARTY_2^*$ that achieves $1/2$ statistical distance for the joint output distribution $\psi_1, \psi_2$.
\item Show that the construction is tight for unbounded predicates 
\end{itemize}}

\noindent
\textsc{\textcolor{red}{(c)} Output extractor for $\PARTY_2$.}
\textcolor{red}{
\begin{itemize}
\item Show that outputs are computationally indistinguishable for time-bounded predicates
\end{itemize}}

\end{solution}
\end{document}

