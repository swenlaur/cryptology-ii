\documentclass{crypto-exercise}
\contributor[Created the first draft solution]{Bingsheng Zhang}
\contributor[Significantly expanded the original exposition]{Sven Laur}
\editor{Sven Laur}
\tags{security definition, hash functions, impossibility result}


\begin{document}
\begin{exercise}{Security of standard password hashing}
  Salted hashing is common mechanism for storing passwords. Each user
  is first granted an identifier $\ID$. Every time the password
  is hashed, the identifier $\ID$ is appended to it and then a
  system-wide hash function is used to compute the digest. The
  authentication is successful if the digest coincides with the digest
  stored in \texttt{passwd} file.
  \begin{enumerate}
  \item Formalise the system by using abstract primitives for
    appropriate actions. Describe the functional requirements that are
    needed for seamless authentication. Compare the formalisation with
    the lottery system described in the first exercise.
  \item Define an attack scenario where the attacker tries to reverse
    engineer passwords from salted hashes. Formalise the corresponding
    security condition. Can this security condition be met in
    practice?
  \item Extend the attack description to the setting where the
    identifier $\ID$ depends on the identity of a user. What does such
    a design choice give to the attacker? Modify the corresponding
    attack scenario and derive the corresponding security requirement.
  \end{enumerate}
\end{exercise}
\begin{solution} 
  Formally, the password authentication consists of two main functions
  $\mathsf{SumbitHash}$ and $\mathsf{VerifyPasswd}$ where
  $\mathsf{SubmitHash}:\XSPACE\times \III\to \CSPACE$ and
  $\mathsf{VerifyPasswd}: \XSPACE\times \III\times \CSPACE\to
  \set{0,1}$ such that
  \begin{align*}
    \forall x\in\XXX\ \forall \ID\in\III:\quad%
    \mathsf{VerifyPasswd}(x, \ID, \mathsf{SubmitHash}(x,\ID))\equiv 1\enspace.
  \end{align*}
  In practice, both functions are implemented by using a single hash
  function of type $h: \XSPACE\times\III \to \DSPACE$ where $\DDD$ is
  the digest space, $\XSPACE$ is password space and $\III$ is the set
  of all possible identities. The hash function with the complex
  domain $\XSPACE\times\III$ is usually constructed from a (moderately
  slow) general purpose hash function of type $f:\set{0,1}^*\to
  \set{0,1}^{256}$ by using a special envelope for representing
  elements of $\XSPACE\times\III$ as bit strings. The most simplest
  way is of course concatenation of $x$ and $\ID$ denoted by
  $x\|\ID$. However, there are many better alternatives, as
  well. Hence, we abstract away exact details how a general purpose
  hash function is converted to a hash function with dedicated domain
  and state all notions in terms of dedicated hash functions.

  The function $\mathsf{SumbitHash}$ is commonly implemented as
  $h(x,\ID)$ and thus we must evaluate a predicate $h(x,\ID)= y$ in
  order to test whether $x$ is the correct password for the identity
  $\ID$ and the digest $y$. In that sense, password based
  authentication is completely equivalent to the lottery system
  described above. 

 

  The first attack scenario, where an attacker tries to reverse
  passwords from $q$ hashes, can be modelled as the following game
  \begin{align*}
    \begin{game}{\GAME_1^{\AD}}
      &  x_1,\ldots,x_q \getsu \MSPACE\\
      & \ID_1\ldots,\ID_q\getsu \III\\
      & \FOR i\in\set{1,\ldots, q}\ \DO y_i\gets h(x_i,\ID_i)\\
      & (x_*, k) \gets  \AD(h(y_1,\ldots,y_q)\\
      & \RETURN [ h(x_*,\ID_k) = y_k]
    \end{game}
  \end{align*}
  provided that all users generate passwords randomly and identities
  are also randomly assigned. The game models an attack without
  specific target identity given access to $q$ password hashes
  $y_1,\ldots, y_q$. One can make the game more realistic by
  specifying a distribution of passwords---a huge dictionary of words
  together with their occurrence probabilities. Of course, the
  complete specification of dictionary distribution is usually out of
  reach and thus one often only postulates certain aspects of the
  distribution, such as the total amount of passwords and maximal
  occurrence probability (min-entropy). 

  In many practical systems, the password length is limited to eight
  alphanumeric symbols. As a result, the total size of the password
  space is $(2\cdot 26+10)^{8}\lesssim 2.2\cdot 10^{14}$. This space
  can be completely traversed in $692$ computer years assuming that we
  can compute $10,000$ evaluations of hash function in a second on an
  ordinary computer. If an adversary uses a moderate botnet of size
  $10,000$ then the whole space can be looked through in $25$
  days. Consequently, hashes in password files provide only a moderate
  level security. More importantly, it is almost impossible to
  increase security margin in large-scale systems. First of all,
  average users are not motivated or cannot memorise reliably
  passwords with more characters. Second, we cannot make the hash
  function much slower, as servers must handle thousand login attempts
  in a second.

  Hash salting only assures that one has to relaunch the attack when
  he or she needs to compromise different identity. If the salt $\ID$
  depends on the user name then one might be able to compute single
  password-hash table once and use it to attack many independent
  systems. Under our assumptions the full size of such dictionary is
  around $6400$ Terabytes, which is a lot but tractable for large
  governmental agencies. Hence, it is not a good idea if a user
  \texttt{root} has the same salt in many systems. 

  More formally, if the salt depends on the identity, then a corrupted
  system administrator might allure users to take specific user
  names. In the most severe case, the adversary can directly specify
  salts for all users, which leads to the following security game:
  \begin{align*}
    \begin{game}{\GAME_2^{\AD}}
      & \ID_1\ldots,\ID_q\gets \AD\\
      &  x_1,\ldots,x_q \getsu \MSPACE\\
      & \FOR i\in\set{1,\ldots, q}\ \DO y_i\gets h(x_i,\ID_i)\\
      & (x_*, k) \gets  \AD(h(y_1,\ldots,y_q)\\
      & \RETURN [ h(x_*,\ID_k) = y_k]\enspace.
    \end{game}
  \end{align*}
  Note that it does not make sense to model the adversaries influence
  on passwords. First of all, if the adversary can influence $x_k$
  then we can assume that it controls the identity $\ID_k$
  anyway. Secondly, the description of the hash function $h$ is public
  and thus corruption of specific users does not give adversary new
  power for guessing the passwords of uncorrupted users.
\end{solution}
\end{document}

