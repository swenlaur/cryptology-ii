\documentclass{crypto-exercise}
\author{Sven Laur}
\contributor[Created the first draft solution]{Sanja Šćepanović}
\contributor[Inproved presentation]{Bingsheng Zhang}
\editor{Sven Laur}
\tags{security definition, hash functions, impossibility result}

\begin{document}
\begin{exercise}{Electronic lottery}
  A company holds electronic lotteries briefly described below. A
  secure hardware is used to generate the output $x\in \set{0,
    1}^{20}$.  The input $x$ is fed into a specially designed garbling
  algorithm (worldwide patented), which takes in $x$ and broadcasts a
  garbled output $y$ to everybody. All participants submit their
  guesses $q_i$ about $x$. The company releases $x$ and everybody uses
  the description of the garbling algorithm to verify that y was
  computed form $x$. The one who guessed $x$ gets a prize.
  \begin{enumerate}
  \item Formalize the lottery system using abstract primitives for
    appropriate actions. Describe such functional requirements that
    the organizer cannot cheating and that participants can verify the
    correctness.
  \item Define an attack scenario where participants try to
    cheat. Quantify the success of the malicious
    participant. Formalize the corresponding security definition.
  \item Show that no garbling algorithm can meet functional
    requirements and be secure at the same time.
  \end{enumerate}
\end{exercise}
\begin{solution}
  The lottery system can be viewed as a public function $f:
  \set{0,1}^{20}\rightarrow \YYY$ together with a broadcast mechanism
  that provides authentic transmission of $y$ value to all
  participants of the lottery.
  \begin{figure}[h]
    \illustration[scale=0.6]{figures/0101-lottery-function}
    \caption{Lottery design for three participants}
    \label{fig:0101:electronic-lottery}
  \end{figure}
 

  The easiest way to assure that the organiser of the lottery cannot
  cheat is to assure that $f$ is one-to-one mapping, i.e., there are
  no $20$-bit strings $x_1\neq x_2$ that would hash to the same value
  $y$. Then everybody can verify the legitimacy of the lottery by
  computing $f(x)$ and comparing it with $y$. A more careful analysis
  shows that $f$ does not have to be one-to-one mapping as long as the
  organiser cannot easily find two different inputs that hash to the
  same value. However, this condition cannot be satisfied unless the
  function is injective due to the small domain size. As there are
  only $2^{20}$ possible inputs, the organiser can evaluate $f$ on
  each of them and quickly find and tabulate all possible collisions
  $f(x_1)=f(x_2)$.
 
  Let us now consider security against malicious participants of the
  lottery.  For clarity, assume that organises of the lottery collect
  guesses during a single day, i.e., each participant has $86,400$
  seconds to submit their guess. Given that an ordinary computer can
  do $10^9$ operation per second, an attacker must construct an
  algorithm that gives an output after $10^{14}$ elementary
  operations. Of course, this is a rough lower bound on available
  computing power. For example, the adversary can rent large computing
  cluster or use botnet to crack the function $f$. Although these
  actions may increase computational power of an adversary, we can
  always fix some (not so) well-justified limit $t$ of computational
  operations, which is infeasible for all potential adversaries or has
  to high cost compared to final gains. Hence, the protocol is secure
  if all $t$-time algorithms $\AD$ does not have overly big
  probability to win the following game
  \begin{align*}
    \begin{game}{\GAME^{\AD}}
      & x \getsu \set{0,1}^{20} \\
      & x_* \gets \AD(f(x))\\
      & \RETURN [x_* \iseq x]
    \end{game}
  \end{align*}
  which captures the honest lottery organisation. There are several
  aspects to note here. First, we do not have to give $f$ as an input
  to the adversary $\AD$. The function $f$ is fixed and we can always
  assume that its description is directly hardwired into the code of
  $\AD$. Second, one can easily win the lottery with probability
  $2^{-20}$ by just choosing a single value $x_*$. Hence, it is common
  to measure the additional gain $\AD$ gets from seeing $f(x)$:
  \begin{align*}
    \ADV(\AD)=\pr{\GAME^\AD=1}-2^{-20}\enspace,
  \end{align*}
  where $\GAME^\AD$ denotes the output of the game and the probability
  is taken over the choice of $x$ and random coins used by $\AD$. To
  be concrete, lets postulate that the additional gain is negligible
  if it is below $10^{-6}$. You will be murdered in a year with this
  probability and yet you do not worry about it. Also, the probability
  that a legitimate user wins the lottery with probability about
  $10^{-6}$. 

  Now if we assume that $f$ is one-to-one mapping, it is rather
  straightforward to invert $f$.  Given input $y$, we can try all
  possible inputs $x\in\set{0,1}^{20}$ and outputs the value if
  $f(x)=y$. Hence, after $2^{20}$ evaluations of $f$, the adversary
  $\AD$ wins the game $\GAME$ with maximal probability $1$. Moreover,
  if one tries values of $x$ in random order, then success after $k$
  evaluations is approximately $k\cdot 10^{-6}$ and thus advantage
  around $10^{-6}$ is achievable only if one can evaluate $f$ on a
  single value during a day and even then the adversary can
  significantly increase its chances by cloud computing. 
\end{solution}
\end{document}
