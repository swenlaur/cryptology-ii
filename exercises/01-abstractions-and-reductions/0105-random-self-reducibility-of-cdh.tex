\documentclass{crypto-exercise}
\author{Sven Laur}
\contributor[Created the initial version of the solution]{Filipp Ivanov}
\tags{Computational Diffie-Hellman problem, random self-reducibility, missing solution}

\begin{document}
\begin{exercise}{Random self-reducibility of CDH}
  Let $\GG$ be a finite group such that all elements $y\in\GG$ can be
  expressed as powers of $g\in\GG$. Then the Computational
  Diffie-Hellman (CDH) problem is following. Given $x=g^a$ and
  $y=g^b$, find a group element $z=g^{ab}$.
  \begin{enumerate}
  \item Show that Computational Diffie-Hellman problem is random
    self-reducible, i.e., for any algorithm $\ADB$ that achieves
    advantage
    \begin{align*}
      \ADVCDH{\GG}{\ADB}\doteq\pr{x,y\getsu\GG: \ADB(x,y)=g^{\log_g
          x\log_g y}}
    \end{align*}
    there exists an oracle algorithm $\AD^{\ADB}$ that for any input
    $x,y\in\GG$ outputs the correct answer with the probability
    $\ADVCDH{\GG}{\ADB}$ and has roughly the same running time.
  \item Given that the CDH problem is random self-reducible, show that
    the difficulty of CDH instances cannot wary a lot. Namely, let
    $\ADB$ be a $t$-time algorithm that achieves maximal advantage
    $\ADVCDH{\GG}{\ADB}$. What can we say about worst-case advantage
    \begin{align*}
      \min_{x,y\in\GG}\pr{\AD(x,y)=g^{\log_g x\log_g y}}?
    \end{align*}
    Can there be a large number of pairs $(x,y)$ for which the CDH problem
    is easy?
   \item Show how to amplify the success rate of $\ADB$ by
    repetitions. Sketch the corresponding time-success profile
    $\varepsilon(t)$. What does this say about time-success profile of
    CDH problem in general?
  \end{enumerate}
\end{exercise}
\begin{solution}
\textsc{Random Self-reducibility.}
Given an original adversary $\ADB$ against computational Diffie-Hellman problem we can 
construct the following algorithm:
	\begin{align*}
		\begin{fblock}{\AD^{\ADB}(x,y)}
			& a, b \getsu \mathbb{Z}_{|\GG|} \\
			& c \gets \ADB(x \cdot g^a, y \cdot g^b) \\
			& \RETURN c \cdot x^{-b} \cdot y^{-a} \cdot g^{-ab}\enspace.
		\end{fblock}
	\end{align*}
For the analysis, let $\alpha = \log_g x$ and $ \beta = \log_g y$. Then by the definition, the tuple $x \cdot g^a, y \cdot g^b, c$ is a valid Diffie-Helmann tuple only if   
\begin{align*}
c = g^{(\alpha+a)(\beta+b)} \quad\Longleftrightarrow\quad c=g^{\alpha\beta} \cdot g^{\alpha b} \cdot g^{ab} \cdot g^{\beta a}\enspace.
\end{align*}
From this we can conclude 
\begin{align*}
c = g^{(\alpha+a)(\beta+b)} \quad\Longleftrightarrow\quad g^{\alpha\beta}= c \cdot (g^{\alpha})^{-b}  \cdot (g^{\beta})^{a}\cdot g^{ab}\enspace,
\end{align*}
which itself implies that the adversary $\AD^\ADB$ succeed if and only if $\ADB$ produces a Diffie-Helmann tuple:  
\begin{align*}
c = g^{(\alpha+a)(\beta+b)} \quad\Longleftrightarrow\quad  g^{\alpha\beta}= c \cdot x^{-b} \cdot y^{-a} \cdot g^{-ab}\enspace.
\end{align*}
Hence, the advantage of $\AD^{\ADB}$ can be calculated as follows:
\begin{align*}
\pr{\AD^{\ADB}(x,y) = g^{\alpha\beta}} =
\pr{\smash{a,b \getsu \mathbb{Z}_{|\GG|} : \ADB(x \cdot g^a, y \cdot g^b) = g^{(\alpha + a) \cdot (\beta + b)}}} \enspace.
\end{align*}
Now it is easy to see that for any $\forall \alpha,\beta \in \ZZ_{\abs{\GG}}$, the group elements $x \cdot g^a$ and $ y \cdot g^b$ are independent and have uniform distribution. Hence, the adversary $\ADB$ inside $\AD^\ADB$ gets correctly formed CDH challenges and we thus we can conclude
\begin{align*}
\pr{\smash{a,b \getsu \mathbb{Z}_{|\GG|} ; \ADB(x \cdot g^a, y \cdot g^b) = g^{(\alpha + a) \cdot (\beta + b)}}}=\ADVCDH{\GG}{\ADB}\enspace.
\end{align*}
If $\ADB$ runs in $t$-time, $\AD^{\ADB}$ runs in $(t + \delta)$-time, where $\delta$ is a small time required to perform element sampling and multiplications.

\vspace*{2ex}
\noindent\textsc{Uniformity.} Because $\AD$ reduces each problem instance to a random one, $\pr{\AD(x,y)=g^{\log_g x\log_g y}}$ is equal to $\ADVCDH{\GG}{\ADB}$ for each pair $(x,y)$. Therefore, the worst-case advantage of $\AD$ is the same as advantage of $\ADB$ and if there are a lot of CDH instances, which are easy for $\ADB$, the performance of $\AD$ is good on any instance.

\vspace*{2ex}
\noindent\textsc{Amplification effects.}
\textbf{To be added}	
	
\end{solution}
\end{document}
