\documentclass{crypto-exercise}
\author{Sven Laur}
\contributor[Created the first draft solution]{Sanja Šćepanović}
\contributor[Updated the first draft]{Bingsheng Zhang}
\contributor[Significantly expanded the original exposition]{Sven Laur}
\editor{Sven Laur}

\tags{simple reductions, Discrete Logarithm problem, Computational
  Diffie-Hellman problem, Decisional Diffie-Hellman problem,
  time-success profile}

\begin{document}
\begin{exercise}{Simple reductions to DL problem}
  Let $\GG=\langle g\rangle$ be a finite group of an order $q$
  generated by the powers of an element $g$. Let $\ADVDL{\GG}{\cdot},
  \ADVCDH{\GG}{\cdot}, \ADVDDH{\GG}{\cdot}$ denote corresponding
  advantages for Discrete Logarithm, Computational Diffie-Hellman, and
  Decisional Diffie-Hellman problems. 
  \begin{enumerate}
  \item Show that Decisional Diffie-Hellman problem can be reduced to
    Computational Diffie-Hellman problem, i.e., for any algorithm
    $\ADB$ that achieves advantage $\ADVCDH{\GG}{\ADB}$, there exists
    an oracle algorithm $\AD^{\ADB}$ with the advantage
    $\ADVDDH{\GG}{\AD}$ that has roughly the same running time.
  \item Show that Computational Diffie-Hellman problem can be reduced to
    Discrete Logarithm problem, i.e., for any algorithm
    $\ADB$ that achieves advantage $\ADVDL{\GG}{\ADB}$, there exists
    an oracle algorithm $\AD^{\ADB}$ with the advantage
    $\ADVCDH{\GG}{\AD}$ that has has roughly the same running time.
  \item Show by composing previous solutions that Decisional
    Diffie-Hellman problem can be reduced to Discrete Logarithm
    problem, i.e., for any algorithm $\ADB$ that achieves advantage
    $\ADVDL{\GG}{\ADB}$, there exists an oracle algorithm
    $\AD^{\ADB}$ with the advantage $\ADVDDH{\GG}{\AD}$ that has 
    roughly the same running time.
  \end{enumerate}
\end{exercise}
\begin{solution}
  \paragraph{Ddh$\Rightarrow$Cdh.}
  Let $\ADB$ be a successful CDH solver. Then we can compare its
  output $z_*$ to the element $z$ in question to determine whether the
  triple $x,y,z$ is a Diffie-Hellman triple or not. The corresponding
  formal reduction is given below:
  \begin{align*}
    \begin{fblock}{\AD^{\ADB}(x,y,z)}
      &z_*\gets \AD(x,y)\\
      &\RETURN [z_* \iseq z]\enspace.
    \end{fblock}
  \end{align*}
  Now if we substitute $\AD$ to the DDH games 
  \begin{align*}
    &\begin{game}{\GAME_0^{\AD}}
      & a, b\getsu \ZZ_q\\
      & c \getsu \ZZ_q\\
      & \RETURN \AD^\ADB(g^a,g^b,g^c)
    \end{game}
   &&\begin{game}{\GAME_1^{\AD}}
      & a, b \getsu \ZZ_q\\
      & c \gets ab\\
      & \RETURN \AD^\ADB(g^a,g^b,g^c)
    \end{game}
  \end{align*}
  we get the following games after some simplifications of the code:
  \begin{align*}
    &\begin{game}{\GAME_0^{\ADB}}
      & a, b, c\getsu \ZZ_q\\
      & z_*\gets \AD(g^a,g^b)\\
      & \RETURN [z_*\iseq g^{c}]
    \end{game}
   &&\begin{game}{\GAME_1^{\ADB}}
      & a, b \getsu \ZZ_q\\
      & z_*\gets \AD(g^a,g^b)\\
      & \RETURN [z_*\iseq g^{ab}]\enspace.
    \end{game}
  \end{align*}
  which allows us to establish the following equations:
  \begin{align*}
    \pr{\GAME_1^\AD=1}&=\ADVCDH{\GG}{\ADB}\enspace,\\
    \pr{\GAME_0^\AD=1}&=\frac{1}{q}\enspace.
  \end{align*}
  The first equation follows form As the game $\GAME_1^\ADB$ is
  identical to the game which defines $\ADVCDH{\GG}{\ADB}$. The second
  equality follows from the fact that $g^c$ is independent from $g^a$
  and $g^b$ and thus we rewrite the game in the following way:
  \begin{align*}
    &\begin{game}{\GAME_0^{\ADB}}
      & a, b\getsu \ZZ_q\\
      & z_*\gets \AD(g^a,g^b)\\
      & c\gets \ZZ_q\\ 
      & \RETURN [z_*\iseq g^{c}]\enspace.
    \end{game}
  \end{align*}
  Since $g^c$ is uniformly distributed over $\GG$ after the value
  $z_*$ is fixed the probability of hitting $z_*$ is indeed
  $\frac{1}{q}$. As a consequence, we have proved that 
  \begin{align*}
    \ADVDDH{\GG}{\AD} = \ADVCDH{\GG}{\ADB}-\frac{1}{q}\enspace.
  \end{align*}
  As the $\AD$ does a simple comparison operation besides calling
  $\ADB$, running times differ only by a constant. 

  \begin{remark}
    At first glance, asymptotic estimates do not make sense in the
    setting where the group is fixed as the problem instance cannot
    grow. However, we can also observe how the maximal advantage
    changes when we modify the time limit $t$. Consequently, we can
    talk about asymptotics w.r.t. and say that the difference between
    running times is $c=\Oh(1)$. The reduction above assures that if
    $\varepsilon(t)$ is the time-success profile for the CDH problem
    then $\varepsilon(t+c)-\frac{1}{q}$ is lower bound for the DDH
    time success-profile, see
    Figure~\ref{fig:0107-time-success-profiles}.
    \begin{figure}[t]
      \illustration[scale = 0.75]{figures/0107-time-success-profiles}
      \caption{Upper and lower bounds induced by the
        \textsc{Ddh}$\Rightarrow$\textsc{Cdh} reduction. Solid black
        line shows the time-success profile for DDH on the left and
        the time-success profile for CDH for on the right. Dotted red
        line shows the achievable advantage if we take best CDH
        solvers and convert them into DDH distinguishers. Dotted blue
        line shows the upper bound for CDH time-success profile due to
        the reduction.}
      \label{fig:0107-time-success-profiles}
    \end{figure}
  \end{remark}

  \paragraph{Cdh$\Rightarrow$Dl.} Let $\ADB$ be a successful DL
  solver. Then it is straightforward to use it in order to compute the
  last element of the Diffie-Hellman triple $x,y,z$. If we know the
  discrete logarithm of $x$ we can compute $z$ in the same way as in
  the Diffie-Hellman key exchange protocol. The corresponding
  reduction construction is given below:
  \begin{align*}
    \begin{fblock}{\AD^{\ADB}(x,y)}
      & a_*\gets \ADB(x)\\
      &\RETURN y^{a_*}\enspace. 
    \end{fblock}
  \end{align*}
  Now if we substitute this construction into the DL game we obtain
  after straightforward simplification
  \begin{align*}
    &\begin{game}{\GAME^{\AD}}
      & a, b\getsu \ZZ_q\\
      & a_*\gets \ADB(g^a)\\
      & \RETURN [g^{a_*b}\iseq g^{ab}]\enspace.
    \end{game}
  \end{align*}
  Clearly if $a_*=a$ then $g^{a_*b}= g^{ab}$ and thus
  \begin{align*}
   \ADVCDH{\GG}{\ADB}= \pr{\GAME^{\ADB}=1}\geq\pr{a\getsu\ZZ_q:\ADB(g^a)=a}=\ADVDL{\GG}{\ADB}\enspace.
  \end{align*}

  \paragraph{Ddh$\Rightarrow$Dl.} From previous parts, we know that
  for any DL-finder $\ADC$ we can construct a CDH-finder $\ADB$ such
  that $\ADVDL{\GG}{\ADC}\leq\ADVCDH{\GG}{\ADB}$ and from any
  CDH-finder $\ADB$ we can construct DDH-distinguisher $\AD$ such that
  \begin{align*}
    \ADVCDH{\GG}{\ADB}-\frac{1}{q}\leq\ADVDDH{\GG}{\AD}\enspace.
  \end{align*}
  By doing both transformations, we get a new construction
  \begin{align*}
    \begin{fblock}{\AD^{\ADC}(x,y,z)}
      &a_*\gets \ADC(x)\\
      &z_*\gets y^{a_*}\\
      &\RETURN [z_* \iseq z]\enspace,
    \end{fblock}
  \end{align*}
  which is guaranteed to satisfy 
  \begin{align*}
    \ADVDL{\GG}{\ADC}-\frac{1}{q}\leq\ADVDDH{\GG}{\AD}\enspace.
  \end{align*}
\end{solution}
\end{document}
