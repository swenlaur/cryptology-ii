\documentclass{crypto-exercise}
\usepackage{todonotes}
\author{Sven Laur}
\contributor{}
\editor{}

\tags{simple reductions, random self-reducibility, Quadratic Residuosity problem, incomplete solution}

\begin{document}
\begin{exercise}{Random self-reducibility of Quadratic Residuosity}
Show that Quadratic Residuosity problem for a fixed $N=pq$ where $p$ and $q$ are  
some Blum primes is randomly self-reducible.
Recall that a prime $p$ is a Blum prime if $p\equiv 3\mod 4$ and an element $y\in\ZZ_n$ is a quadratic residue if there exists $x$ such that $y=x^2\mod n$. 
The element $y$ can be quadratic residue only if its Jacobi symbol $\brak{\frac{y}{n}}=1$.
However, only half of the elements with Jacobi symbol one 
\begin{align*}
   J_n=\set{y\in\ZZ_N:\displaystyle{\brak{\frac{y}{n}}=1}}\enspace   
\end{align*}
belong to the set of quadratic residues
\begin{align*}
   QR_n=\set{y\in\ZZ_n:\exists x\in\ZZ_n: x^2\equiv y \pmod N}\enspace.   
\end{align*}
The Quadratic Residuosity problem is to distinguish between random elements of $QR_n$ and $J_n\setminus QR_n$.
\end{exercise}

\begin{solution}
\paragraph{Advantage against Quadratic Residuosity.}
\todo[inline]{Define advantage }

\begin{align*}
\advQRPXX{n}{\AD}= ??
\end{align*}
where
\begin{align*}
    \begin{game}{\BGAME_0^\AD}
      &x\getsu QR_n\\
      &??
    \end{game}
    \qquad\qquad
       \begin{game}{\BGAME_1^\AD}
      &x\getsu J_n\setminus QR_n\\
      &??
    \end{game}
\end{align*}
\ \\

\paragraph{Number theoretic properties of Quadratic Residues.}
One can show that Jacobi symbols satisfies following equations
\begin{align*}
    \brak{\frac{ab}{n}}=\brak{\frac{a}{n}}\cdot\brak{\frac{b}{n}}
    \qquad\text{and}\qquad
    \brak{\frac{a^2}{n}}=1\enspace.
\end{align*}

\todo[inline]{Explain how one can rerandomise Quadratic Residues}  


\paragraph{Reduction construction.}
\todo[inline]{Give a reduction construction}
\todo[inline]{Analyse it success and running-time.}

\end{solution}

\end{document}
