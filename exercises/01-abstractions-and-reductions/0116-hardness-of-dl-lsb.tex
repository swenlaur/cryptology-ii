\documentclass{crypto-exercise}
\author{Sven Laur}
\contributor[Significantly expanded the original exposition]{Sven Laur}
\editor{Sven Laur}

\tags{simple reductions, Discrete Logarithm problem, incomplete solution}

\begin{document}
\begin{exercise}{Hardness of least-significant discrete logarithm bit}
  Let $\GG=\langle g\rangle$ be a finite group of an order $q$
  generated by the powers of an element $g$. Then the Least Significant Discrete
  Logarithm Bit (LSB-DL) problem is following. For any element $y\in\GG$ find a bit $b$ such that
  such that $g^{2x+b}=y$ for $0\leq x < q/2$.
  \begin{enumerate}
  \item Show that if $q$ is even number then LSB-DL is easy.
  \item Show that if $q$ is odd number and algorithm that always solves LSB-DL then the DL
    problem is easy.
  \end{enumerate}
  \textsc{Hint.} As the order of $g$ is $q$, the equality $g^x=g^y$ implies $x\equiv y \pmod q$.\\
  \textsc{Hint.} Define a simple algorithm for finding square roots if $q$ is odd.

\end{exercise}


\begin{solution}
  

\end{solution}

\end{document}
