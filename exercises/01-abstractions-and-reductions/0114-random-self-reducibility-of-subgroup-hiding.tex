\documentclass{crypto-exercise}
\author{Sven Laur}
\contributor{}
\tags{Subgroup Hiding problem, random self-reducibility}

\newcommand{\ADVSH}[2]{\ADV_{#1}^{\mathsf{sgh}}(#2)}

\begin{document}
\begin{exercise}{Random self-reducibility of Subgroup Hiding}
 Let $\GG_1=\langle g_1\rangle$ be a $q$-element subgroup of a finite group $\GG$. 
 We say that $\GG_*$ is $(t,\varepsilon)$-indistinguishable from $\GG$ if for any $t$-time
 adversary $\AD$
  \begin{align*}
  \ADVSH{\GG}{\AD}=\abs{\pr{x\getsu\GG:\AD(x)=1}-\pr{x\getsu\GG_1:\AD(x)=1}}
  \leq\varepsilon\enspace.
  \end{align*}
 Show that if $\GG$ is a cyclic subgroup of $n$ elements then $\GG_1$ cannot be 
 indistinguishable form $\GG$. As a consequence, there must exist a base set 
 $\set{g_1,\ldots,g_\ell}$ such that any element of $\GG$ is uniquely representable as
 $g_1^{\alpha_1}\cdots g_\ell^{\alpha_\ell}$ for $\alpha_1,\ldots,\alpha_\ell\in\ZZ_q$. 
 Show that under this assumption subgroup hiding is randomly self-reducible. For that,
 define an algorithm $\ADB$ such that 
 \begin{align*}
 \abs{\pr{\ADB(x)=1}-\pr{\ADB(y)=1}}=\ADVSH{\GG}{\AD}
 \end{align*} 
 for any $x\in\GG_1$ and for any $y\in\GG\setminus\GG_*$. What is the additional requirement to 
 $q$ and what happens if this assumption is not satisfied? How one can define subgroup hiding 
 for cyclic groups?  
\end{exercise}
\begin{solution}
\textbf{Hint:} Let $g$ be the generator of $\GG$ how $g_*$ looks like and what can you tell about 
the structure of $\GG_*$ in terms of powers of $g$.
\textbf{Clarification:} Last question can be neglected.
	
\end{solution}
\end{document}
