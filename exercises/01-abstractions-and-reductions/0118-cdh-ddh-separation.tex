\documentclass{crypto-exercise}
\usepackage{hyperref}
\author{Sven Laur}

\tags{Computational Diffie-Hellman problem, Decisional Diffie-Hellman problem, bilinear pairing}

\begin{document}
\begin{exercise}{Separation between CDH and DDH}
  Let $\GG$ be a finite additive group of prime order $q$ such that all elements $y\in\GG$ can be
  expressed as multiples of $g\in\GG$.
  \begin{itemize}
  \item  Then the Computational
  Diffie-Hellman (CDH) problem is following. Given $x=a\cdot g$ and
  $y=b\cdot g$, find a group element $z=ab\cdot g$. 
  \item Then the Decisional
  Diffie-Hellman (DDH) problem is the following. For any triple
  $x,y,z\in\GG$, you must decide whether it is a Diffie-Hellman triple
  or not.
  \item The group $\GG$ has a bilinear pairing $\langle\cdot,\cdot \rangle: \GG\times\GG:\to\GG_*$ when following equalities hold
  \begin{align*}
     \langle x_1+x_2,y\rangle= \langle x_1,y\rangle\cdot \langle x_2,y\rangle\\
     \langle x,y_1+y_2\rangle= \langle x,y_1\rangle\cdot \langle x,y_2\rangle
  \end{align*}
  and the pairing is efficiently computable and non-degenerate, i.e., $\langle g,g \rangle\neq 1$.
  \end{itemize}
  Prove that $(t,\varepsilon)$-CDH group with a bilinear pairing cannot be DDH group.  
  \end{exercise}
\begin{solution}

\end{solution}
\end{document}
