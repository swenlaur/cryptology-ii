\documentclass{crypto-exercise}
\author[Formalisation of folklore]{Sven Laur}
\contributor[Initial solution]{Prastudy Fauzi}
\editor{Sven Laur}
\tags{Decisional Diffie-Hellman problem, random self-reducibility, success amplification}

\begin{document}
\begin{exercise}{Weak random self-reducibility of DDH}
  Let $\GG=\langle g\rangle$ be a finite group of a prime order $q$
  generated by the powers of an element $g$. Then the Decisional
  Diffie-Hellman (DDH) problem is the following. For any triple
  $x,y,z\in\GG$, you must decide whether it is a Diffie-Hellman triple
  or not. Formally, the corresponding distinguishing task is specified
  through two games:
  \begin{align*}
   &\begin{game}{\BGAME_0^{\ADB}}
      & a, b\getsu \ZZ_q\\
      & c \getsu \ZZ_q\\
      & \RETURN \ADB(g^a,g^b,g^c)
    \end{game}
   &&\begin{game}{\BGAME_1^{\ADB}}
      & a, b \getsu \ZZ_q\\
      & c \gets ab\\
      & \RETURN \ADB(g^a,g^b,g^c)
    \end{game}
  \end{align*}
  where the advantage is computed as
  $\ADVDDH{\GG}{\ADB}=\abs{\pr{\smash{\BGAME_0^\ADB=1}}-\pr{\smash{\BGAME_1^\ADB=1}}}$.
  Show that DDH problem is weakly random self-reducible in the following sense. For 
  any algorithm $\ADB$ that tries to distinguish Diffie-Hellman tuples from random 
  tuples, there exists an algorithm $\AD$, which has roughly the same running-time 
  than $\ADB$ and can for any pair of group elements $g^a$ and $g^b$ distinguish 
  $g^{ab}$ form a random group element $g^c$ with roughly the same advantage as 
  $\ADVDDH{\GG}{\ADB}$. More precisely, let the following games 
  \begin{align*}
   &\begin{game}{\GAME_0^{\AD}}
      & c \getsu \ZZ_q\\
      & \RETURN \AD(g^a,g^b,g^c)
    \end{game}
   &&\begin{game}{\GAME_1^{\AD}}
      & c \gets ab\\
      & \RETURN \AD(g^a,g^b,g^c)
    \end{game}
  \end{align*}
  model the distinguishing task. Then the corresponding advantage is
  \begin{align*}
  \ADV_{\GG,a,b}^{\mathsf{sf\text{-}ddh}}(\AD)=\abs{\pr{\GAME_0^{\AD}=1}-\pr{\GAME_1^{\AD}=1}}\enspace.
  \end{align*}
  Show that for any $a,b\in\ZZ_q$, the advantage $\ADV_{\GG,a,b}^{\mathsf{sf\text{-}ddh}}(\AD)$ can be bounded from below by a multiple of $\ADVDDH{\GG}{\ADB}$, while
  the running-time of $\AD$ is linear  wrt the running-time of $\ADB$. 
         
\end{exercise}
  
\begin{solution}  
Before going to the solution lets prove the following simple fact that for any element $x\in \ZZ_q$, the element $x+z$ is uniformly random  if $z \in \mathbb{Z}_q$ chosen uniformly at random from $\ZZ_q$. Indeed, note that for any $a \in \ZZ_q$, we have
\begin{align*}
 \pr{z\getsu\ZZ_q: x+z = a | z = y} = \pr{z\getsu\ZZ_q: z = a-x} = \frac{1}{q}\enspace.
\end{align*}
By knowing this fact, will show that for fixed values $a, b$ we can define an adversary $\AD$ such that $\ADV_{\GG,a,b}^{\mathsf{sf\text{-}ddh}}(\AD) = \ADVDDH{\GG}{\ADB}$ and the running time of $\AD$ is the running time of $\ADB$ plus some constant. Let us define the adversary $\AD$ as follows:
\begin{align*}
   &\begin{game}{\AD(g^a,g^b,g^c)}
      & x, y \getsu \ZZ_q\\
      & \RETURN \ADB(g^a \cdot g^x  ,g^b \cdot g^y, g^c \cdot (g^a)^y \cdot (g^b)^x \cdot g^{xy})\enspace.
    \end{game}
\end{align*}
Obviously the running time of this adversary is the same as $\ADB$ plus a constant $\delta$, which is approximately the time it takes to do 2 samplings from $\mathbb{Z}_q$, one multiplication in $\mathbb{Z}_q$, 5  multiplications in $\GG$ and 5 exponentiations in $\GG$. So it remains to show that $\ADV_{\GG,a,b}^{\mathsf{sf\text{-}ddh}}(\AD) = \ADVDDH{\GG}{\ADB}$.

First note that with the adversary $\AD$ defined above, the games can be rewritten as follows: 
\begin{align*}
   &\begin{game}{\GAME_0^{\AD}}
      & c \getsu \ZZ_q\\
      & x, y \getsu \ZZ_q\\
      & \RETURN \ADB(g^a \cdot g^x  ,g^b \cdot g^y, g^c \cdot (g^a)^y \cdot (g^b)^x \cdot g^{xy})
    \end{game}
   &&\begin{game}{\GAME_1^{\AD}}
      & c \gets ab\\
      & x, y \getsu \ZZ_q\\
      & \RETURN \ADB(g^a \cdot g^x  ,g^b \cdot g^y, g^c \cdot (g^a)^y \cdot (g^b)^x \cdot g^{xy})
    \end{game}
  \end{align*}

Next notice that swapping the first two steps in both games does not change anything, and after simplifying the exponents in the parameters for $\ADB$, we get the adjusted two games with the same advantage in distinguishing between them:
\begin{align*}
   &\begin{game}{\GAME_{01}^{\AD}}
      & x, y \getsu \ZZ_q\\
      & c \getsu \ZZ_q\\
      & \RETURN \ADB(g^{a+x}  ,g^{b+y}, g^{c + ay + bx + xy})
    \end{game}
   &&\begin{game}{\GAME_{11}^{\AD}}
      & x, y \getsu \ZZ_q\\
      & c \gets ab\\
      & \RETURN \ADB(g^{a+x}  ,g^{b+y}, g^{(a+x)(b+y)})
    \end{game}
  \end{align*}
Since $x$ and $y$ are independently and uniformly chosen from $\ZZ_q$, the elements $\overline{x}=x+a$ and $\overline{y}=y+b$ are also independent and have uniform distribution over $\ZZ_q$. Hence, we can further simplify the games without changing the advantage:   
\begin{align*}
   &\begin{game}{\GAME_{02}^{\AD}}
      & \overline{x}, \overline{y}  \getsu \ZZ_q\\
      & c  \getsu \ZZ_q\\
      & \RETURN \ADB(g^{\overline{x}}  ,g^{\overline{y}}, g^{\overline{c}})
    \end{game}
   &&\begin{game}{\GAME_{12}^{\AD}}
      & \overline{x}, \overline{y}  \getsu \ZZ_q\\
      & \overline{c} \gets \overline{x}\cdot\overline{y}\\
      & \RETURN \ADB(g^{\overline{x}}  ,g^{\overline{y}},  g^{\overline{c}})
    \end{game}
  \end{align*}
where $\overline{c}=c + a(\overline{y}-b) + b(\overline{x}-a) + (\overline{x}-a)(\overline{y}-b)$ in the first game. As the $\overline{c}$ value is again sum of a fixed value $( a(\overline{y}-b) + b(\overline{x}-a) + (\overline{x}-a)(\overline{y}-b))$ and a uniformly chosen $c$, we can  further simplify the first game:
\begin{align*}
   &\begin{game}{\GAME_{03}^{\AD}}
      & \overline{x}, \overline{y}  \getsu \ZZ_q\\
      & \overline{c}  \getsu \ZZ_q\\
      & \RETURN \ADB(g^{\overline{x}}  ,g^{\overline{y}}, g^{\overline{c}})
    \end{game}
   &&\begin{game}{\GAME_{13}^{\AD}}
      & \overline{x}, \overline{y}  \getsu \ZZ_q\\
      & \overline{c} \gets \overline{x}\cdot\overline{y}\\
      & \RETURN \ADB(g^{\overline{x}}  ,g^{\overline{y}},  g^{\overline{c}})
    \end{game}
  \end{align*}
Since games $\GAME_{03}$ and $\GAME_{13}$ are identical to the standard Decisional Diffie-Hellman challenges games, we get
\begin{eqnarray*}
\ADV_{\GG,a,b}^{\mathsf{sf\text{-}ddh}}(\AD)
&=& \abs{\pr{\GAME_0^{\AD}=1}-\pr{\GAME_1^{\AD}=1}} 
= \abs{\pr{\GAME_{03}^{\AD}=1}-\pr{\GAME_{13}^{\AD}=1}} \\
&=& \abs{\pr{\smash{\BGAME_0^\ADB=1}}-\pr{\smash{\BGAME_1^\ADB=1}}} 
= \ADVDDH{\GG}{\ADB}\enspace
\end{eqnarray*}
as desired and the proof is complete.
\end{solution}
\end{document}
