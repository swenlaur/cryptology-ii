\documentclass{crypto-exercise}
\usepackage{amsthm}
\author{Sven Laur}
\contributor[Initial solution]{Ivo Kubjas}
\contributor[Better treatment of corner cases]{Sven Laur}
\editor{Sven Laur}
\tags{non-malleability, IND-CPA security, target relation}

\newcommand{\ADVNMCPA}[2]{\ADV^{\mathsf{s\text{-}nm\text{-}cpa}}_{#1}{(#2)}}
\newcommand{\CS}{\mathfrak{C}}
\theoremstyle{plain}\newtheorem{stmt}{Statement}
\renewcommand{\qedsymbol}{}
\begin{document}


\begin{exercise}{SNM-CPA implies IND-CPA} Let $\CS=(\GEN,\ENC,\DEC)$ be a public key cryptosystem. We say that the crypto-system is $(t,t_\pi,t_m,\varepsilon)$-non-malleable against statically chosen relations (SNM-CPA) if for all $t$-time adversaries $\AD$ the advantage
\begin{align*}
\ADVNMCPA{\CS}{\AD}=\abs{\pr{\GAME_0^\AD=1}-\pr{\GAME_1^\AD=1}}\leq \varepsilon
\end{align*}
where the security games are defined as follows 
\begin{align*}
  \begin{game}{\GAME_0^\AD}
	&\pi(\cdot)\gets\AD\\
    &(\SK,\PK)\gets\GEN\\
    &\MSPACE_0\gets\AD(\PK)\\
    &m\gets\MSPACE_0\\
    &c\gets\ENC_\PK(m)\\
    &\hat{c}_1,\ldots\hat{c}_n\gets\AD(c)\\
    &\IF c\in\set{\hat{c}_1,\ldots\hat{c}_n}\THEN \RETURN 0\\
    &\RETURN \pi(m,\dec{\hat{c}_1},\ldots,\dec{\hat{c}_n})
  \end{game}
  \qquad\qquad
  \begin{game}{\GAME_1^\AD}
	&\pi(\cdot)\gets\AD\\
    &(\SK,\PK)\gets\GEN\\
    &\MSPACE_0\gets\AD(\PK)\\
    &m\gets\MSPACE_0,\overline{m}\gets{\MSPACE_0}\\
    &\overline{c}\gets\ENC_\PK(\overline{m})\\
    &\hat{c}_1,\ldots\hat{c}_n\gets\AD(\overline{c})\\
    &\IF \overline{c}\in\set{\hat{c}_1,\ldots\hat{c}_n}\THEN \RETURN 0\\
    &\RETURN \pi(m,\dec{\hat{c}_1},\ldots,\dec{\hat{c}_n})
  \end{game}
\end{align*}
and the sampling algorithm $\MSPACE_0$ is guaranteed to be a $t_m$-time algorithm and the predicate $\pi$ is guaranteed to  be $t_\pi$-time algorithm. Show that SNM-CPA security implies IND-CPA security by giving an explicit definition of the target relation $\pi$. Also, give the explicit quantification of running-time bounds. Interpret the results. 
\end{exercise}
  
\begin{solution} 
Let $\ADB$ be an adversary against IND-CPA. Then we construct the tree stage adversary $\AD$. The first stage of the adversary $\AD$ always outputs the binary relation $\pi(x,y)=[x= y]$ and the second and the third stage of the adversary are defined as follows 
\begin{align*}
& \begin{game}{\AD(\PK)}
& (m_0, m_1) \gets \ADB(\PK) \\
& \MSPACE_0 = \{m_0, m_1\} \\
& \RETURN \MSPACE_0
\end{game}
& \begin{game}{\AD(c)}
& g \gets B(c) \\
& \hat{c}_1 \gets \ENC_\PK(m_g) \\
& \RETURN \hat{c}_1 
\end{game}
\end{align*}
where $\MSPACE_0$ denotes the uniform distribution over $m_0$ and $m_1$. 

As the predicate $\pi$ is fixed, we can omit the predicate selection stage and replace the definition of the predicate into the games $\GAME_0$ and $\GAME_1$. As a result, we get the following simplification of the games:
\begin{align*}
  \begin{game}{\GAME_2^\AD}
    &(\SK,\PK)\gets\GEN\\
    &\MSPACE_0\gets\AD(\PK)\\
    &m\gets\MSPACE_0\\
    &c\gets\ENC_\PK(m)\\
    &\hat{c}_1,\ldots\hat{c}_n\gets\AD(c)\\
    &\IF c\in\set{\hat{c}_1,\ldots\hat{c}_n}\THEN \RETURN 0\\
    &\RETURN [m \neq \dec{\hat{c}_1}]
  \end{game}
  \qquad\qquad
  \begin{game}{\GAME_3^\AD}
    &(\SK,\PK)\gets\GEN\\
    &\MSPACE_0\gets\AD(\PK)\\
    &m\gets\MSPACE_0,\overline{m}\gets{\MSPACE_0}\\
    &\overline{c}\gets\ENC_\PK(\overline{m})\\
    &\hat{c}_1,\ldots\hat{c}_n\gets\AD(\overline{c})\\
    &\IF \overline{c}\in\set{\hat{c}_1,\ldots\hat{c}_n}\THEN \RETURN 0\\
    &\RETURN [m\neq \dec{\hat{c}_1}]
  \end{game}
\end{align*}

As the second stage of $\AD$ returns $\MSPACE_0$ that is a uniform distribution over $\set{m_0, m_1}$, we can replace sample generation by uniform choice of index bits $i,j\gets\set{0,1}$. We can also replace the query $\AD(c)$ by the explicit description of the third stage of the adversary. This leads us to the following games:
\begin{align*}
  \begin{game}{\GAME_4^\ADB}
    &(\SK,\PK)\gets\GEN\\
    &(m_0, m_1) \gets\ADB(\PK)\\
    & i \gets \{0, 1\} \\
    &c\gets\ENC_\PK(m_i)\\
    & g \gets B(c) \\
    & \hat{c} \gets \ENC_\PK(m_g) \\
    &\IF c = \hat{c}\ \THEN \RETURN 0\\
    &\RETURN [m_i \neq \dec{\hat{c}}]
  \end{game}
  \qquad\qquad
  \begin{game}{\GAME_5^\ADB}
    &(\SK,\PK)\gets\GEN\\
    &(m_0, m_1) \gets \ADB(\PK) \\
    & i \gets \{0, 1\}, j \gets\{0,1\} \\
    &\overline{c}\gets\ENC_\PK(m_j)\\
    & g \gets B(c) \\
    & \hat{c} \gets \ENC_\PK(m_g) \\
    &\IF \overline{c} = \hat{c}\ \THEN \RETURN 0\\
    &\RETURN [m_i\neq\dec{\hat{c}}]
  \end{game}
\end{align*}
We can further simplify the games, as $\DEC_\SK(\hat{c})=m_g$ by the perfect functionality of the cryptosystem:
\begin{align*}
  \begin{game}{\GAME_6^\ADB}
    &(\SK,\PK)\gets\GEN\\
    &(m_0, m_1) \gets\ADB(\PK)\\
    & i \gets \{0, 1\} \\
    &c\gets\ENC_\PK(m_i)\\
    & g \gets B(c) \\
    &\IF c = \ENC_\PK(m_g)\ \THEN \RETURN 0\\
    &\RETURN [m_i\neq m_g]
  \end{game}
  \qquad\qquad
  \begin{game}{\GAME_7^\ADB}
    &(\SK,\PK)\gets\GEN\\
    &(m_0, m_1) \gets \ADB(\PK) \\
    & i \gets \{0, 1\}, j \gets\{0,1\} \\
    &\overline{c}\gets\ENC_\PK(m_j)\\
    & g \gets B(\overline{c}) \\
    &\IF \overline{c} = \ENC_\PK(m_g)\ \THEN \RETURN 0\\
    &\RETURN [m_i\neq m_g]
  \end{game}
\end{align*}
As the if branch makes the analysis of the games more difficult, we first analyse the without these branches. This a good approximation. A cryptosystem can be IND-CPA secure only if such ciphertext collisions occur with a negligible probability. We will later analyse this issue in further.

\begin{align*}
  \begin{game}{\GAME_8^\ADB}
    &(\SK,\PK)\gets\GEN\\
    &(m_0, m_1) \gets\ADB(\PK)\\
    & i \gets \{0, 1\} \\
    &c\gets\ENC_\PK(m_i)\\
    & g \gets B(c) \\
    &\RETURN [m_i\neq m_g]
  \end{game}
  \qquad\qquad
  \begin{game}{\GAME_9^\ADB}
    &(\SK,\PK)\gets\GEN\\
    &(m_0, m_1) \gets \ADB(\PK) \\
    & i \gets \{0, 1\}, j \gets\{0,1\} \\
    &\overline{c}\gets\ENC_\PK(m_j)\\
    & g \gets B(\overline{c}) \\
    &\RETURN [m_i\neq m_g]
  \end{game}
\end{align*}
Since $i$ is not used before the last line in the game $\GAME_9$, we can rewrite the game as follows:
\begin{align*}
  \begin{game}{\GAME_9^\ADB}
    &(\SK,\PK)\gets\GEN\\
    &(m_0, m_1) \gets \ADB(\PK) \\
    &  j \gets\{0,1\} \\
    &\overline{c}\gets\ENC_\PK(m_j)\\
    & g \gets B(\overline{c}) \\
    &i \gets \{0, 1\}\\
    &\RETURN [m_i\neq m_g]
  \end{game}
\end{align*}
From this game construction it is very tempting to say that $\pr{\GAME_9^\ADB=1}=\frac{1}{2}$. However, this is not true! For example consider an adversary $\AD$ that always outputs $m_0=m_1$. Then clearly $\pr{\GAME_9^\ADB=1}=1$.

Intuitively, an adversary $\ADB$ that return $m_0=m_1$ with high probability is not useful, as this event diminishes their success against the IND-CPA games. However, we must prove this formally. Let us consider the advantage of $\ADB$ against IND-CPA games:
\begin{align*}
\begin{game}{\BGAME_0^\ADB}
    &(\SK,\PK)\gets\GEN\\
    &(m_0, m_1) \gets \ADB(\PK) \\
    &c\gets\ENC_\PK(m_0)\\
    &\RETURN \ADB(c)
\end{game}
\qquad\qquad
\begin{game}{\BGAME_0^\ADB}
    &(\SK,\PK)\gets\GEN\\
    &(m_0, m_1) \gets \ADB(\PK) \\
    &c\gets\ENC_\PK(m_1)\\
    &\RETURN \ADB(c)
\end{game}
\end{align*}
Now we can express the success of $\ADB$ against IND-CPA games by exhaustive decomposition:
\begin{align*}
\pr{\BGAME_0^\ADB=1}&=\pr{\BGAME_0^\ADB=1\wedge m_0=m_1}+ \pr{\BGAME_0^\ADB=1\wedge m_0\neq m_1}\\
\pr{\BGAME_1^\ADB=1}&=\pr{\BGAME_1^\ADB=1\wedge m_0=m_1}+ \pr{\BGAME_1^\ADB=1\wedge m_0\neq 
m_1}
\end{align*}   
If $m_0=m_1$ then the ciphertext in both games has identical distribution and thus we can prove
\begin{align*}
\pr{\BGAME_0^\ADB=1\wedge m_0=m_1}=\pr{\BGAME_1^\ADB=1\wedge m_0=m_1}\enspace.
\end{align*}  
As a consequence, 
\begin{align*}
\ADVINDCPA{\CS}{\ADB}=\abs{\pr{\BGAME_0^\ADB=1\wedge m_0\neq m_1}-\pr{\BGAME_1^\ADB=1\wedge m_0\neq m_1}}\enspace.
\end{align*}

There are now two ways how to proceed with the proof. First, we could do the same decomposition for the games $\GAME_8$ and $\GAME_9$. However, this would lead us to the lengthy probability calculations. Instead, we use a shortcut. Note that given $\ADB$ which can occasionally output $m_0=m_1$, we can define $\ADB^\circ$ that runs $\ADB$ and filters out these unfortunate events:
\begin{align*}
\begin{fblock}{\ADB^\circ(\PK)}
&(m_0,m_1)\gets\ADB(\PK)\\
&\IF (m_0=m_1)\ \THEN \RETURN (m_0^*,m_1^*)\\
&\ELSE \RETURN (m_0,m_1)  
\end{fblock}
\qquad\qquad
\begin{aligned}
\begin{fblock}{\ADB^\circ(c)}
&\IF (m_0=m_1)\ \THEN \RETURN 0\\
&\ELSE \RETURN \ADB(c)\enspace.  
\end{fblock}\\
\ \\
\end{aligned}
\end{align*} 
Then clearly $\ADVINDCPA{\CS}{\ADB^\circ}=\ADVINDCPA{\CS}{\ADB\circ}$ and the running times differ by a constant.

Hence, we can assume that $\ADB$ never returns $m_0=m_1$ and proceed further with the analysis. Under the assumption $m_0 \neq m_1$, the condition $m_i\neq m_g$ is equivalent to the condition $i\neq g$ and we can indeed conclude that $\pr{\GAME_9^\ADB=1}=\frac{1}{2}$. The game $\GAME_8$ simplifies to
\begin{align*}
  \begin{game}{\GAME_8^\ADB}
    &(\SK,\PK)\gets\GEN\\
    &(m_0, m_1) \gets\ADB(\PK)\\
    & i \gets \{0, 1\} \\
    &c\gets\ENC_\PK(m_i)\\
    & g \gets B(c) \\
    &\RETURN [i\neq g]\enspace,
  \end{game}
\end{align*} 
for which we know from previous exercises that $\pr{\GAME_8^\ADB=1}=\frac{1}{2}\pm\frac{1}{2}\cdot\ADVINDCPA{\CS}{\ADB}$. Thus, we have proved 
\begin{align*}
\abs{\pr{\GAME_8^\ADB=1}-\pr{\GAME_9^\ADB=1}}=\frac{1}{2}\cdot\ADVINDCPA{\CS}{\ADB}\enspace.
\end{align*} 
  
To complete the proof, we must relate this difference to the difference between the games $\GAME_6$ and $\GAME_7$. First note that if $c=\ENC_\PK(m_g)$ then $m_i=m_g$ and the adversary cannot win the game $\GAME_8$. Thus, we have proved
\begin{align*}
\pr{\GAME_6^\ADB=1}=\pr{\GAME_8^\ADB=1}\enspace.
\end{align*}
For the game $\GAME_7$, we can conclude only that
\begin{align*}
\pr{\GAME_7^\ADB=1}\leq\pr{\GAME_9^\ADB}\leq \frac{1}{2}\enspace.
\end{align*}
Without loss of generality we can assume that the $\ADB$ is such that
\begin{align*}
\pr{\GAME_8^\ADB=1}\geq \frac{1}{2}\enspace.
\end{align*}
If the inequality does not hold we can invert the the outcome of $\ADB$ to make it hold. The running time will increase only by constant. Now combining all inequalities we obtain 
\begin{align*}
\pr{\GAME_6^\ADB=1}-\pr{\GAME_7^\ADB=1}\geq \pr{\GAME_8^\ADB=1}-\frac{1}{2}= \abs{\pr{\GAME_8^\ADB=1}-\pr{\GAME_9^\ADB=1}}=\frac{1}{2}\cdot\ADVINDCPA{\CS}{\ADB}\enspace.
\end{align*} 
Hence, we have proved that 
\begin{align*}
\ADVNMCPA{\CS}{\AD}\geq\frac{1}{2}\cdot\ADVINDCPA{\CS}{\ADB}
\end{align*}
for the adversary $\AD$ for which the running time is only constant times larger than the running time of $\ADB$. Thus, $(t,t_\pi,t_m,\varepsilon)$-non-malleablity against statically chosen relations implies $(t,2\varepsilon)$-IND-CPA security.

\end{solution}
\end{document}


