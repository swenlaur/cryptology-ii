\documentclass{crypto-exercise}
\author{Sven Laur}
\contributor{Sander Siim}
\editor{Sven Laur}
\tags{Goldwasser-Micali cryptosystem, quadratic residuocity}

\begin{document}
\begin{exercise}{Security of Goldwasser-Micali cryptosystem}
Show that the Goldwasser-Micali cryptosystem is IND-CPA secure if the Quadratic Residuosity Problem is hard. 
\end{exercise}
  
\begin{solution}
Before we can give a corresponding proof we must define several concepts. Without them we cannot even define the Goldwasser-Micali cryptosystem.    
  

\vspace*{3ex}
\noindent
\textsc{Quadratic residioucity.}
A prime $p$ is a Blum prime if $p\equiv 3\mod 4$. Let $N=pq$ where $p,q$ are Blum primes. Then for each element $a\in\ZZ_N$, we can efficiently compute the Jacobi symbol $\bigl(\frac{a}{n}\bigr)$. One can show that Jacobi symbols satisfies following conditions:
\begin{align*}
 \biggl(\frac{ab}{n}\biggl)=\biggl(\frac{a}{n}\biggr)\cdot\biggl(\frac{b}{n}\biggr)
    \qquad\text{and}\qquad
    \biggl(\frac{a}{n}\biggr)=\pm 1\enspace.
\end{align*}
Also, recall that an element $b$ is a quadratic residue if there exists $a$ such that $b=a^2\mod N$.  The set of quadratic residues is denoted by $QR_N$. By the properties of Jacobi symbols all quadratic residues must be inside the following set
\begin{align*}
   J_N(1)=\set{x\in\ZZ_N:\brak{\frac{x}{n}}=1}\enspace.   
\end{align*}
Moreover, it can be shown using the Chinese Reminder Theorem that the set of quadratic non-residues $J_N(1)\setminus QR_N$ is as big as the set of quadratic residues $QR_N$.

\vspace*{3ex}
\noindent
\textsc{Quadratic residuosity problem.}  
Let $\PP_n$ denote uniform distribution over $n$-bit Blum primes. We say that the set of $n$-bit Blum primes is $(t,\varepsilon)$-secure with respect to quadratic residuosity problem if for all $t$-time adversaries $\ADB$ the advantage 
\begin{align*}
  \advQRPXX{\PP_n}{\ADB}=\abs{\pr{\smash{\BGAME_0^\ADB=1}}-\pr{\smash{\BGAME_1^\ADB=1}}}\leq\varepsilon
\end{align*}
where the security games are defined as follows:
\begin{align*}
    \begin{game}{\BGAME_0^\ADB}
      &p,q\getsu\PP(n)\\
      &N\gets pq\\ 
      &x\getsu QR_N\\
      &\RETURN \ADB(x,N)
    \end{game}
    \qquad\qquad
       \begin{game}{\BGAME_1^\ADB}
      &p,q\getsu\PP(n)\\
      &N\gets pq\\ 
      &x\getsu J_N(1)\setminus QR_N\\
      &\RETURN \ADB(x,N)\enspace.
    \end{game}
\end{align*}

\vspace*{3ex}
\noindent
\textsc{Definition of a cryptosystem.}
Goldwasser-Micali cryptosystem uses Blum primes and quadratic residuosity to encrypt bits using following algorithms. 
\begin{itemize}
\item \textbf{Key generation.}  Sample primes $p,q\in\PP(n)$ and choose quadratic non-residue $y\in J_N(1)$ modulo $N=pq$. Use $(N,y)$ as a public key $\PK$ and $(p,q)$ as a private key $\SK$.

\item\textbf{Encryption.} First choose a random $x\gets\ZZ_N^*$ and then compute
    \begin{align*}
      \enc{0}=x^2\mod N \quad\text{and}\quad \enc{1}=yx^2\mod N.
    \end{align*}
  \item\textbf{Decryption.} Output $0$ if the ciphertext $c$ is
    quadratic residue and $1$ otherwise. The latter is easy if the
    factorisation of $N$ is known.
\end{itemize}

\vspace*{3ex}
\noindent
\textsc{IND-CPA security.} Recall that IND-CPA security is defined through the following security games:
\begin{align*}
&\begin{game}{\GAME_0}
& (\PK,\SK) \gets \GEN\\
& (m_0,m_1) \gets \AD(\PK)\\
&\RETURN \AD(\ENC_\PK(m_0))
\end{game}
&
&\begin{game}{\GAME_1}
& (\PK,\SK) \gets \GEN\\
& (m_0,m_1) \gets \AD(\PK)\\
&\RETURN \AD(\ENC_\PK(m_1))\enspace.
&\end{game}
\end{align*}
More precisely, a public key cryptosystem is ($t,\varepsilon$)-IND-CPA secure, if the advantage of any $t$-time adversary $\AD$ against games $\GAME_0$ and $\GAME_1$ is bounded:
\begin{align*}
\advINDCPAXX{}{\AD}=\abs{\pr{\GAME_0^\AD=1} - \pr{\GAME_1^\AD = 1}}\leq \varepsilon\enspace.
\end{align*}
If we instantiate the IND-CPA security games for Goldwasser-Micali cryptosystem we get the following games:
\begin{align*}
&\begin{game}{\GAME_0}
& p,q \getsu \PP_n\\
& N \gets pq\\
& y \getsu J_N \setminus QR_N\\
& \PK \gets (N,y)\\
& (m_0,m_1) \gets \AD(\PK)\\
& x \getsu \ZZ_N^*\\
& \begin{ifblock}{m_0 = 0 \ } 
& c \gets x^2 \bmod N 
&\end{ifblock}\\
&\begin{elseblock}
& c \gets yx^2 \bmod N 
&\end{elseblock}\\
&\RETURN \AD(c)
\end{game}
&
&\begin{game}{\GAME_1}
& p,q \getsu \PP_n\\
& N \gets pq\\
& y \getsu J_N \setminus QR_N\\
& \PK \gets (N,y)\\
& (m_0,m_1) \gets \AD(\PK)\\
& x \getsu \ZZ_N^*\\
& \begin{ifblock}{m_1 = 0 \ } 
& c \gets x^2 \bmod N 
&\end{ifblock}\\
&\begin{elseblock}
& c \gets yx^2 \bmod N 
&\end{elseblock}\\
&\RETURN \AD(c)
&\end{game}
\end{align*}


Let us assume that there is an adversary $\AD$ which breaks the IND-CPA security of Goldwasser-Micali cryptosystem. We will perform a reduction to the quadratic residuosity problem, by constructing an adversary $\ADB$. The adversary construction is presented below:

\begin{align*}
&\begin{game}{\ADB(x,N)}
& \PK \gets (N,x)\\
& (m_0,m_1) \gets \AD(\PK)\\
& \hat{x} \getsu \ZZ_N^*\\
& b \getsu \set{0,1}\\
& \begin{ifblock}{m_b = 0 \ } 
& c \gets \hat{x}^2 \bmod N 
&\end{ifblock}\\
&\begin{elseblock}
& c \gets y\hat{x}^2 \bmod N 
&\end{elseblock}\\
&\RETURN [b \iseq \AD(c)]
&\end{game}
\end{align*}
Note that the construction is valid, since the adversary $\ADB$ knows $N$ and can therefore perform all the required operations. By inlining $\ADB$ into the games $\BGAME_0$ and $\BGAME_1$ defining the hardness of quadratic residuosity, we  get the the following games:
\begin{align*}
  \begin{game}{\BGAME_0^\ADB}
    &p,q\getsu\PP(n)\\
    &N\gets pq\\ 
    &x\getsu QR_N\\
    & \PK \gets (N,x)\\
    & (m_0,m_1) \gets \AD(\PK)\\
    & \hat{x} \getsu \ZZ_N^*\\
    & b \getsu \set{0,1}\\
    & \begin{ifblock}{m_b = 0 \ } 
    & c \gets \hat{x}^2 \bmod N 
    &\end{ifblock}\\
    &\begin{elseblock}
    & c \gets x\hat{x}^2 \bmod N 
    &\end{elseblock}\\
    &\RETURN [\AD(c)\smash \iseq b]
  \end{game}
    &
    &\begin{game}{\BGAME_1^\ADB}
    &p,q\getsu\PP(n)\\
    &N\gets pq\\ 
    &x\getsu J_N\setminus QR_N\\
    & \PK \gets (N,x)\\
    & (m_0,m_1) \gets \AD(\PK)\\
    & \hat{x} \getsu \ZZ_N^*\\
    & b \getsu \set{0,1}\\
    & \begin{ifblock}{m_b = 0 \ } 
    & c \gets \hat{x}^2 \bmod N 
    &\end{ifblock}\\
    &\begin{elseblock}
    & c \gets x\hat{x}^2 \bmod N 
    &\end{elseblock}\\
    &\RETURN [\AD(c)\smash\iseq b]
  \end{game}
\end{align*}

Let us first compute the probability $\pr{\BGAME_0^\ADB = 1}$. For that note that $\hat{x}^2$ and $x\hat{x}^2$ are completely indistinguishable to the adversary. Since $x$ is a quadratic residue, it can be written as $x = a^2 \bmod N$ for some $a\in \ZZ_N^*$ and thus $x\hat{x}^2 = (a\hat{x})^2 \bmod N$. Since $\hat{x}$ is generated uniformly randomly after $a$ has been fixed, the element $a\hat{x}$ is a random element from $\ZZ_N^*$. Consequently, $\hat{x}^2$ and $x\hat{x}^2$ have the same distributions and we can further simplify the game:
\begin{align*}
  \begin{game}{\BGAME_0^\ADB}
    &p,q\getsu\PP(n)\\
    &N\gets pq\\ 
    &x\getsu QR_N\\
    & \PK \gets (N,x)\\
    & (m_0,m_1) \gets \AD(\PK)\\
    & \hat{x} \getsu \ZZ_N^*\\
    & b \getsu \set{0,1}\\
    & c \gets \hat{x}^2 \bmod N \\
    &\RETURN [\AD(c)\smash \iseq b]
  \end{game}
\end{align*}
As the adversary $\AD$ receiver no information about $b$, the probability $\pr{\BGAME_0^\ADB = 1} = \frac{1}{2}$.


Let us now analyse the game $\BGAME_1^\ADB$. The game returns $1$ only if the adversary $\AD$ guesses the bit $b$. Thus, we must split the game into two sub-games based on the value of $b$. When $b=0$ the game $\BGAME_1^\ADB$ is equivalent to the game $\GAME_0^\AD$ and when $b=1$ the game $\BGAME_1^\ADB$ is equivalent to the game $\GAME_1^\AD$. Consequently, we can express the success probability as follows:
\begin{align*}
\pr{\BGAME_1^\ADB = 1} &= \pr{b = 0}\cdot\pr{\GAME_0^\AD = 0} + \pr{b = 1}\cdot\pr{\GAME_1^\AD = 1} \\
&= \frac{1}{2}\left(1 - \pr{\GAME_0^\AD = 1}\right) + \frac{1}{2}\cdot\pr{\GAME_1^\AD = 1} \\ 
&= \frac{1}{2} + \frac{1}{2}\cdot\pr{\GAME_1^\AD = 1} - \frac{1}{2}\cdot\pr{\GAME_0^\AD = 1} \enspace.
\end{align*}
As a result, we get a direct connection between the advantages of $\AD$ and $\ADB$:
\begin{align*}
\advQRPXX{\PP_n}{\ADB}&=\abs{\pr{\BGAME_0^\ADB = 1} - \pr{\BGAME_1^\ADB = 1}}\\
 &= \abs{\frac{1}{2} - \frac{1}{2} - \frac{1}{2}\cdot\pr{\GAME_1^\AD = 1} + \frac{1}{2}\cdot\pr{\GAME_0^\AD = 1}}\\
 &= \frac{1}{2}\cdot\abs{\pr{\GAME_0^\AD = 1} - \pr{\GAME_1^\AD = 1} }\\
 &= \frac{1}{2}\cdot\advINDCPAXX{}{\AD}\enspace.
\end{align*}

To complete the proof, we must also find the relation between the running-times of $\AD$ and $\ADB$. It is easy to see that the running-time of $\ADB$ is only by a constant $c$ larger than the running-time $\AD$. Consequently, the advantage of a $(t-c)$-time $\AD$ adversary is bounded:
\begin{align*}
\advINDCPAXX{}{\AD} \leq 2 \cdot \advQRPXX{\PP_n}{\ADB}\leq 2\varepsilon 
\end{align*}
and we have shown that Goldwasser-Micali cryptosystem is $(t-c,2\varepsilon)$-IND-CPA secure given that the set of $n$-bit Blum integers is $(t,\varepsilon)$-secure with respect to the quadratic residuosity problem.

\end{solution}



\end{document}