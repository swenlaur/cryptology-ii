\documentclass{crypto-exercise}
\usepackage{amsthm}
\author{Sven Laur}
\tags{proof-of-possession, disjunctive proof, witness-indistinguishability }


\begin{document}
\begin{exercise}{Disjunctive POP is witness indistinguishable}
Let $(\SK_0, \PK_0)$ and $(\SK_1, \PK_1)$ be private and public keys of an encryption scheme $(\GEN,\ENC,\DEC)$. 
Then a disjunctive proof for the possession of secret keys $\SK_0$ and $\SK_1$ is defined as follows:
\begin{enumerate}
  \item[1.] The verifier $\VERIFIER$ chooses $m\getsu\MSPACE$ and
    sends the corresponding challenge $\ENC_{\SK_0}(m;r_0)$ for
    $r_0\gets\RSPACE$, and $\ENC_{\SK_1}(m;r_1)$ for $r_1\gets\RSPACE$
    together with encryptions of random nonces $\ENC_{\SK_0}(r_1)$ and
    $\ENC_{\SK_1}(r_0)$ to $\PROVER$.
  \item[2.]  Given challenge ciphertexts $c_1,c_2,c_3,c_4$, the prover
    $\PROVER$ uses one of the secret keys $\SK_i$ to decrypt a
    challenge $\overline{m}$ and the nonce $r_{\neg i}$ used to
    randomise the other encryption $c_{\neg i}$. If $c_{\neg i}=
    \ENC_{\PK_{\neg i}}(\overline{m};r_{\neg i})$, the prover
    $\PROVER$ sends $\overline{m}$ to $\VERIFIER$, otherwise $\PROVER$
    can halt as $\VERIFIER$ cheats.
  \item[3.] The verifier $\VERIFIER$ accepts if $\overline{m}=m$.
\end{enumerate}
Prove that even an unbounded cheating verifier cannot learn whether the prover possesses $\SK_0$ or $\SK_1$  
\end{exercise}
\begin{solution}
Let us first formalise by games $\GAME_0$ and $\GAME_1$ what happens if the prover knows $\SK_0$ or $\SK_1$.\ldots 
Let us now modify the games so that the game pair encodes indistinguishability advantage \ldots
Now we can prove that in both games the honest prover rejects exactly the same challenges \ldots Next note that for a valid challenge the response is the same in both games. \ldots




\end{solution}
\end{document}