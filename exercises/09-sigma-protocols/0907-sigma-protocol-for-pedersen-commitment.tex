\documentclass{crypto-exercise}
\usepackage{amsthm}
\author{Sven Laur}
\contributor[Initial solution]{Alisa Pankova}
\contributor[Significant rewrite]{Sven Laur}
\editor{Sven Laur}
\tags{sigma protocol, discrete logarithm, zero-knowledge, special soundness, Pedersen commitment}

\begin{document}
\begin{exercise}{Sigma protocol for Pedersen commitment}
Let $\GG$ be a discrete logarithm group with a prime number $q$ elements. Then public parameters of the Pedersen commitments are two group elements $g_1$ and $g_2$. To commit an element $m\in\ZZ_q$, the commiter has to choose a random element $x\in\ZZ_q$ and compute a corresponding commitment $c=g_1^mg_2^x$. As the commitment is perfectly hiding, the committer can only prove that he or she knows $m$ and $x$ such that $c=g_1^mg_2^x$. Prove that the sigma protocol depicted below is  a sigma protocol for proving knowledge $\POK[\exists m \exists x: c=g_1^mg_2^x]$.
\begin{center}
  \begin{tabular}{ccc}
    $\PPP$ & & $\VVV$\\
    $r_1,r_2\getsu\ZZ_q$ \\
    &$\xrightarrow{\makebox[6cm]{$\alpha=g_1^{r_1} g_2^{r_2}$}}$ \\
    && $\beta\getsu\ZZ_q$ \\
    &$\xleftarrow{\makebox[6cm]{$\beta$}}$\\
    \\  
    &$\xrightarrow{\makebox[6cm]{$\gamma_1= m\beta+r_1, \gamma_2= x\beta+r_2$}}$\\
    && $\bigl[g_1^{\gamma_1}g_2^{\gamma_2}\iseq y^{\beta}\alpha\bigr]$\\  
  \end{tabular}
\end{center}  

\end{exercise}

\begin{solution}
As the protocol has the right message structure, we must show only that the protocol is functional and has special soundness and zero-knowledge property. 

\vspace*{2ex}
\noindent
\textsc{Functionality.} 
If both parities are honest then $y=g_1^mg_2^x$. Consequently,
\begin{align*}
g_1^{\gamma_1}g_2^{\gamma_2}
& = g_1^{m\beta + r_1}g_2^{x\beta + r_2} 
  = g_1^{m\beta}g_1^{r_1}g_2^{x\beta}g_2^{r_2}
  =  \left(g_1^{m}g_2^{x}\right)^{\beta}g_1^{r_1}g_2^{r_2}
  =  y^{\beta}\alpha
\end{align*}
and thus the verifier always reaches the accepting state.

\vspace*{2ex}
\noindent
\textsc{Special soundness.}
Let $(\alpha,\beta,\gamma_1,\gamma_2)$ and $(\alpha,\bar{\beta},\bar{\gamma}_1,\bar{\gamma}_2)$ two accepting protocol transcripts. To get the extraction formulae, we first express secrets $m$ and $x$ in terms of $\beta,\bar{\beta},\ldots,\bar{\gamma}_2$ under the assumption that the prover is honest. This leads us to the system of linear equations: 
\begin{align*}
&\begin{aligned}
\gamma_1 &= m\beta+r_1\enspace,\\
\bar{\gamma}_1 &= m\bar{\beta}+r_1\enspace,\\
\end{aligned}
&&
\begin{aligned} 
\gamma_2 &= x\beta+r_2\enspace,\\
\bar{\gamma}_2 &= x\bar{\beta}+r_2\enspace,
\end{aligned}
\end{align*}
which has the following solution
\begin{align*}
m &= \frac{\gamma_1 - \bar{\gamma}_1}{\beta - \bar{\beta}}\enspace,
&
x &= \frac{\gamma_2 - \bar{\gamma}_2}{\beta - \bar{\beta}}\enspace. 
\end{align*}
Next, we have to show that this holds for any accepting transcript pair. Let us verify this by direct algebraic manipulation:
\begin{align*}
g_1^mg_2^x=g_1^{\frac{\gamma_1 - \bar{\gamma}_1}{\beta - \bar{\beta}}}g_2^{\frac{\gamma_2 - \bar{\gamma}_2}{\beta - \bar{\beta}}} 
= \left(\frac{g_1^{\gamma_1}}{g_1^{\bar{\gamma_1}}}\cdot\frac{g_2^{\gamma_2}}{g_2^{\bar{\gamma_2}}}\right)^{\frac{1}{\beta - \bar{\beta}}} 
=  \left(\frac{\alpha y^{\beta}}{\alpha y^{\bar{\beta}}}\right)^{\frac{1}{\beta - \bar{\beta}}} 
 =  \left(y^{\beta - \bar{\beta}}\right)^{\frac{1}{\beta - \bar{\beta}}} 
 =  y\enspace.
\end{align*}

\vspace*{2ex}
\noindent
\textsc{Zero-knowledge property.}
Recall that the sigma protocol satisfies the zero-knowledge property if the protocol transcript can be simulated as follows:
\begin{align*}
\begin{fblock}{\SIM}
 &\beta\gets\ZZ_q\\
 &(\alpha,\gamma_1,\gamma_2)\gets\SSS(\beta)\\
 &\RETURN (\alpha,\beta,\gamma_1,\gamma_2)\enspace.
\end{fblock}
\end{align*}
To show the existence of such simulator, let us first analyse the distribution of $\gamma_1$ and $\gamma_2$ for a fixed $\beta$ value. If the prover is honest then 
\begin{align*}
\gamma_1&= m\beta+r_1\\
\gamma_2&= x\beta+r_2
\end{align*}
for randomly chosen $r_1,r_2\getsu\ZZ_q$ and thus $\gamma_1$ and $\gamma_2$ are independently and uniformly distributed over $\ZZ_q$. Hence, we know how to sample $\gamma_1$ and $\gamma_2$ for a fixed $\beta$. It remains to derive the value of $\alpha$. As the valid transcript must satisfy the verification condition
\begin{align*}
g_1^{\gamma_1}g_2^{\gamma_2}= y^{\beta}\alpha \quad\Leftrightarrow\quad
\alpha= g_1^{\gamma_1}g_2^{\gamma_2}\cdot y^{-\beta}
\end{align*}
we get the following simulator construction
\begin{align*}
\begin{fblock}{\SSS(\beta)}
&\gamma_1,\gamma_2\getsu\ZZ_q\\
&\alpha\gets g_1^{\gamma_1}g_2^{\gamma_2}\cdot y^{-\beta}\\
&\RETURN (\alpha,\gamma_1,\gamma_2)\enspace.
\end{fblock}
\end{align*}
For the complete proof, we should show that the simulation creates the same distribution as in the real protocol execution. The latter follows from two observations proved above:
\begin{itemize}
\item For fixed $\beta, \gamma_2,\gamma_2$ there exists only one $\alpha$ such that $(\alpha,\beta,\gamma_1,\gamma_2)$ is accepting protocol transcript.
\item In the protocol execution, the distribution of $(\beta,\gamma_1,\gamma_2)$ is uniform over $\ZZ_q\times\ZZ_q\times\ZZ_q$. 
\end{itemize}
Indeed, the simulator $\SIM$ generates first $(\beta,\gamma_1,\gamma_2)$ by uniform sampling and then picks the only possible $\alpha$ value. Thus, it must get the same message distribution. 

\end{solution}
\end{document}