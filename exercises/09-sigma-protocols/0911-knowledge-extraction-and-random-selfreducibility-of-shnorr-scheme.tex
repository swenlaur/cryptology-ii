\documentclass{crypto-exercise}
\usepackage{amsthm}
\author{Sven Laur}
\contributor{Sven Laur}
\tags{sigma protocol, discrete logaritm problem, special soundness, random self-reducibility}

\begin{document}
\begin{exercise}{Malleability of Schnorr identification scheme}
The Schnorr identification scheme  is directly based on the discrete logarithm problem. The identification scheme is a honest verifier zero-knowledge proof that the prover knows $x$ such that $g^x=y$ in a group $\GG$ of size $q$. The protocol itself
is following.
\begin{center}
  \begin{tabular}{ccc}
    $\PPP$ & & $\VVV$\\
    $r\getsu\ZZ_q$ \\
    &$\xrightarrow{\makebox[6cm]{$\alpha=g^r $}}$ \\
    && $\beta\getsu\ZZ_q$ \\
    &$\xleftarrow{\makebox[6cm]{$\beta$}}$\\
    \\  
    &$\xrightarrow{\makebox[6cm]{$\gamma= \beta x+r$}}$\\
    && $\bigl[g^{\gamma}\iseq y^\beta\alpha\bigr]$\\  
  \end{tabular}
\end{center}  
Show that if an honest $t$-time prover $\PPP^*$ that can convince the honest verifier with probability $\varepsilon$ on average over all $y\in\GG$ can also solve the discrete logarithm problem well enough. 
\end{exercise}

\begin{solution}
Consider a modified prover $\PPP^{**}$ that re-randomises the statement to be proven. That is it gets a statement $\POK_y[\exists x: g^x=y]$ and then asks $\PPP^*$ to prove $\POK_{y'}[\exists x': g^{x'}=y']$ for $y'=yg^{\delta}$.  Show how it can use the repies of $\PPP^*$ to pass  $\POK_y[\exists x: g^x=y]$. What does this mean on the success rate of $\PPP^{**}$ -- can there be more successful statements. 
\end{solution}
\end{document}