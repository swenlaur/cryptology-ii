\documentclass{crypto-exercise}
\usepackage{amsthm}
\author{Sven Laur}
\contributor{Sven Laur}
\editor{Sven Laur}
\tags{Simultaneous message exchange, input extraction, output equivocation, stateful protocols}

\newcommand{\REWRITE}{\qquad\scalebox{1.5}{$\rightsquigarrow$}\qquad}

\newcommand{\CS}{\mathcal{C}om}
\newcommand{\IEXTR}{\mathscr{K}}
\newcommand{\OEQUIV}{\mathscr{E}}
\newcommand{\FAIL}{\mathsf{Fail}}
\newcommand{\ELIF}{\mathsf{else\ if}\ }

\makeatletter
%\newcommand{\xleftrightarrow}[2][]{\ext@arrow 3359\leftrightarrowfill@{#1}{#2}}
\newcommand{\xdashrightarrow}[2][]{\ext@arrow 0359\rightarrowfill@@{#1}{#2}}
\newcommand{\xdashleftarrow}[2][]{\ext@arrow 3095\leftarrowfill@@{#1}{#2}}
\newcommand{\xdashleftrightarrow}[2][]{\ext@arrow 3359\leftrightarrowfill@@{#1}{#2}}
\def\rightarrowfill@@{\arrowfill@@\relax\relbar\rightarrow}
\def\leftarrowfill@@{\arrowfill@@\leftarrow\relbar\relax}
\def\leftrightarrowfill@@{\arrowfill@@\leftarrow\relbar\rightarrow}
\def\arrowfill@@#1#2#3#4{%
  $\m@th\thickmuskip0mu\medmuskip\thickmuskip\thinmuskip\thickmuskip
   \relax#4#1
   \xleaders\hbox{$#4#2$}\hfill
   #3$%
}
\makeatother

\begin{document}


\begin{exercise}{Security of simultaneous message exchange protocol}
Analyse security of the following  simplistic protocol for simultaneous message exchange 
\begin{align*}
&\PARTY_1(x_1) &&  &&\PARTY_2(x_2) \\
&&\xdashleftrightarrow[\hspace*{2cm}]{\fbox{$\PK\gets\GEN$}\atop {\phantom{ \textstyle c} \atop\PK}}\\
&(c,d)\gets\COM_\PK(x_1)\\
&&\xrightarrow[\hspace*{2cm}]{\PK,c}\\
&&\xleftarrow[\hspace*{2cm}]{x_2}\\
&&\xrightarrow[\hspace*{2cm}]{d}\\
&&&&& x_1\gets\OPEN_\PK(c,d)\\
&\RETURN (x_1,x_2)
&& &&\RETURN (x_1,x_2)
\end{align*}
where bits $x_1$ and $x_2$ are private protocol inputs and a triple of algorithms $(\GEN,\COM,\OPEN)$ is  a commitment scheme $\CS$ with appropriate properties. The dashed line denotes sub-protocol for fixing the commitment parameters.
In the following, we assume that the protocol has a trusted setup where parameter generation is done by a trusted third party. 
Consider security only against static malicious corruption. 
\end{exercise}

\begin{solution}
\ \\
\textsc{Right ideal implementation.}
As the first party $\PARTY_1$ can refuse to open its input based on the opponents input $x_2$, we must consider the  idealised functionality where the first party $\PARTY_1$ is in the dominant position:
\begin{align*}
&\PARTY_1(x_1) &&&& \TTP  &&&&\PARTY_2(x_2) \\
&&&\xrightarrow[\hspace*{2cm}]{x_1}
&&&\xleftarrow[\hspace*{2cm}]{x_2}\\
&&&\xleftarrow[\hspace*{2cm}]{x_2}\\
&&&\xrightarrow[\hspace*{2cm}]{\textsc{Abort}/\textsc{Continue}}\\
&&&&&&\xrightarrow[\hspace*{2cm}]{x_1}\\
&\RETURN (x_1,x_2)
&&&&&&&&\RETURN (x_1,x_2)
\end{align*}


\noindent
\textsc{\textcolor{red}{(e)} Output equivocation based on trusted setup.} 
We can use equivocal commitments to bypass problems in the output equivocation phase. But this leads to a setting with a trusted setup. 

\medskip
\noindent
\textcolor{red}{
\begin{itemize}
\item Construct the corresponding simulator for malicious $\PARTY_2$ by modifying the input extraction and output equivocation blocks
\item Prove that the corresponding simulator achieves the desired goal. That is, the joint output distributions are identical in the real and ideal world.   
\end{itemize}
}

\noindent
\textsc{\textcolor{red}{(f)} Input extraction based on trusted setup.} 
The simulation efficiency for a malicious $\PARTY_1^*$ depends on the size of $\XXX_2$ as the input extractor needs to iterate over all potential inputs $x_2$ to unlock the commitment. This problem can be bypassed if we use trusted setup with extractable commitment schemes.

\medskip
\noindent
\textcolor{red}{
\begin{itemize}
\item Construct the corresponding simulator for malicious $\PARTY_1$ by modifying the input extraction block so that its efficiency does not depend on the size of $\XXX_2$.
\item Prove that the corresponding simulator achieves the desired goal. That is, the joint output distributions are identical in the real and ideal world.   
\end{itemize}
}


\end{solution}
\end{document}

