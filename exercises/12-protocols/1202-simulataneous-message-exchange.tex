\documentclass{crypto-exercise}
\usepackage{amsthm}
\author{Sven Laur}
\contributor{Sven Laur}
\editor{Sven Laur}
\tags{Simultaneous message exchange, input extraction, output equivocation, stateful protocols}

\newcommand{\REWRITE}{\qquad\scalebox{1.5}{$\rightsquigarrow$}\qquad}

\newcommand{\CS}{\mathcal{C}om}
\newcommand{\IEXTR}{\mathscr{K}}
\newcommand{\OEQUIV}{\mathscr{E}}
\newcommand{\FAIL}{\mathsf{Fail}}
\newcommand{\ELIF}{\mathsf{else\ if}\ }

\makeatletter
%\newcommand{\xleftrightarrow}[2][]{\ext@arrow 3359\leftrightarrowfill@{#1}{#2}}
\newcommand{\xdashrightarrow}[2][]{\ext@arrow 0359\rightarrowfill@@{#1}{#2}}
\newcommand{\xdashleftarrow}[2][]{\ext@arrow 3095\leftarrowfill@@{#1}{#2}}
\newcommand{\xdashleftrightarrow}[2][]{\ext@arrow 3359\leftrightarrowfill@@{#1}{#2}}
\def\rightarrowfill@@{\arrowfill@@\relax\relbar\rightarrow}
\def\leftarrowfill@@{\arrowfill@@\leftarrow\relbar\relax}
\def\leftrightarrowfill@@{\arrowfill@@\leftarrow\relbar\rightarrow}
\def\arrowfill@@#1#2#3#4{%
  $\m@th\thickmuskip0mu\medmuskip\thickmuskip\thinmuskip\thickmuskip
   \relax#4#1
   \xleaders\hbox{$#4#2$}\hfill
   #3$%
}
\makeatother

\begin{document}


\begin{exercise}{Security of simultaneous message exchange protocol}
Analyse security of the following  simplistic protocol for simultaneous message exchange 
\begin{align*}
&\PARTY_1(x_1) &&  &&\PARTY_2(x_2) \\
&&\xdashleftrightarrow[\hspace*{2cm}]{\fbox{$\PK\gets\GEN$}\atop {\phantom{ \textstyle c} \atop\PK}}\\
&(c,d)\gets\COM_\PK(x_1)\\
&&\xrightarrow[\hspace*{2cm}]{\PK,c}\\
&&\xleftarrow[\hspace*{2cm}]{x_2}\\
&&\xrightarrow[\hspace*{2cm}]{d}\\
&&&&& x_1\gets\OPEN_\PK(c,d)\\
&\RETURN (x_1,x_2)
&& &&\RETURN (x_1,x_2)
\end{align*}
where bits $x_1$ and $x_2$ are private protocol inputs and a triple of algorithms $(\GEN,\COM,\OPEN)$ is  a commitment scheme $\CS$ with appropriate properties. The dashed line denotes sub-protocol for fixing the commitment parameters.
Prove that there exist an efficient simulator for $\PARTY_2$.
\end{exercise}

\begin{solution}
\ \\
\textsc{Right ideal implementation.}
As the first party $\PARTY_1$ can refuse to open its input based on the opponents input $x_2$, we must consider the  idealised functionality where the first party $\PARTY_1$ is in the dominant position:
\begin{align*}
&\PARTY_1(x_1) &&&& \TTP  &&&&\PARTY_2(x_2) \\
&&&\xrightarrow[\hspace*{2cm}]{x_1}
&&&\xleftarrow[\hspace*{2cm}]{x_2}\\
&&&\xleftarrow[\hspace*{2cm}]{x_2}\\
&&&\xrightarrow[\hspace*{2cm}]{\textsc{Abort}/\textsc{Continue}}\\
&&&&&&\xrightarrow[\hspace*{2cm}]{x_1}\\
&\RETURN (x_1,x_2)
&&&&&&&&\RETURN (x_1,x_2)
\end{align*}

\noindent
\textsc{High-level descriptions for simulator constructions.}
The interaction pattern is somewhat different if the second party $\PARTY_2$ is malicious. Then the corresponding simulator $\SIM$ must first provide an input $\hat{x}_2$ to the trusted third party $\TTP$ who replies $x_1$. After that the simulator must make $\PARTY_2^*$ to believe that honest party opened an input $x_1$ and the protocol outcome would be $(x_1, \hat{x}_2)$. As a result, the simulator consists of and input extraction followed by the output equivocation. If the commitment parameters are generated by $\PARTY_1$, we get the the simulator 
\begin{align*}
\begin{fblock}{\SIM(\phi_2,x_2)}
& \PK\gets\GEN\\
& \omega_2\getsu\Omega_2\\
& \hat{x}_2\gets \IEXTR(\phi_2,x_2,\PK,\omega_2)\\
& \text{Send $\hat{x}_2$ to $\TTP$ and store the reply as $x_1$}\\
& \RETURN \OEQUIV_\ell(\phi_2,x_1,x_2,\hat{x}_2,\PK,\omega_2)
\end{fblock}
\end{align*}
Note that the input extractor $\IEXTR$ and the output equivocator $\OEQUIV_\ell$  must share inputs and randomness used by $\PARTY_2^*$ or otherwise we cannot assure that the actions of $\PARTY_2^*$ are consistent between both algorithms. The consistency is essential for getting a simulation with the right output distribution. If commitment parameters are generated by $\PARTY_2$ then the plumbing between the simulator components changes
\begin{align*}
\begin{fblock}{\SIM(\phi_2,x_2)}
& \omega_2\getsu\Omega_2\\
& \PK\gets\PARTY_2^*(\phi_2,x_2;\omega_2)\\
& \hat{x}_2\gets \IEXTR(\phi_2,x_2,\omega_2)\\
& \text{Send $\hat{x}_2$ to $\TTP$ and store the reply as $x_1$}\\
& \RETURN \OEQUIV_\ell(\phi_2,x_1,x_2,\hat{x}_2,\PK,\omega_2)
\end{fblock}
\end{align*}
but the overall scheme remains the same. 


\medskip
\noindent
\textsc{\textcolor{red}{(b)} Input extractor for $\PARTY_2$.}
First note that the input extractor for $\PARTY_2^*$ must work in a black-box manner. The reasoning is analogous to the reasoning given for malicious $\PARTY_1^*$. As $\PARTY_2$ releases its actual input $\hat{x}_2$ only after seeing $c_\circ$ we must provide some sort of commitment during extraction. However, differently form its opponent $\PARTY_2^*$ reply may freely depend on $c_\circ$ and thus the semantics $\hat{x}_2$ is somewhat different -- it is the input which can be later successfully combined with the revealed input $x_1$. This means that the quality of input extraction must be considered together with output equivocation. If commitment parameters $\PK$ are generated by the opponent $\PARTY_1$ then the most natural input extraction strategy is following  
\begin{align*}
\begin{fblock}{\IEXTR(\phi_2,x_2,\PK,\omega_2)}
& (c_\circ, d_\circ)\gets\COM_\PK(0)\\
& \hat{x}_2\gets \PARTY_2^*(\phi_2,x_2,\PK;\omega_2)\\
& \RETURN \hat{x}_2\\
\end{fblock}
\end{align*}
where we set $\hat{x}_2$ to $\bot$ if $\PARTY_2^*$ refuses to communicate after obtaining $c_\circ$. If commitment parameters $\PK$ are generated internally by $\PARTY_2^*$ then the most natural input extraction strategy is following  
\begin{align*}
\begin{fblock}{\IEXTR(\phi_2,x_2,\omega_2)}
& \PK\gets\PARTY_2^*(\phi_2,x_2;\omega_2)\\
& (c_\circ, d_\circ)\gets\COM_\PK(0)\\
& \hat{x}_2\gets \PARTY_2^*(x_2,\phi_2,\PK;\omega_2)\\
& \RETURN \hat{x}_2\\
\end{fblock}
\end{align*}

\medskip
\noindent
\textcolor{red}{
Prove the following facts
\begin{itemize}
\item If the commitment is perfectly hiding then the protocol output $y_1$ of $\PARTY_1$ is the same in the real and ideal world. Note that the output is completely determined by the values $(\phi_2,x_2,\omega_2, x_1)$ and thus can be considered as a deterministic  function $y_1(\phi_2,x_2,\omega_2, x_1)$.
\item Analyse what changes if we consider the setting with computationally hiding commitments where $\PK$ is provided by $\PARTY_1$. Show that corresponding distributions must be computationally indistinguishable. How is the corresponding time-bound connected to the running-time of the extractor.  
\item Show that if the number of possible input values $\XXX_2$ is small then the computational distance and statistical distance are equivalent, i.e., likelihood ratio test is efficient. 
\end{itemize}
} 
   
\medskip
\noindent
\textsc{\textcolor{red}{(c)} Limited form of output equivocation for $\PARTY_2$.}  
Although the simulator using the input extractor $\IEXTR$ can perfectly match the output distribution of honest $\PARTY_1$, we need to show closeness of the joint output distribution.  This is straightforward for a limited class of malicious adversaries $\PARTY_2^*$ that consist of two algorithms $\AD$ and $\ADB$ with isolated states that are sequentially combined
\begin{align*}
\hspace*{2cm}&\PARTY_1&&&&\PARTY_2^*(\phi_2,x_2)\\
&&&&&\omega_{\mathrm{a}}\gets\Omega_2\\
&&&&&\PK\gets\AD(\phi_2,x_2;\omega_a)\\
&&\xleftarrow[\hspace*{2cm}]{\PK}\\
&\ldots&\xrightarrow[\hspace*{2cm}]{c}\\
&&&&& \hat{x}_2\gets\AD\\
&&\xleftarrow[\hspace*{2cm}]{\hat{x_2}}\\
&\ldots&\xrightarrow[\hspace*{2cm}]{d}\\
&&&&& x_1\gets\OPEN_\PK(c,d)\\
&\ldots&&&& \RETURN \ADB(\phi_2,x_2,x_1,\hat{x}_2,\omega_\mathrm{a})\\
\end{align*}
For such adversaries, the output equivocator is following
\begin{align*}
\begin{fblock}{\OEQUIV(\phi_2,x_1,x_2,\hat{x}_2,\PK,\omega_2)}
& \text{Split $\omega_2$ into $\omega_\mathrm{a}$ and $\omega_{\mathrm{b}}$}\\
& \RETURN \ADB(\phi_2,x_2,x_1,\hat{x}_2,\omega_\mathrm{a};\omega_{\mathrm{b}})\\
\end{fblock}
\end{align*}
The randomness splitting is trivial if algorithms $\AD$ and $\ADB$ have explicit description of the number f used random bits. If this is implicit, we can split the randomness by running $\AD$ with $\omega_2$ and set $\omega_{\mathrm{b}}$ as the list of unused bits. Thus, the randomness splitting step is relatively efficient.

\medskip
\textcolor{red}{
\begin{itemize}
\item Prove that if the commitment is perfectly hiding and the adversary has the structure described above then 
 the joint output distributions in the real and ideal world are identical.
\item Interpret the result. For which kind of security goals the malicious adversary $\PARTY_2^*$ might obtain a significant gain is speed or in success. For that note that an isolated adversary $\ADB$ might completely restore the state of $\AD$ is we additionally give him the randomness used to create the commitment decommitment pair $(c,d)$. Are the class of neglected security goals relevant in the practice.  
\end{itemize}
}   

\medskip
\noindent
\textsc{\textcolor{red}{(d)} Complete output equivocation for $\PARTY_2$.} 
To protect against all attack goals, we need equivocation algorithm works for malicious adversaries without structural restrictions. Let  $\ell\in\NN$ be a parameter determines a tradeoff between efficiency and quality of simulation. Then the following algorithm 
\begin{align*}
&\begin{fblock}{\OEQUIV_\ell(\phi_2,x_1,x_2,\hat{x}_2,\PK,\omega_2)}
&\begin{forblock}{i\in\set{1,\ldots,\ell}\ }
&(c,d)\gets\COM_\PK(x_1)\\
& x_2^*\gets \PARTY_2^*(x_2,\phi_2,\PK;\omega_2)\\
& \IF x_2^*=\hat{x}_2\ \THEN \RETURN \PARTY_2^*(d)
&\end{forblock}\\
&\RETURN \FAIL
\end{fblock}
\end{align*}
performs rejection sampling over all possible protocol runs with the opponents input $x_1$ and $\PARTY_2$ reply $\hat{x}_2$ and thus gets the desired output distribution when $\OEQUIV$ does not fail.  
 
\medskip
\noindent
\textcolor{red}{
Prove the following facts
\begin{itemize}
\item Assume that if the commitment is perfectly hiding and let $\hat{x}_2$ be the actual input fixed by the input extractor. Estimate the probability that $\OEQUIV_\ell$ returns $\FAIL$ as a function of probability $p(\hat{x}_2)=\pr{\KEXTR=\hat{x}_2}$.
\item Let $\XXX_2$ be the set of all potential inputs for $\PARTY_2$ compute the maximal failure probability when $\hat{x}_2$ is sampled by $\IEXTR$.
\item Prove that the statistical difference between joint output distributions is equal to the probability that $\pr{\OEQUIV_\ell=\FAIL}$.  
\item Interpret results. How does the efficiency depend on desired statistical distance
\item What changes if the commitment is only computationally hiding. 
\end{itemize}
}

\end{solution}
\end{document}

