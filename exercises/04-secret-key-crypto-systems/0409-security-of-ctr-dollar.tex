\documentclass{crypto-exercise}
\usepackage{amsthm}
\author{Sven Laur}
\contributor[Initial solution which need more work]{Sven Laur}
\editor{Sven Laur}
\tags{CTR-$ encryption scheme, horizon-splitting, ind-cpa security}

\newcommand{\COLL}{\mathsf{Coll}}
\newcommand{\CTRS}{\textsc{Ctr-\$}}

\begin{document}
\begin{exercise}{Security of $\CTRS$ encryption scheme.} 
Let $f:\MSPACE\times\KSPACE\to\MSPACE$ be a $(t,
\varepsilon)$-pseudorandom permutation and let $\CTRS$ be a symmetric
encryption scheme defined as follows:
\begin{itemize}
\item A secret key is a randomly chosen $k\getsu\KSPACE$.
\item To encrypt a message $m_1,\ldots,m_n$, choose a random nonce
  $s_0\getsu\MSPACE$ and output a ciphertext vector
  $s_0,m_1+f(s_0+1,k),\ldots, m_n+f(s_0+n,k)$.
\item To decrypt $s_0,c_1,\ldots,c_n$, output
  $c_1-f(s_0+1,k),\ldots,c_n-f(s_0+n,k)$.
\end{itemize}
Prove that $\CTRS$ is IND-CPA secure cryptosystem.
\end{exercise}

\begin{solution}
Recall that  a symmetric cryptosystem 
$\mathscr{C}=(\GEN,\ENC,\DEC)$ is $(t,\varepsilon)$-IND-CPA secure, if
for any $t$-time adversary $\AD$:
\begin{align*}
  \advINDCPAXX{\mathscr{C}}{\AD}=\abs{\pr{\smash{\GAME_0^\AD=1}}-\pr{\smash{\BGAME_1^\AD=1}}}\leq
  \varepsilon
\end{align*}
where the oracle $\ORA_1(\cdot)$ in  the indistinguishability games
\begin{align*}
  \begin{game}{\BGAME_0^\AD}
    &\SK\gets\GEN\\
    &(m_0,m_1)\gets\AD^{\ORA_1(\cdot)}\\
    &\RETURN \AD^{\ORA_1(\cdot)}(\ENC_\SK(m_0))
  \end{game}
  \qquad\qquad
  \begin{game}{\BGAME_1^\AD}
    &\SK\gets\GEN\\
    &(m_0,m_1)\gets\AD^{\ORA_1(\cdot)}\\
    &\RETURN \AD^{\ORA_1(\cdot)}(\ENC_\SK(m_1))
  \end{game}
\end{align*}
serves encryption calls. As the first step towards a solution, lets
substitute the definition of $\CTRS$ encryption scheme into the games
$\BGAME_0$ and $\BGAME_1$:
\begin{align*}
  \begin{game}{\BGAME_0^\AD}
    &k\getsu\KSPACE\\
    &(\vec{m}_0,\vec{m}_0)\gets\AD^{\ORA_1(\cdot)}\\
    &s_0\getsu \MSPACE\\ 
    &\begin{forblock}{i\in\set{1,\ldots,n}}
    & c_i\gets m_0^i+f(s_0 + i, k)\\
    \end{forblock}\\
    &\RETURN \AD^{\ORA_1(\cdot)}(s_0,c_1,\ldots,c_n)
  \end{game}
  \qquad\qquad
  \begin{game}{\BGAME_1^\AD}
    &k\getsu\KSPACE\\
    &(\vec{m}_0,\vec{m}_1)\gets\AD^{\ORA_1(\cdot)}\\
    &s_0\getsu \MSPACE\\ 
    &\begin{forblock}{i\in\set{1,\ldots,n}}
    & c_i\gets m_0^i+f(s_0 + i, k)\\
    \end{forblock}\\
    &\RETURN \AD^{\ORA_1(\cdot)}(s_0,c_1,\ldots,c_n)
  \end{game}
\end{align*}
As the second step, we can replace all invocations of a pseudorandom
function $f_k(x)=f(x,k)$ with a truly random function. Note that the
function $f$ is invoked not only in the explicit encryption call but also
in the oracle $\ORA_1(\cdot)$. Let $\ORA_f(\cdot)$ denote the new
encryption oracle with the following construction
\begin{align*}
  \begin{fblock}{\ORA_f(m_1,\ldots,m_n)}
    &\overline{s}_0\getsu \MSPACE\\ 
    &\begin{forblock}{i\in\set{1,\ldots,n}}
    & \overline{c}_i\gets m_0^i+f(s_0 + i)\\
    \end{forblock}\\    
    &\RETURN (\overline{s}_0,\overline{c}_1,\ldots,\overline{c}_n)
  \end{fblock}
\end{align*}
and let $\GAME_0$ and $\GAME_1$ denote the resulting games
\begin{align*}
    \begin{game}{\GAME_0^\AD}
    &f\getsu\FFFALL\\
    &(\vec{m}_0,\vec{m}_0)\gets\AD^{\ORA_f(\cdot)}\\
    &s_0\getsu \MSPACE\\ 
    &\begin{forblock}{i\in\set{1,\ldots,n}}
    & c_i\gets m_0^i+f(s_0 + i)\\
    \end{forblock}\\
    &\RETURN \AD^{\ORA_f(\cdot)}(s_0,c_1,\ldots,c_n)
  \end{game}
  \qquad\qquad
  \begin{game}{\GAME_1^\AD}
    &f\getsu\FFFALL\\
    &(\vec{m}_0,\vec{m}_1)\gets\AD^{\ORA_f(\cdot)}\\
    &s_0\getsu \MSPACE\\ 
    &\begin{forblock}{i\in\set{1,\ldots,n}}
    & c_i\gets m_0^i+f(s_0 + i)\\
    \end{forblock}\\
    &\RETURN \AD^{\ORA_f(\cdot)}(s_0,c_1,\ldots,c_n)
  \end{game}
\end{align*}
As the function $f$ is $(t,\varepsilon)$-pseudorandom function, we can
easily prove that
\begin{align}
 &\abs{\pr{\BGAME_0^\AD=1}-\pr{\GAME_0^\AD=1}}\leq \varepsilon\enspace\label{eq:step-i}\\
 &\abs{\pr{\BGAME_1^\AD=1}-\pr{\GAME_1^\AD=1}}\leq \varepsilon\label{eq:step-ii}\enspace
\end{align}
for all adversaries $\AD$ with sufficiently small running
time. Indeed, let $\BGAME_2$ and $\BGAME_3$ be the
indistinguishability games for pseudorandom functions:
\begin{align*}
  \begin{game}{\BGAME_2^{\ADB}}
    &f\gets\FFFALL\\
    &\RETURN \ADB^{f(\cdot)}
  \end{game}\qquad\qquad
  \begin{game}{\BGAME_3^{\ADB}}
    &f\gets\FFF\\
    &\RETURN \ADB^{f(\cdot)}
  \end{game}
\end{align*}
Then we can define adversaries $\ADB_0$ and $\ADB_1$ by in-lining the
common parts of game pairs $\BGAME_0$ and $\GAME_0$ and $\BGAME_1$ and
$\GAME_1$ into the adversary construction
\begin{align*}
    \begin{fblock}{\ADB_0^{f(\cdot)}}
    &(\vec{m}_0,\vec{m}_0)\gets\AD^{\ORA_f(\cdot)}\\
    &s_0\getsu \MSPACE\\ 
    &\begin{forblock}{i\in\set{1,\ldots,n}}
    & c_i\gets m_0^i+f(s_0 + i)\\
    \end{forblock}\\
    &\RETURN \AD^{\ORA_f(\cdot)}(s_0,c_1,\ldots,c_n)
  \end{fblock}
  \qquad\qquad
  \begin{fblock}{\ADB_1^{f(\cdot)}}
    &(\vec{m}_0,\vec{m}_1)\gets\AD^{\ORA_f(\cdot)}\\
    &s_0\getsu \MSPACE\\ 
    &\begin{forblock}{i\in\set{1,\ldots,n}}
    & c_i\gets m_1^i+f(s_0 + i)\\
    \end{forblock}\\
    &\RETURN \AD^{\ORA_f(\cdot)}(s_0,c_1,\ldots,c_n)
  \end{fblock}
\end{align*}
As a result, it is straightforward to prove that
$\BGAME_2^{\ADB_i}\equiv\BGAME_i^\AD$ and $\BGAME_3^{\ADB_i}\equiv
\GAME_i^\AD$. Hence, the
equations~\eqref{eq:step-i}~and~\eqref{eq:step-ii} must hold as soon
as the running times of $\ADB_0$ and $\ADB_1$ are less or equal to
$t$. 

Unfortunately, these running times depend on the number of
encryption queries. Let $q$ be the number of oracle calls to $\ORA_f$
made by $\AD$ and $t_f$ be the time penalty of a single call to the
oracle $f(\cdot)$. Then the running time of $\ADB_i$ is $t_\AD+nq\cdot
t_f+\Oh(n)$ and consequently the running time of $\AD$ must be below
$t-nq\cdot t_f-\Oh(n)$ for the
equations~\eqref{eq:step-i}~and~\eqref{eq:step-ii} to hold.
 
 
For the further analysis, there are two possible options.  The most
straightforward way is to substitute also the encryption oracle
$\ORA_f(\cdot)$ into the games according to its
specification. However, this would produce rather lengthy games that
are difficult to analyse by hand. A more elegant solution is based on
\emph{horizon splitting}. Let $\COLL$ denote the event that a range
$[s_0 + 1,\ldots, s_0 + n]$ for the challenge encryption overlaps with
range $[\overline{s}_0 + 1,\ldots, \overline{s}_0 + n]$ for another
encryption generated by the oracle $\ORA_f(\cdot)$ during the
game. Then obviously
\begin{align*}
\pr{\GAME_0^\AD=1} &= \pr{\GAME_0^\AD=1\wedge \COLL} + \pr{\GAME_0^\AD=1\wedge \neg\COLL}\enspace,\\
\pr{\GAME_1^\AD=1} &= \pr{\GAME_1^\AD=1\wedge \COLL} + \pr{\GAME_1^\AD=1\wedge \neg\COLL}\enspace,
\end{align*}
and by triangle inequality 
\begin{eqnarray*}
   \advINDXX{\GAME_0,\GAME_1}{\AD} &\leq& \abs{\pr{\GAME_0^\AD=1\wedge \COLL}-\pr{\GAME_1^\AD=1\wedge \COLL}}\\
 &&+ \abs{\pr{\GAME_0^\AD=1\wedge \neg\COLL}-\pr{\GAME_1^\AD=1\wedge \neg\COLL}}\enspace.
\end{eqnarray*}
Hence, we can estimate the advantage $\advINDXX{\GAME_0,\GAME_1}{\AD}$ in terms of
four new games
\begin{align*}
  \begin{game}{\GAME_2^\AD}
    &f\getsu\FFFALL\\
    &(\vec{m}_0,\vec{m}_0)\gets\AD^{\ORA_f(\cdot)}\\
    &s_0\getsu \MSPACE\\ 
    &\begin{forblock}{i\in\set{1,\ldots,n}}
    & c_i\gets m_0^i+f(s_0 + i)\\
    \end{forblock}\\
    &\IF \neg\COLL\ \THEN \RETURN \bot\\
    &\RETURN \AD^{\ORA_f(\cdot)}(s_0,c_1,\ldots,c_n)
  \end{game}
  \qquad\qquad
  \begin{game}{\GAME_3^\AD}
    &f\getsu\FFFALL\\
    &(\vec{m}_0,\vec{m}_1)\gets\AD^{\ORA_f(\cdot)}\\
    &s_0\getsu \MSPACE\\ 
    &\begin{forblock}{i\in\set{1,\ldots,n}}
    & c_i\gets m_0^i+f(s_0 + i)\\
    \end{forblock}\\
    &\IF \neg\COLL\ \THEN \RETURN \bot\\
    &\RETURN \AD^{\ORA_1(\cdot)}(s_0,c_1,\ldots,c_n)
  \end{game}\\
  \begin{game}{\GAME_4^\AD}
    &f\getsu\FFFALL\\
    &(\vec{m}_0,\vec{m}_0)\gets\AD^{\ORA_f(\cdot)}\\
    &s_0\getsu \MSPACE\\ 
    &\begin{forblock}{i\in\set{1,\ldots,n}}
    & c_i\gets m_0^i+f(s_0 + i)\\
    \end{forblock}\\
    &\IF \neg\COLL\ \THEN \RETURN \bot\\
    &\RETURN \AD^{\ORA_f(\cdot)}(s_0,c_1,\ldots,c_n)
  \end{game}
  \qquad\qquad
  \begin{game}{\GAME_5^\AD}
    &f\getsu\FFFALL\\
    &(\vec{m}_0,\vec{m}_1)\gets\AD^{\ORA_f(\cdot)}\\
    &s_0\getsu \MSPACE\\ 
    &\begin{forblock}{i\in\set{1,\ldots,n}}
    & c_i\gets m_0^i+f(s_0 + i)\\
    \end{forblock}\\
    &\IF \neg\COLL\ \THEN \RETURN \bot\\
    &\RETURN \AD^{\ORA_f(\cdot)}(s_0,c_1,\ldots,c_n)
  \end{game}
\end{align*}
Lets analyse the games $\GAME_4$ and $\GAME_5$ first. Since both games
end with $\bot$ when a collision occurs, the invocations of $f(s_0+i)$
can be replaced with uniform sampling $y_i\getsu\MSPACE$. The cases
where this would be detectable are guaranteed to end with $\bot$
anyway. As $m_i+y_i$ for $y_i\getsu\MSPACE$ has a uniform
distribution, we can further simplify the games without changing their
semantics. Let $\GAME_6$ and $\GAME_7$ denote the resulting games
\begin{align*}
  \begin{game}{\GAME_6^\AD}
    &f\getsu\FFFALL\\
    &(\vec{m}_0,\vec{m}_0)\gets\AD^{\ORA_f(\cdot)}\\
    &s_0\getsu \MSPACE\\ 
    &\begin{forblock}{i\in\set{1,\ldots,n}}
    & c_i\getsu\MSPACE\\
    \end{forblock}\\
    &\IF \neg\COLL\ \THEN \RETURN \bot\\
    &\RETURN \AD^{\ORA_f(\cdot)}(s_0,c_1,\ldots,c_n)
  \end{game}
  \qquad\qquad
  \begin{game}{\GAME_7^\AD}
    &f\getsu\FFFALL\\
    &(\vec{m}_0,\vec{m}_1)\gets\AD^{\ORA_f(\cdot)}\\
    &s_0\getsu \MSPACE\\ 
    &\begin{forblock}{i\in\set{1,\ldots,n}}
    & c_i\getsu\MSPACE\\
    \end{forblock}\\
    &\IF \neg\COLL\ \THEN \RETURN \bot\\
    &\RETURN \AD^{\ORA_f(\cdot)}(s_0,c_1,\ldots,c_n)
  \end{game}
\end{align*}
Then by the reasoning given above
$\GAME_4^\AD\equiv\GAME_6^\AD=\GAME_7^\AD\equiv \GAME_5^\AD$ and
consequently
\begin{eqnarray*}
  \advINDXX{\GAME_0,\GAME_1}{\AD}
  \leq \abs{\pr{\GAME_0^\AD=1\wedge \COLL}-\pr{\GAME_1^\AD=1\wedge \COLL}}
  =\advINDXX{\GAME_2,\GAME_3}{\AD}\enspace.
\end{eqnarray*}
Note that if ranges $[s_0 + 1,\ldots, s_0 + n]$ and $[\overline{s}_0 +
1,\ldots, \overline{s}_0 + n]$ overlap at a position $s_0+i$, the
value $f(s_0+i)$ becomes public and $\AD$ can easily detect whether
$\vec{m}_0$ or $\vec{m}_1$ was encrypted as long as $m_0^i\neq
m_1^i$. Hence, we use only a trivial bound
\begin{align*}
  \advINDXX{\GAME_2,\GAME_3}{\AD}%
  \leq \max\set{\pr{\COLL \text{ in } \GAME_0}, \pr{\COLL \text{ in } \GAME_1}}\enspace.
\end{align*}
Fortunately, the collision probability depends only on the number of
oracle calls. For a fixed range $[s_0 + 1,\ldots, s_0 + n]$ and
uniformly chosen $\overline{s}_0$, the probability of an overlap is
$\frac{2n-1}{\abs{\MSPACE}}$. As $\AD$ makes $q$ encryption calls, the
union bound yields
\begin{align*}
 \advINDXX{\GAME_2,\GAME_3}{\AD}\leq\frac{q(2n-1)}{\abs{\MSPACE}}\enspace.
\end{align*}
To summarise, we have established that the IND-CPA advantage of an
adversary $\AD$ with a running time $t-nq\cdot t_f-\Oh(n)$ can be bounded:
\begin{align*}
  \advINDCPAXX{\CTRS}{\AD}\leq \frac{q(2n-1)}{\abs{\MSPACE}}+ 2\varepsilon
\end{align*}
where $q$ is the upper bound on the encryption queries. This result is
precise but contains unknown value $q$. Since $q\leq t_\AD$ for
obvious reasons, the inequality $t_\AD\leq t-nt_\AD\cdot t_f-\Oh(n)$
about running time implies
\begin{align*}
 t_\AD\leq\frac{t-\Oh(n)}{1+n t_f}
\end{align*}
and consequently
\begin{align*}
  \advINDCPAXX{\CTRS}{\AD}\leq \frac{t-\Oh(n)}{1+n t_f}\cdot\frac{(2n-1)}{\abs{\MSPACE}}+ 2\varepsilon\enspace.
\end{align*}

\paragraph{Remark.} The horizon splitting technique is rather powerful
and allows us to analyse many settings by splitting the proof into
different branches. It can be easily generalised for estimating the
success against a single game, as well.
\end{solution}


\end{document}
