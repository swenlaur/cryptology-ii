\documentclass{crypto-exercise}
\usepackage{amsthm}
\author{Sven Laur}
\tags{semantic security, trivial attack}


\begin{document}
\begin{exercise}{Alternative definition for semantic security}
The standard notion of semantic security is  defined through the following games:
\begin{align*}
  \begin{game}{\GAME_0^\AD}
    &\SK\gets\GEN\\
    &\MSPACE_0\gets\AD^{\ENC_\SK(\cdot)}\\
    &m\gets\MSPACE_0\\
    &c\gets\ENC_\SK(m)\\
    &\RETURN [g(m)\iseq \AD(c)]
  \end{game}
  \qquad\qquad
  \begin{game}{\GAME_1^\AD}
    &\SK\gets\GEN\\
    &\MSPACE_0\gets\AD^{\ENC_\SK(\cdot)}\\
    &m\gets\MSPACE_0, \overline{m}\gets\MSPACE_0\\
    &\overline{c}\gets\ENC_\SK(\overline{m})\\
    &\RETURN [g(m)\iseq \AD(\overline{c})]
  \end{game}
\end{align*}
where the second game $\GAME_1$ models a very specific attack in the setting where the adversary does not see the encryption of a challenge message.
This does not reflect reality close enough as the adversary can perform other more successful attacks in this setting. To capture that we define a new security game
\begin{align*}
  \begin{game}{\GAME_2^{\AD_*}}
    &\SK\gets\GEN\\
    &\MSPACE_0\gets\AD_*^{\ENC_\SK(\cdot)}\\
    &m\gets\MSPACE_0\\
    &\RETURN [g(m)\iseq \AD_*]
  \end{game}
\end{align*}
This allows us to define two advantages
\begin{align*}
 \ADV_g^{\mathsf{sem}}(\AD)&=\pr{\GAME_0^\AD=1}-\pr{\GAME_1^\AD=1}\\
\ADV_g^{\mathsf{sem*}}(\AD,\AD_*)&=\pr{\GAME_0^\AD=1}-\pr{\GAME_2^{\smash{\AD_*}}=1}
\end{align*}
The cryptosystem is $(t,\varepsilon)$-weakly semantically secure if for any $t$-time adversaries $\AD$ and $\AD_*$ the advantage 
$\ADV_g^{\mathsf{sem*}}(\AD,\AD_*)\leq \varepsilon$. Prove that semantic security implies weak semantic security for the same function $g$.
Show that  for large enough $t$ it is possible to get $\pr{\GAME_1^\AD=1}\ll\pr{\GAME_2^{\smash{\AD_*}}=1}$ for some adversaries $\AD$. Does this mean that weak semantic security does not imply semantic security? 
\end{exercise}

\begin{solution}
\end{solution}

\end{document}