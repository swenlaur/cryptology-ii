\documentclass{crypto-exercise}
\usepackage{amsthm}
\author{Sven Laur}
\contributor[Initial solution]{Margus Niitsoo}
\editor{Sven Laur}
\tags{missing illustration, attack construction, Feistel cipher, pseudorandom permutation}


\begin{document}
\begin{exercise}{Insecurity of three round Feistel cipher}
Show that the tree-round Feistel cipher
$\textsc{Feistel}_{f_1,f_2,f_3}(L_0\|R_0)$ is not pseudorandom if the
adversary can also make decryption queries.
\end{exercise}

\begin{solution}
Let $L_0\|R_0$ be an arbitrary message. Then the corresponding
ciphertexts is 
\begin{align*}
  L_3&=R_0\oplus f_2(L_0\oplus f_1(R_0))\enspace,\\
  R_3&=L_0\oplus f_1(R_0)\oplus f_3(R_0\oplus f_2(L_0\oplus f_1(R_0))\enspace.
\end{align*}
Now the ciphertext of a modified message $L_0\oplus\delta\|R_0$ is
\begin{align*}
  L_3'&=R_0\oplus f_2(L_0\oplus\delta\oplus f_1(R_0))\enspace,\\
  R_3'&=L_0\oplus \delta\oplus f_1(R)\oplus f_3(R_0\oplus f_2(L_0\oplus \delta\oplus f_1(R_0))\enspace.
\end{align*}
As a next step, we can use decryption operation to find $L_0^*\|R_0^*$
such that the corresponding ciphertext is
\begin{align*}
  L_3^*&=L_3'\oplus 0=R_0\oplus f_2(L_0\oplus\delta\oplus f_1(R_0))\enspace,\\
  R_3^*&=R_3'\oplus\delta=L_0\oplus f_1(R_0)\oplus f_3(R_0\oplus f_2(L_0\oplus \delta\oplus f_1(R_0))\enspace.
\end{align*}
By the definition of the Feistel cipher we can express
\begin{align*}
  L_2^*&=R_3^*\oplus f_3(L_3^*)=L_0\oplus f_1(R_0)=L_2\enspace,\\
  L_1^*&=R_2^*\oplus f_2(L_2^*)= R_2^*\oplus f_2(L_2)=L_3^*\oplus f_2(L_2) \enspace,\\
  R_0^*&=L_1^*=L_3^*\oplus f_2(L_2)\enspace.
\end{align*}
Similarly, we can derive
\begin{align*}
    R_0&=L_1=R_2\oplus f_2(L_2)=L_3\oplus f(L_2)
\end{align*}
and thus we have obtained a relation
\begin{align*}
  R_0^*\oplus L_3^*= f_2(L_2)=R_0\oplus L_3
\end{align*}
that holds with probability $1$. The same relation between input and
output pairs holds with probability 
\begin{align*}
  \frac{1}{2^n-2}
\end{align*}
for random permutation. Hence, the computational difference is really
small for the three round Feistel cipher if decryption operations are
allowed.
\end{solution}

\end{document}