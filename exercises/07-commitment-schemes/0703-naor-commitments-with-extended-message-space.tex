\documentclass{crypto-exercise}
\author{Sven Laur}
\tags{Naor commitment, hiding, binding}


\begin{document}
\begin{exercise}{Naor commitments with extended message space}
The main drawback of the Naor commitment scheme is message expansion -- to commit one bit one must send $n$ bits. 
One possibility is to increase the size of the message space.  
Let the message space $\MSPACE$ be a subset of a finite field $(\FF_{2^n};+,\times)$ such that we can treat all $n$-bit strings as elements of $\FF_{2^n}$. 
Then we can define modified commitment scheme:
\begin{align*}
  \begin{aligned}
  \begin{fblock}{\GEN}
    & \PK\getsu\FF_{2^n}^*\\
    &\RETURN \PK       
    \end{fblock}\\
    \
  \end{aligned}
  \qquad
  \begin{fblock}{\COM_\PK(x)}
    & d\gets\set{0,1}^k\\
    & c\gets f(d)+ x\times \PK\\
    & \RETURN (c,d) 
  \end{fblock}
  \qquad
  \begin{fblock}{\OPEN_\PK(c,d)}
    &y\gets c\oplus f(d)\\
    &\IF y\notin\PK\times\MSPACE\ \THEN \RETURN \bot\\
    &\ELSE \RETURN y\times \PK^{-1}
  \end{fblock}
\end{align*}
Establish the corresponding security guarantees under the assumption that $f:\set{0,1}^k\to\set{0,1}^n$ is a $(t_1,\varepsilon_1)$-pseudorandom generator. How big must be the message space $\MSPACE\subseteq \FF_{2^n}$ to achieve reasonable security guarantees against double openings?
\end{exercise}

\begin{solution} \

\vspace*{2ex}
\noindent
\textsc{Binding}.
The outcome $c, d_1,d_2$  of an adversary $\AD$ can be double opening only if $\PK$ is a solution to equation \ldots.
As this equation can have at most \ldots solutions the number of public keys that can lead to a double opening is bounded by \ldots. Consequently, \ldots

\vspace*{2ex}
\noindent
\textsc{Hidding}.
Recall that commitment scheme is $(t,\varepsilon)$-hiding if any $t$-time adversary \ldots
Recall that a function $f$ is a $(t,\varepsilon)$-pseudorandom generator if \ldots 

\vspace*{2ex}
\noindent
\textsc{Qualitative analysis of the binding bound}.\ldots as a result the size of the message space $\MMM$ is bounded by \ldots



\end{solution}



\end{document}
