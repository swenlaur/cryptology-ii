\documentclass{crypto-exercise}
\usepackage{amsthm}
\author{Sven Laur}
\contributor[Initial solution]{Prastudy Fauzi}
\contributor[Streamlining]{Sven Laur}

\editor{Sven Laur}
\tags{user-aided key agreement, ind-cca2 security, non-malleable encryption}

\newcommand{\ADVNMCPA}[2]{\ADV^{\mathsf{nm\text{-}cpa}}_{#1}{(#2)}}
\newcommand{\CS}{\mathfrak{C}}

\begin{document}
\begin{exercise}{Security of user-aided key agreement}
Consider the following simple user-aided key agreement protocol. The public key $\PK$ of a server $\PARTY_1$ is known to all participants. If a participant $\PARTY_2$ wants to connect to $\PARTY_1$ it generates a random session key $k\getsu\KSPACE$ and a short authentication nonce $r\getsu\set{0,\ldots,9999}$ and sends $\ENC_\PK(k\|r)$ to $\PARTY_1$. Next $\PARTY_1$ recovers $k$ and $r$ and sends $r$ as an SMS back to $\PARTY_2$. The client $\PARTY_2$ halts if the SMS does not correspond to his or her authentication nonce.
Prove that a $t$-time adversary can alter the ciphertext without being detected with probability at most $10^{-4}+\varepsilon$ provided that the cryptosystem is $(t,\varepsilon)$-NM-CPA secure and no adversary cannot alter the SMS message.
\end{exercise}
\begin{solution}
For brevity, let $\RRR=\set{0000,\ldots,9999}$ denote the nonce space. Then we can formalise the security goal through the following game: 
\begin{align*}
\begin{game}{\GAME_0}
 &(\PK,\SK)\gets\GEN\\
 &k\getsu\KSPACE, r\getsu\RRR\\
 &c\gets\ENC_\PK(k||r)\\
 &\hat{c}\gets\AD(c)\\
 &\hat{k}||\hat{r}\gets\DEC_\SK(\hat{c})\\
 &\IF r\neq \hat{r}\ \RETURN 0\\
 &\RETURN \neg[k\iseq \hat{k}]\enspace.
\end{game}
\end{align*}
Note that if the adversary return $\hat{c}=c$, he or she is guaranteed to lose the game.
Hence, we can consider only adversaries that always return $\hat{c}\neq c$. More formally, it is straightforward to modify any adversary to output a different encryption if $\hat{c}=c$. This would only increase the adversaries success probability with the cost of constant overhead in running time. 

Now consider a small change in the game, where instead of getting an encryption of $(k || r)$, the adversary gets an encryption of a random message $(\bar{k} || \bar{r})$. To emphasise the change, let us incorporate the generation of this encryption into the original game without no further use of it. This leads to the game pair 
\begin{align*}
&\begin{game}{\GAME_0}
 &(\PK,\SK)\gets\GEN\\
 &k\getsu\KSPACE, r\getsu\RRR\\
 &\bar{k}\getsu\KSPACE, \bar{r}\getsu\RRR\\
 &c\gets\ENC_\PK(k||r)\\
 &\overline{c}\gets\ENC_\PK(\bar{k}||\bar{r})\\
 &\hat{c}\gets\AD(c)\\
 &\hat{k}||\hat{r}\gets\DEC_\SK(\hat{c})\\
 &\IF r\neq \hat{r}\ \RETURN 0\\
 &\RETURN \neg[k\iseq \hat{k}]\enspace
\end{game}
&
&\begin{game}{\GAME_1}
 &(\PK,\SK)\gets\GEN\\
 &k\getsu\KSPACE, r\getsu\RRR\\
 &\bar{k}\getsu\KSPACE, \bar{r}\getsu\RRR\\
 &c\gets\ENC_\PK(k||r)\\
 &\overline{c}\gets\ENC_\PK(\bar{k}||\bar{r})\\
 &\hat{c}\gets\AD(c_1)\\
 &\hat{k}||\hat{r}\gets\DEC_\SK(\hat{c})\\
 &\IF r\neq \hat{r}\ \RETURN 0\\
 &\RETURN \neg[k\iseq \hat{k}]\enspace.
\end{game}
\end{align*}
Note that it is straightforward to rewrite the end-game correctness check as a relation between plaintexts
\begin{align}
\label{eq:target-relation}
\pi(k||r, \hat{k}||\hat{r})= [r \iseq \hat{r}] \wedge [k \neq \hat{k}]\enspace.
\end{align}
Hence, we can directly reduce these games to the non-malleability games
\begin{align*}
&\begin{game}{\BGAME_0}
 &(\PK,\SK)\gets\GEN\\
 &\MSPACE_0\gets\ADB(\PK)\\
 &m\gets\MSPACE_0, \overline{m}\gets\MSPACE_0\\ 
 &c\gets\ENC_\PK(m)\\
 &\overline{c}\gets\ENC_\PK(\overline{m})\\
 &\pi,\hat{c}_1,\ldots,\hat{c}_n\gets\ADB(c)\\
 &\IF c\in\set{\hat{c}_1,\ldots,\hat{c}_n}\ \THEN\RETURN 0\\
 &\hat{m}_i\gets\DEC_\SK(\hat{m}_i)\text{ for }i\in\set{1,\ldots,n}\\
 &\RETURN \pi(m,\hat{m}_1,\ldots,\hat{m}_n)
\end{game}
&
&\begin{game}{\BGAME_1}
 &(\PK,\SK)\gets\GEN\\
 &\MSPACE_0\gets\ADB(\PK)\\
 &m\gets\MSPACE_0, \overline{m}\gets\MSPACE_0\\ 
 &c\gets\ENC_\PK(m)\\
 &\overline{c}\gets\ENC_\PK(\overline{m})\\
 &\pi,\hat{c}_1,\ldots,\hat{c}_n\gets\ADB(\overline{c})\\
 &\IF c\in\set{\hat{c}_1,\ldots,\hat{c}_n}\ \THEN\RETURN 0\\
 &\hat{m}_i\gets\DEC_\SK(\hat{m}_i)\text{ for }i\in\set{1,\ldots,n}\\
 &\RETURN \pi(m,\hat{m}_1,\ldots,\hat{m}_n)
\end{game}
\end{align*}
The corresponding reduction is following. First, $\ADB(\PK)$ returns an algorithm $\MSPACE_0$ that return a uniformly chosen element form $\KSPACE\times\RSPACE$. Second, $\ADB(c)$ runs $\AD(c)$ and returns the description of the relation given as the equation~\eqref{eq:target-relation}. Since $\AD$ is guaranteed never to return $\hat{c}=c$, the direct substitutions gives $\BGAME_0^\ADB\equiv \GAME_0^\AD$ and $\BGAME_1^\ADB\equiv \GAME_1^\AD$. The latter implies  
\begin{align*}
\abs{\pr{\GAME_0^\AD=1}-\pr{\GAME_1^\AD=1}}\leq\ADVNMCPA{\CS}{\ADB}\enspace.
\end{align*}

Let us now concentrate on the simplification of the game $\GAME_1$. Since the adversaries reply does not depend on $c$, $r$ and $k$, we can simplify the game by delaying these random choices:
\begin{align*}
&\begin{game}{\GAME_2}
 &(\PK,\SK)\gets\GEN\\
 &\bar{k}\getsu\KSPACE, \bar{r}\getsu\RRR\\
 &\overline{c}\gets\ENC_\PK(\bar{k}||\bar{r})\\
 &\hat{c}\gets\AD(c_1)\\
 &\hat{k}||\hat{r}\gets\DEC_\SK(\hat{c})\\
 &k\getsu\KSPACE, r\getsu\RRR\\
 &\IF r\neq \hat{r}\ \RETURN 0\\
 &\RETURN \neg[k\iseq \hat{k}]\enspace.
\end{game}
\end{align*}
Since $r$ is chosen uniformly after $\hat{r}$ is fixed, we get 
\begin{align*}
\pr{\GAME_1^\AD=1}=\pr{\GAME_2^\AD=1}\leq \frac{1}{\abs{\RSPACE}}
\end{align*}      
where the last inequality follows from the fact that we must pass the $\hat{r}\iseq r$ test in order to end the game with one. Consequently, we have proven that for $t$-time adversaries $\AD$, the success is bounded by
\begin{align*}
\pr{\GAME_0^\AD=1}\leq\frac{1}{\abs{\RSPACE}}+\varepsilon\enspace
\end{align*} 
provided that the cryptosystem is $(t+\Oh(1),\varepsilon)$-NM-CPA secure. 

\end{solution}

\end{document}