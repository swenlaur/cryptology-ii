\documentclass{crypto-exercise}
\usepackage{amsthm}
\author{Sven Laur}
\contributor[Initial solution]{Ivo Kubjas}
\editor{Sven Laur}
\tags{layered commitment, binding, partial opening, list commitment}

\newcommand{\CS}{\mathfrak{C}}
\newcommand{\ADCOM}{\AD_\textsf{Com}}
\newcommand{\ADCR}{\AD_\textsf{CR}}
\newcommand{\COLL}{\mathsf{Collision}}
\newcommand{\ADVBIND}[2]{\ADV_{#1}^{\mathsf{bind}}(#2)}
\newcommand{\ADVCR}[2]{\ADV_{#1}^{\mathsf{cr}}(#2)}


\begin{document}
\begin{exercise}{Security against partial double-opening}
Let $\CS=(\GEN,\COM,\OPEN)$ be commitment scheme and $\HHH$ be a collision resistant hash function family with an appropriate domain. Then we can build a list commitment scheme on top of the ordinary commitment scheme:
    \begin{align*}
      \begin{fblock}{\GEN^\star}
        &\PK\gets\GEN\\
        &h\getsu\HHH\\
        &\RETURN (\PK,h)
      \end{fblock}
      \qquad
      \begin{fblock}{\COM_{\PK,h}^\star(x_1,\ldots,x_{\ell})}
        &(c_i,d_i)\gets \COM_{\PK}(x_i),\ i\in\set{1,\ldots,\ell}\\ 
        &c_*\gets h(c_1,\ldots,c_{\ell})\\
        &\RETURN (c_*,(c_1,\ldots,c_{\ell},d_1,\ldots,d_{\ell})) 
      \end{fblock}
    \end{align*}
where the decommitment procedure just verifies $c_*=h(c_1,\ldots,c_\ell)$ and restores
$x_i\gets\OPEN_\PK(c_i,d_i)$ for $i\in\set{1,\ldots,\ell}$. Prove that the commitment scheme is secure against partial double openings defined through the following security game
\begin{align*}
\begin{game}{\GAME}
& (\PK,h)\gets\GEN^\star\\
& (c_*, c_1,\ldots c_\ell, \hat{c}_1,\ldots \hat{c}_\ell)\gets\AD(\PK,h)\\
& (i,d_i,\hat{d}_i)\gets\AD(\PK,h)\\
& \IF c_*\neq h(c_1,\ldots c_\ell) \vee  c_*\neq h(\hat{c}_1,\ldots \hat{c}_\ell)\ \THEN
   \RETURN 0\\
& \RETURN \bot\neq\OPEN_\PK(c_i,d_i)\neq \OPEN_\PK(\hat{c}_i,\hat{d}_i)\neq \bot\enspace  
\end{game}
\end{align*}
provided that the base commitment is $(t,\varepsilon_1)$-binding and the hash function family is $(t,\varepsilon_2)$-collision resistant.
\end{exercise}
\begin{solution}
Intuitively, there are two possible ways how the adversary $\AD$ can breach the security. First, the adversary $\AD$ may find a double opening for the base commitment scheme $\CS$. Second, the  adversary $\AD$ can breaking collision resistant hash function $h \in \HHH$. 

Given the output $(c_*, c_1,\ldots c_\ell, \hat{c}_1,\ldots \hat{c}_\ell)$ is straightforward to decide whether the adversary found a hash collision or not. Namely, the collision occurs if $h(c_1,\ldots c_\ell)=h(\hat{c}_1,\ldots \hat{c}_\ell)$ and there exists $c_i\neq\hat{c}_i$. Thus, we can convert the original adversary $\AD$ into two adversaries $\AD_1$ and $\AD_2$. The adversary $\AD_1$ runs internally $\AD$ and outputs $(c_*, c_1,\ldots c_\ell, \hat{c}_1,\ldots \hat{c}_\ell)$ only if the event $\COLL$ does not occur, otherwise it halts. The adversary $\AD_2$ also runs internally $\AD$ but continues only if the event $\COLL$ occurs.  By the construction it is straightforward to note that 
\begin{align*}
\pr{\GAME^\AD=1}=\pr{\GAME^{\AD_1}=1}+\pr{\GAME^{\AD_2}=1}\enspace
\end{align*}   
and thus it is sufficient if we analyse the success of both adversaries separately. 

Note that $\AD_1$ can succeed only if $\AD$ double opens some commitment value $c_i$, since it always outputs $c_i = \hat{c}_i$ for all $i \in \{1, \ldots, \ell\}$. More formally, let 
\begin{align*}
        \begin{game}{\BGAME^\ADB}
            & \PK \gets \GEN \\
            & (c, d, \hat{d}) \gets \ADB(\PK) \\
            & \RETURN \bot \neq \OPEN_\PK(c, d) \neq \OPEN_\PK(c, \hat{d}) \neq
            \bot \enspace
        \end{game}
\end{align*}
be the binding game. Then we can use the following a reduction construction  
\begin{align*}
        \begin{fblock}{\ADB(\PK)}
            & h \getsu \HHH \\
            & (c_*, c_1,\ldots c_\ell, \hat{c}_1,\ldots \hat{c}_\ell) \gets \AD_1(\PK, h) \\
            & (i, d_i, \hat{d}_i) \gets \AD_1(\PK, h) \\
            & \RETURN (c_i, d_i, \hat{d}_i) \enspace.
        \end{fblock}
\end{align*}
By inlining the definition of $\ADB$ into the game $\BGAME$ we obtain a slightly modified game
\begin{align*}
        \begin{game}{\GAME_1}
            & \PK,\gets \GEN, h\getsu \HHH\\ 
            & (c_*, c_1,\ldots c_\ell, \hat{c}_1,\ldots \hat{c}_\ell) \gets \AD_1(\PK, h) \\
            & (i, d_i, \hat{d}_i) \gets \ADCOM(\PK, h) \\
            & \RETURN \bot \neq \OPEN_\PK(c_i, d_i) \neq \OPEN_\PK(c_i,
            \hat{d}_i) \neq
            \bot \enspace
        \end{game}
\end{align*}
which is more liberal compared to the original security game $\GAME$ due to omitted tests. As a result, we get
\begin{align*}
\pr{\GAME^{\AD_1}=1}\leq\pr{\smash{\GAME_1^{\AD_1}=1}}=\pr{\BGAME^\ADB=1}=\ADVBIND{\CS}{\ADB}\enspace.
\end{align*}
Now note that the time needed to check whether the collision exists or not is $\Theta(\ell)$ and thus the running time of $\AD_1$ and $\ADB$ is only $\Theta(\ell)$ bigger that the running time for $\AD$. Hence for $(t-\Oh(\ell))$-time adversaries $\AD$, we can conclude that $\pr{\GAME^{\AD_1}=1}\leq\varepsilon_1$ if the commitment is $(t,\varepsilon_1)$-binding.
 

By the construction, $\AD_2$ can succeed only if $\AD_1$ finds a hash collision and thus its success is bounded by $\varepsilon_2$. Formally, we must still prove it by providing an explicit reduction to the collision-resistance game 
\begin{align*}
        \begin{game}{\QQQ'}
            & h \getsu \HHH \\
            & (m_1, m_2) \gets \ADB(h) \\
            & \RETURN m_1 \neq m_2 \land h(m_1) = h(m_2)\enspace.
        \end{game}
\end{align*}
The reduction is trivial
\begin{align*}
        \begin{fblock}{\ADB(h)}
            & \PK \gets \GEN \\
            & (c_*, c_1,\ldots c_\ell, \hat{c}_1,\ldots \hat{c}_\ell) \gets \AD_2(\PK, h) \\
            & m_1\gets (c_1,\ldots, c_\ell)\\
            & m_2\gets (\hat{c}_1,\ldots, \hat{c}_\ell)\\
            & \RETURN (m_1,m_2) \enspace.
        \end{fblock}
\end{align*}
By inlining this adversary definition in to the game $\BGAME$, we obtain a more liberal game 
\begin{align*}
        \begin{game}{\GAME_2}
            & \PK,\gets \GEN, h\getsu \HHH\\ 
            & (c_1, \ldots, c_\ell, \hat{c}_1, \ldots, \hat{c}_\ell) \gets
            \AD_2(\PK, h) \\
            & \RETURN h(c_1, \ldots, c_\ell) = h(\hat{c}_1, \ldots,
            \hat{c}_\ell)\wedge (c_1, \ldots, c_\ell) \neq (\hat{c}_1, \ldots,
            \hat{c}_\ell)
        \end{game}
\end{align*} 
compared to the game $\GAME$. Thus, we arrive at 
\begin{align*}
\pr{\GAME^{\AD_2}=1}\leq\pr{\smash{\GAME_2^{\AD_2}=1}}=\pr{\BGAME^\ADB=1}=\ADVCR{\HHH}{\ADB}\enspace.
\end{align*}
Again, the overhead in the running-time of $\ADB$ is $\Oh(\ell)$ and thus for all $(t-\Oh(\ell))$-time adversaries $\AD$, we can conclude that $\pr{\GAME^{\AD_2}=1}\leq\varepsilon_2$ if the hash function family is $(t,\varepsilon_2)$-collision resistant.  
\end{solution}

\end{document}
