\documentclass{crypto-exercise}
\usepackage{amsthm}
\author[Fiat-Shamir]{Sven Laur}
\contributor{Pille Pullonen}
\editor{Sven Laur}
\tags{statistical indistinguishability, distributions, simulatability, quadratic non-residuosity proof}

\renewcommand{\ADVIND}[2]{\ADV^{\mathsf{ind}}_{#1}(#2)}
\newcommand{\REWRITE}{\qquad\scalebox{1.5}{$\rightsquigarrow$}\qquad}

\newcommand{\ISQNR}{\mathsf{IsQnr}}
\begin{document}


\begin{exercise}{Simulator for the QNR-ZKD protocol}
Let $n$ be a composite number with a factorisation $n=pq$ known to the prover $\PPP$. Let $v\in\ZZ_n^*$ be a number for which the prover wants to prove that it is quadratic non-residue. Let $(\GEN,\COM,\OPEN)$ be a perfectly binding and computationally hiding commitment.  Show that the the following zero-knowledge  protocol    
\begin{center}
\begin{tabular}{lllcl}
	&$\PPP$ &&   $\VVV$\\
	$\PK\gets\GEN$
	&&&& $b\getsu\set{0,1},\ $\\
    &&$\xleftarrow{\makebox[4cm]{$c=r^2v^b$}}$ 
	&& $r\getsu\ZZ_n^*$ \\
	
	$\overline{b}\gets\ISQNR_{p,q}(c)$\\
	$(\overline{c},\overline{d})\gets\COM(\overline{b})$
	&&$\xrightarrow{\makebox[4cm]{{$\PK,\overline{c}$}}}$\\
	&&$\xleftarrow {\makebox[4cm]{$r,b$}}$ \\
	$c\iseq r^2v^b$\\
	&&$\xrightarrow{\makebox[4cm]{$\overline{d}$}}$
	&& $\OPEN_\PK(\overline{c},\overline{d})\iseq b$
	\\
\end{tabular}
\end{center}  
where the verifier $\VVV$ releases $r,b$ is simulatable. For that construct first a semi-efficient algorithm $\KEXTR^{\VVV_*}$ for extracting $r,b$ that correspond to initial message $c$. Next show that the following simulator construction
\begin{align*}
\begin{fblock}{\VVV_\circ(\phi)}
 & \omega\gets\Omega\\
 & c\gets\VVV_*(\phi;\omega)\\
 & (b_*,r_*)\gets\KEXTR^{\VVV_*}(\phi;\omega)\\
 & \alpha\gets\VVV_*\\
 & \beta\getsu\BBB\\
 & \gamma\gets\VVV_*(\beta)\\
 & \IF \VERIFY(\alpha,\beta,\gamma)=0\ \THEN \RETURN \VVV_*(\bot)\\ 
 & \IF \beta_*\neq \bot \ \THEN \RETURN \VVV_*(b_*)\\
 & \ELSE \RETURN \bot
\end{fblock}
\end{align*}   
can create an output distribution $\psi_\circ$ that is computationally  $(t_\circ,\varepsilon_\circ)$-distant from the output distribution of malicious verifier $\VVV_*$ that interacts with the honest prover. Also, estimate how the running-time of the simulator depends on the desired distance $\varepsilon_\circ$.
\end{exercise}
\begin{solution}
\end{solution}
\end{document}

