\documentclass{crypto-exercise}
\usepackage{amsthm}
\author{Sven Laur}
\tags{indistinguishability, hard-core bit, regularity}


\begin{document}
\begin{exercise}{Hard-core bits and regularity}
A predicate $\pi:\SSS\to\set{0,1}$ is said to be a
  \emph{$\varepsilon$-regular} if the output distribution for uniform
  input distribution is nearly uniform:
  \begin{align*}
   \Delta(\pi)=\abs{\pr{s\getsu\SSS:\pi(s)=0}-\pr{s\getsu\SSS:\pi(s)=1}}
   \leq\varepsilon\enspace.     
  \end{align*}
  A predicate $\pi$ is a \emph{$(t,\varepsilon)$-unpredictable} also
  known as \emph{$(t,\varepsilon)$-hardcore predicate} for a function
  $f:\SSS\to\XXX$ if for any $t$-time adversary
  \begin{align*}
    \advHCPredXX{f,\pi}{\AD}%
    =2\cdot\abs{\pr{s\getsu\SSS:\AD(f(s))=\pi(s)}-{\textstyle\frac{1}{2}}}\leq\varepsilon\enspace.
  \end{align*}
Let us first define two sets:
\begin{align*}
\SSS_0&=\set{s\in\SSS: \pi(s)=0}\\
\SSS_1&=\set{s\in\SSS: \pi(s)=1}\enspace.
\end{align*}    
Then we can define following distinguishing games:
\begin{align*}
&\begin{game}{\GAME_0}
 & s\getsu \SSS_0\\
 & x\gets f(s)\\
 & \RETURN \AD(x)
\end{game}
&&\begin{game}{\GAME_1}
 & s\getsu \SSS_1\\
 & x\gets f(s)\\
 & \RETURN \AD(x)
\end{game}
\end{align*} 
Show that even if $\SSS_0$ and $\SSS_1$ are completely indistinguishable, the predicate does not have to be $(t,\varepsilon)$-unpredictable if the predicate $\pi$ is not regular.  
\end{exercise}
\begin{solution}
\end{solution}
\end{document}