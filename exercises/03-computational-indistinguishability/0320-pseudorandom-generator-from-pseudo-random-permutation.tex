\documentclass{crypto-exercise}
\usepackage{amsthm}
\usepackage{float}
\author{Sven Laur}
\contributor{}
\editor{Sven Laur}
\tags{indistinguishability, pseudorandom generator, pseudorandom permutation}


\begin{document}
\begin{exercise}{PRG from PRP}
Let $\FFF$ be a $(q,t,\varepsilon)$-secure pseudorandom permutation family defined by a deterministic function $f:\KKK\times\MMM\to\MMM$ such that all functions $f_k(m):=f(k,m)$ are different. Show that functions $g_m:\KKK\to\MMM^n$ defined through the following iteration algorithm
\begin{align*}
\begin{fblock}{g_m(k)}
& c_1\gets f(k, m)\\
& c_2\gets f(k, c_1)\\
&\cdots\\
& c_n\gets f(k, c_{n-1})\\ 
&\RETURN c_1,c_2,\ldots,c_n\\
\end{fblock}
\end{align*} 
are pseudorandom generators for any $m\in\MMM$ for small enough $n$.
\end{exercise}
\begin{solution}\ \\
\textsc{Subproof}. Let us prove the claim under the assumption that we can replace all function invocations by random samplings from  $\MMM$.\\\\
\textsc{Subproof}. Define the collision event and analyse what is the probability that such event occurs under the assumption that function family is the set of all functions
$\FFF_{\textsc{all}}(\MMM\to\MMM)$. Conclude that the construction is pseudorandom generator under this assumption.\\\\ 
\textsc{Subproof}. Use PRP/PRF switchng lemma to complete the proof
\end{solution}
\end{document}