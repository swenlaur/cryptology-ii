\documentclass{crypto-exercise}
\usepackage{amsthm}
\usepackage{float}
\author{Sven Laur}
\contributor{}
\editor{Sven Laur}
\tags{indistinguishability, hard-core predicate}


\begin{document}
\begin{exercise}{Hard-core predicate based on indistinguishability}
A predicate $\pi$ is a \emph{$(t,\varepsilon)$-unpredictable} also known as \emph{$(t,\varepsilon)$-hardcore predicate} for a function $f:\SSS\to\XXX$ if for any $t$-time adversary
  \begin{align*}
    \advHCPredXX{f}{\AD}%
    =2\cdot\abs{\pr{s\getsu\SSS:\AD(f(s))=\pi(s)}-{\textstyle\frac{1}{2}}}\leq\varepsilon\enspace.
  \end{align*}
Show that $\pi:\SSS\to\set{0,1}$ must be almost regular if $\pi$ is hard-core predicate. 
Let $\XXX_i$ denote the distribution of $f(s)$ for $s\getsu\SSS_i$ where $\SSS_0=\set{s\in\SSS:\pi(s)=0}$ and $\SSS_1=\set{s\in\SSS:\pi(s)=1}$. 
Show that if the distributions $\XXX_0$ and $\XXX_1$ are $(t,\varepsilon)$-indistinguishable then $\pi$ is also a hardcore predicate. 
Analyse how the prediction advantage depends on regularity. 
Is it possible to prove the reverse implication? 
\end{exercise}
\begin{solution}\ \\
\textbf{Hint}: Give alternative definition of hard-core bits in terms of two games $\BGAME_0$ and $\BGAME_1$.\\
\textbf{Hint}: Define $\ADB$ such that $\BGAME_0^\ADB\equiv \GAME_0^\AD$. What is the corresponding $\BGAME_1^\ADB$?\\
 


\end{solution}
\end{document}