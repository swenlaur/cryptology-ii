\documentclass{crypto-exercise}
\usepackage{amsthm}
\author{Sven Laur}
\contributor[Initial solution]{Ivo Kubjas}
\contributor[Better treatment of corner cases]{Sven Laur}
\editor{Sven Laur}
\tags{computational indistinguishability, distributions, sampling}

\begin{document}


\begin{exercise}{Lottery game with distributions}
Consider the following game, where an adversary $\AD$ gets three
values $x_1$, $x_2$ and $x_3$. Two of them are sampled from the
efficiently samplable distribution $\XXX_0$ and one of them is sampled
from the efficiently samplable distribution $\XXX_1$. The adversary
wins the game if it correctly determines which sample is taken from
$\XXX_1$. Find an upper bound to the success probability if
distributions $\XXX_0$ and $\XXX_1$ are
$(t,\varepsilon)$-indistinguishable.
\end{exercise}

\begin{solution}
Any such problem can be split into three conceptual parts:
formalisation of the attack scenario, game manipulation, and final
probability computations. One possible formalisation of the attack scenario is given below 
\begin{align*}
  \begin{game}{\GAME_0^\AD}
    &x_1\gets\XXX_0\\
    &x_2\gets\XXX_0\\
    &x_3\gets\XXX_1\\
    &\pi\getsu\mathsf{Perm}(\set{1,2,3})\\
    &i\gets\AD(x_{\pi(1)},x_{\pi(2)},x_{\pi(3)})\\
    &\RETURN [\pi(i)\iseq 3]
  \end{game}
\end{align*}
The fourth line in the game models random shuffling of the values
$x_1,x_2,x_3$. If we choose uniformly a permutation $\pi$ over
$\set{1,2,3}$, the elements $x_{\pi(1)},x_{\pi(2)},x_{\pi(3)}$ are in
a random order. Obviously, the guess of $\AD$ is correct if and only
if $\pi(i)=3$.  As a second step, we modify the game in the following
way
\begin{align*}
  \begin{game}{\GAME_0^\AD}
    &x_1\gets\XXX_0\\
    &x_2\gets\XXX_0\\
    &x_3\gets\XXX_1\\
    &\pi\getsu\mathsf{Perm}(\set{1,2,3})\\
    &i\gets\AD(x_{\pi(1)},x_{\pi(2)},x_{\pi(3)})\\
    &\RETURN [\pi(i)\iseq 3]
  \end{game}
  \qquad\Redarrow{IND}{}\qquad
   \begin{game}{\GAME_1^\AD}
    &x_1\gets\XXX_0\\
    &x_2\gets\XXX_0\\
    &x_3\gets\XXX_0\\
    &\pi\getsu\mathsf{Perm}(\set{1,2,3})\\
    &i\gets\AD(x_{\pi(1)},x_{\pi(2)},x_{\pi(3)})\\
    &\RETURN [\pi(i)\iseq 3]
  \end{game}
\end{align*}
Note that the games differ only in a single line: $x_3$ is chosen
either from $\XXX_0$ or from $\XXX_1$ depending on the game. The
latter allows us to use the entire game as a distinguisher for
$\XXX_0$ and $\XXX_1$. Namely, let us define a new adversary
\begin{align*}
  \begin{fblock}{\ADB(x)}
    &x_1\gets\XXX_0\\
    &x_2\gets\XXX_0\\
    &x_3\gets x\\
    &\pi\getsu\mathsf{Perm}(\set{1,2,3})\\
    &i\gets\AD(x_{\pi(1)},x_{\pi(2)},x_{\pi(3)})\\
    &\RETURN [\pi(i)\iseq 3]
  \end{fblock}
\end{align*}
against the indistinguishability games
\begin{align*}
  \begin{game}{\BGAME_0^\ADB}
    &x\gets\XXX_0\\
    &\RETURN \ADB(x)
  \end{game}
  \qquad\qquad
    \begin{game}{\BGAME_1^\ADB}
    &x\gets\XXX_1\\
    &\RETURN \ADB(x)
  \end{game}
\end{align*}
By the $(t,\varepsilon)$-indistinguishability assumptions 
\begin{align*}
  \advINDXX{\XXX_0,\XXX_1}{\ADB}=\abs{\pr{\BGAME_0^\ADB=1}-\pr{\BGAME_1^\ADB=1}}\leq \varepsilon
\end{align*}
for any $t$-time adversary $\ADB$. Let us estimate the behaviour of
our concrete adversary by inserting its definition into the games
$\BGAME_0$ and $\BGAME_1$:
\begin{align*}
  \begin{game}{\BGAME_0^\ADB}
    &x\gets \XXX_0\\
    &x_1\gets\XXX_0\\
    &x_2\gets\XXX_0\\
    &x_3\gets x\\
    &\pi\getsu\mathsf{Perm}(\set{1,2,3})\\
    &i\gets\AD(x_{\pi(1)},x_{\pi(2)},x_{\pi(3)})\\
    &\RETURN [\pi(i)\iseq 3]
  \end{game}
  \qquad\qquad
   \begin{game}{\BGAME_1^\ADB}
    &x\gets \XXX_1\\
    &x_1\gets\XXX_0\\
    &x_2\gets\XXX_0\\
    &x_3\gets x\\
    &\pi\getsu\mathsf{Perm}(\set{1,2,3})\\
    &i\gets\AD(x_{\pi(1)},x_{\pi(2)},x_{\pi(3)})\\
    &\RETURN [\pi(i)\iseq 3]
  \end{game}
\end{align*}
By doing simple syntactic changes that do not alter the behaviour of
games, we can convert $\BGAME_0^\ADB$ to $\GAME_1^\AD$ and
$\BGAME_1^\ADB$ to $\GAME_0^\AD$. Hence, we have established that
\begin{align*}
  \abs{\pr{\GAME_0^\AD=1}-\pr{\GAME_1^\AD=1}}
  =\abs{\pr{\BGAME_1^\ADB=1}-\pr{\GAME_0^\ADB=1}}\leq \varepsilon
\end{align*}
provided that the running-time of $\ADB$ is less than $t$. Let
$t_{\AD}$ be the running-time of $\AD$ and $t_{\mathrm{s}}$ time
needed to get a sample from $\XXX_0$ or $\XXX_1$. Then the running
time of $\ADB$ is $2t_{\mathrm{s}}+t_{\AD}+\Oh(1)$. Hence, for all
$t-2t_{\mathrm{s}}-\Oh(1)$ time adversaries
\begin{align}\label{eq:i}
  \abs{\pr{\GAME_0^\AD=1}-\pr{\GAME_1^\AD=1}}\leq \varepsilon\enspace.
\end{align}
By doing syntactic changes that do not alter the behaviour of the
game, we can rewrite the game $\GAME_1$ even further
\begin{align*}
   \begin{game}{\GAME_1^\AD}
    &x_1\gets\XXX_0\\
    &x_2\gets\XXX_0\\
    &x_3\gets\XXX_0\\
    &\pi\getsu\mathsf{Perm}(\set{1,2,3})\\
    &i\gets\AD(x_{\pi(1)},x_{\pi(2)},x_{\pi(3)})\\
    &\RETURN [\pi(i)\iseq 3]
  \end{game}
  \qquad\Redarrow{Syntax}{}\qquad
   \begin{game}{\GAME_2^\AD}
    &x_1\gets\XXX_0\\
    &x_2\gets\XXX_0\\
    &x_3\gets\XXX_0\\
    &i\gets\AD(x_1,x_2,x_3)\\
    &\pi\getsu\mathsf{Perm}(\set{1,2,3})\\
    &\RETURN [\pi(i)\iseq 3]
  \end{game}
\end{align*}
Note that the behaviour of the game does not change since $\AD$ gets
the same input distribution $\XXX_0\times\XXX_0\times\XXX_0$ in both
games. As the output of $\AD$ is fixed before the permutation is
chosen, we get
\begin{align}\label{eq:ii}
  \pr{\GAME_2^\AD=1}=\frac{1}{3}\enspace.
\end{align}
By combing~\eqref{eq:i}~and~\eqref{eq:ii} we obtain
\begin{align*}
  \pr{\GAME_0^\AD=1}\leq \frac{1}{3}+\varepsilon
\end{align*}
provided that the running-time of $\AD$ is $t-2t_{\mathrm{s}}-\Oh(1)$.


\paragraph{Comments.}
if distributions $\XXX_0$ and $\XXX_1$ are
$(t,\varepsilon)$-indistinguishable, it is always possible to change
the game by replacing a line $x\gets\XXX_0$ with a line
$x\gets\XXX_1$. The total time-complexity of the game sets limitations
on the overall running time of the adversary, as the corresponding
distinguisher $\ADB$ must simulate the game inside its code. By
applying such rewriting rules long enough, we can prove computational
closeness of many complex games.
\end{solution}

\end{document}
