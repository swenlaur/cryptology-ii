\documentclass{crypto-exercise}
\author{Sven Laur}
\contributor{Sven Laur}
\editor{Sven Laur}
\tags{computational distance}

\DeclareMathOperator*{\EXPO} {\mathbf{E}}
\begin{document}



\begin{exercise}{Explicit estimates of computational distances}
Normally, it is impossible to compute computational distance between
two distributions directly, since the number of potential
distinguishing algorithms is humongous. However, for really small
time-bounds it can be done. Assume that all distinguishers
$\AD:\ZZ_{16}\to\set{0,1}$ are implemented as Boolean circuits
consisting of $\textsc{Not}$, $\textsc{And}$, $\textsc{Or}$ gates and
the corresponding time-complexity is just the number of logic
gates. For example, $\AD(x_3x_2x_1x_0)= x_1$ has time-complexity $0$
and $\AD(x_3x_2x_1x_0)= x_1\vee \neg x_3\wedge x_2$ has
time-complexity $3$.
\begin{enumerate}
   \item Let $\mathcal{X}_0$ be a uniform distribution over $\ZZ_{16}$
     and let $\mathcal{X}_1$ be a uniform distribution over
     $\set{0,2,4,6,8,10,12,14}$. What is $\CD_x^1(\XXX_0,\XXX_1)$?
   \item Find a uniform distribution $\XXX_2$ over some $8$ element
     set such that $\CD_x^1(\XXX_0,\XXX_2)$ is minimal. Compute
     $\CD_x^2(\XXX_0,\XXX_2)$ and $\CD_x^3(\XXX_0,\XXX_2)$.
   \item Find a uniform distribution $\XXX_3$ over some $8$ element
     set such that the distance sum $\CD_x^1(\XXX_1,\XXX_0)+
     \CD_x^1(\XXX_0,\XXX_3)\neq\CD_x^1(\XXX_1,\XXX_3)$.
   \item Estimate for which value of $t$ the distances
     $\CD_x^t(\XXX_0,\XXX_1)$ and $\SD_x(\XXX_0,\XXX_1)$ coincide for
     all distributions over $\ZZ_{16}$.
\end{enumerate}
\end{exercise}

\begin{solution}
As the statistical distance $\SD_x(\XXX_0,\XXX_1)=\frac{1}{2}$ and the
corresponding distinguisher $A(x_3x_2,x_1x_0)=x_0$ consists of zero
gates, we get $\CD_x^0(\XXX_0,\XXX_1)=\frac{1}{2}$. For the second
question, let $\XXX_\phi=\set{x\in\ZZ_{16}: \phi(x)=1}$ denote the
true-set for a circuit $\phi$ and let $\XXX_2$ be some $8$ element
set. Then by definition
\begin{align*}
  \advINDXX{\XXX_0,\XXX_2}{\phi}%
  &=\abs{\pr{x\gets\XXX_0: \phi(x)=1}-\pr{x\gets\XXX_2: \phi(x)=1}}\\
  &=\frac{1}{16}\cdot\abs{\abs{\XXX_\phi}-2\cdot\abs{\XXX_\phi\cap \XXX_2}}
  =\frac{1}{16}\cdot\abs{\abs{\XXX_\phi}-\abs{\XXX_\phi\setminus\XXX_2}}
\end{align*}
and minimal computational distance is achieved by the set $\XXX_2$
that splits almost evenly by all possible sets $\XXX_\phi$. By
considering formulae
\begin{align*}
 \phi_1(x)=x_0,\ldots, \phi_4(x)=x_3,
\phi_5(x)=\neg x_0,\ldots,\phi_8(x)=\neg x_3\enspace,  
\end{align*}
we get that a set $\XXX_2$ can achieve $\CD_x^1(\XXX_0,\XXX_2)=0$ only
if it contains $4$ elements with the $i$th bit set to one and $4$
elements with the $i$th bit set to zero. Formulae
\begin{align*}
  \phi_9(x)&=x_0\wedge x_1,\enspace
  \phi_{10}(x)=x_0\wedge x_2 \ldots,
  \phi_{13}(x)=x_1\wedge x_3,\enspace
  \phi_{14}(x)=x_2\wedge x_3\enspace,\\
  \phi_{15}(x)&=x_0\vee x_1,\enspace
  \phi_{16}(x)=x_0\vee x_2 \ldots,
  \phi_{19}(x)=x_1\vee x_3,\enspace
  \phi_{20}(x)=x_2\vee x_3\enspace
\end{align*}
indicate that such a set must contain exactly $2$ elements with $i$th
and $j$th bit set to one and exactly $2$ elements with $i$th and $j$th
bit set to zero.  A bit representation of a possible solution is
depicted in Figure~\ref{fig:orthogonal-array}. The solution has a
peculiar property: if we choose uniformly element from $\XXX_2$ and
observe a bit pair $i$ and $j$ the corresponding bit-string has
uniform distribution over $\ZZ_4$. Consequently, any formula
consisting of two inputs is incapable from distinguishing $\XXX_0$ and
$\XXX_2$. A formula consisting of two gates can cover three inputs and
thus potential distinguishing capabilities are higher.  As
Figure~\ref{fig:violating-triples} clearly shows, the distribution of
bit triples $x_0,x_2,x_3$ is indeed different from uniform and the
task of building a distinguisher simplifies considerably. In fact, we can express
\begin{align*}
  \advINDXX{\XXX_0,\XXX_2}{\phi}
  =\frac{1}{8}\cdot\abs{\psi(000)+\psi(101)+\psi(110)
  -\psi(001)-\psi(100)-\psi(111)}\enspace.
\end{align*}
for any formula $\phi(x)=\psi(x_0x_2x_3)$. Exhaustive search reveals
that the formulae
\begin{align*}
    x_0\wedge x_2 \wedge x_3\enspace,
    x_0\vee x_2 \vee x_3\enspace,
    x_0\wedge x_3 \vee x_2,\enspace,
    x_0\wedge(x_2\vee x_3)\enspace   
\end{align*}
all achieve optimal advantage
$\advINDXX{\XXX_0,\XXX_2}{\phi}=\frac{1}{8}$. For the next distance
estimate, note that a three gate distinguisher can cover all $4$
inputs if it does not contain $\textsc{Not}$-gates. All of such
distinguishers achieve advantage $\frac{1}{16}$ and thus cannot not be
optimal. Consequently, a potential optimal $3$-gate distinguisher with
$\textsc{Not}$-gate must process inputs $x_0,x_2,x_3$. Indeed, several
formulae with negation achieve again the advantage $\frac{1}{8}$ but
not more. Hence, we have shown that
\begin{align*}
 \CD_x^2(\XXX_0,\XXX_2)=\CD_x^3(\XXX_0,\XXX_2)=\frac{1}{8}\enspace.
\end{align*}


\begin{figure}[!t]
  \centering
  \begin{small}
  \begin{tabular}{|c|c|c|c|c|c|c|c|}
  \hline
  $x_1$ & $x_2$ & $x_3$ & $x_4$ & $x_5$ & $x_6$ & $x_7$ & $x_8$ \\
  \hline
  $1$   & $1$  & $1$   & $1$   & $0$   & $0$  & $0$   & $0$   \\
  $1$   & $1$  & $0$   & $0$   & $1$   & $1$  & $0$   & $0$   \\
  $1$   & $0$  & $1$   & $0$   & $1$   & $0$  & $1$   & $0$   \\
  $1$   & $0$  & $1$   & $0$   & $0$   & $1$  & $0$   & $1$   \\
  \hline 
  $16$  & $3$  & $13$  & $1$   & $6$   & $10$ & $4$   & $8$   \\
  \hline
  \end{tabular}
  \end{small}
  \caption{Orhogonal array with parameters $n=4$ and $k=2$.}
  \label{fig:orthogonal-array}
\end{figure}
\begin{figure}[!h]
  \centering
  \begin{small}
  \begin{tabular}[sdf]{|c|c|c|}
    \hline
    Inputs & Violating triples & $\SD$\\
    \hline
    $x_0,x_1,x_2$ &  No violating triples & $0$\\
    $x_0,x_1,x_3$ &  No violating triples & $0$\\
    $x_0,x_2,x_3$ &  $000\to 0.00, 001\to 0.25, 100\to 0.25$ & $\frac{3}{8}$\\
                 &  $101\to 0.00, 110\to 0.00, 111\to 0.25$ & \\
    $x_2,x_3,x_4$ &  No violating triples & $0$\\
    \hline
  \end{tabular}
  \end{small}
  \caption{Violating triples}
  \label{fig:violating-triples}
\end{figure}

As $\SD_x(\XXX_1,\XXX_1)=0$ and $\SD_x(\XXX_0,\XXX_1)=\frac{1}{2}$, by
taking $\XXX_3=\XXX_1$ we get the required counter-example for the
third question.  Finally, note that any statistical test is a
predicate. As a distinguisher with negated output works as well as the
original, we must bound the gate complexity of a predicate that is
satisfied by at most $8$ inputs. Each of this inputs can be
represented as conjunct consisting of three $\textsc{and}$- and at
most four $\textsc{not}$-gates. Hence, the total gate count is bounded
by $64$ gates, i.e., $\CD_x^{64}(\XXX_0,\XXX_1)=\SD_x(\XXX_0,\XXX_1)$
for all distributions $\XXX_0$ and $\XXX_1$.

\end{solution}
\end{document}
