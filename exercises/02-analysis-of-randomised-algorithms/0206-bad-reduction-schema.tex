\documentclass{crypto-exercise}
\usepackage{amsthm}
\author{Sven Laur}
\contributor{Sven Laur}
\editor{Sven Laur}
\tags{PRP/PRF switching lemma, bad-reduction, diverging games}


\begin{document}
\begin{exercise}{Analysis of bad events}
Prove that statistical distance between the games 
\begin{align*}
  \begin{game}{\GAME_0^\AD}
    & y_0\gets \bot\\
    &\begin{fblock}{\text{For $i\in\set{1,\ldots,q}$ do}}
      & x_i\gets\AD(y_{i-1})\\
      &\IF x_i=\bot\ \THEN \BREAK\\
      & y_i\getsu \MSPACE\\ 
      & \IF y_i\in\set{y_1,\ldots,y_{i-1}}\ \THEN\\
      &\!\!\begin{cblock}
      & \mathsf{do\ nothing}\\
      \end{cblock}\\
    \end{fblock}\\
    &\RETURN \AD
  \end{game}
  \begin{game}{\GAME_1^\AD}
    & y_0\gets \bot\\
    &\begin{fblock}{\text{For $i\in\set{1,\ldots,q}$ do}}
      & x_i\gets\AD(y_{i-1})\\
      &\IF x_i=\bot\ \THEN \BREAK\\
      & y_i\getsu \MSPACE\\
      & \IF y_i\in\set{y_1,\ldots,y_{i-1}}\ \THEN\\
      &\!\!\begin{cblock}
      &y_i\getsu\MSPACE\setminus\set{y_1,\ldots,y_{i-1}}\\         
      \end{cblock}\\
    \end{fblock}\\
    &\RETURN \AD
  \end{game}
\end{align*}
is the cumulative probability that one of the if branches gets
executed.
\end{exercise}


\begin{solution}
Let us first consider the simplified case where the
adversary $\AD$ makes exactly two queries. Then the corresponding
games can be simplified:
\begin{align*}
  &\begin{game}{\GAME_0^\AD}
    & x_1\gets\AD\\
    & y_1\getsu \MSPACE\\
    & x_2\gets\AD(y_1)\\
    & y_2\getsu \MSPACE\\
    & \IF y_2=y_1\ \THEN\\
    &\!\!\begin{cblock}
      & \mathsf{do\ nothing}\\
    \end{cblock}\\
    &\RETURN \AD(y_2)
  \end{game}
  &&
  &\begin{game}{\GAME_1^\AD}
    & x_1\gets\AD\\
    & y_1\getsu \MSPACE\\
    & x_2\gets\AD(y_1)\\
    & y_2\getsu \MSPACE\\
    & \IF y_2=y_1\ \THEN\\
    &\!\!\begin{cblock}
      &y_2\getsu\MSPACE\setminus\set{y_1}\\         
    \end{cblock}\\
    &\RETURN \AD(y_2)
  \end{game}
\end{align*}
We present two ways how to analyse this problem. First, we can do
direct computations of probabilities and derive
\begin{align*}
  \pr{\GAME_0^\AD=1}&=\pr{y_1=y_2\wedge \AD(y_2) = 1} + 
  \pr{y_1\neq y_2\wedge \AD(y_2) = 1}\\
  &\leq\pr{y_1=y_2} + \pr{y_1\neq y_2\wedge \AD(y_2) = 1}\enspace,\\
  \pr{\GAME_1^\AD=1}&=\pr{y_1=y_2\wedge \AD(y_2) = 1} +
   \pr{y_1\neq y_2\wedge \AD(y_2) = 1}\\
  &\geq \pr{y_1\neq y_2\wedge \AD(y_2) = 1}\enspace,\\
  \pr{\GAME_0^\AD=1}&=\pr{y_1=y_2\wedge \AD(y_2) = 1} +
   \pr{y_1\neq y_2\wedge \AD(y_2) = 1}\\
  &\geq  \pr{y_1\neq y_2\wedge \AD(y_2) = 1}\enspace,\\
  \pr{\GAME_1^\AD=1}&=\pr{y_1=y_2\wedge \AD(y_2) = 1} +
   \pr{y_1\neq y_2\wedge \AD(y_2) = 1}\\
  &\leq \pr{y_1=y_2}+ \pr{y_1\neq y_2\wedge \AD(y_2) = 1}\enspace.
\end{align*}
Note that probabilities $\pr{y_1=y_2}$ and $ \pr{y_1\neq y_2\wedge \AD(y_2) =
  1}$ are exactly the same in both games (a formal proof is lengthy
but straightforward) and thus combining all these inequalities yields
\begin{align*}
  -\pr{y_1=y_2}\leq\pr{\GAME_0^\AD=1}-\pr{\GAME_1^\AD=1}\leq\pr{y_1=y_2}\enspace. 
\end{align*}
As a second possible solution let us align use of randomness in both
games. To make it absolutely clear, let us consider the following
games:
\begin{align*}
  &\begin{game}{\GAME_0^\AD}
    & x_1\gets\AD\\
    & y_1\getsu \MSPACE\\
    & x_2\gets\AD(y_1)\\
    & y_2\getsu \MSPACE\\
    & \IF y_2=y_1\ \THEN\\
    &\!\!\begin{cblock}
      & z\getsu\MSPACE\setminus\set{y_1}\\         
    \end{cblock}\\
    &\RETURN \AD(y_2)
  \end{game}
  &&
  &\begin{game}{\GAME_1^\AD}
    & x_1\gets\AD\\
    & y_1\getsu \MSPACE\\
    & x_2\gets\AD(y_1)\\
    & y_2\getsu \MSPACE\\
    & \IF y_2=y_1\ \THEN\\
    &\!\!\begin{cblock}
      &y_2\getsu\MSPACE\setminus\set{y_1}\\         
    \end{cblock}\\
    &\RETURN \AD(y_2)
  \end{game}
\end{align*}
As both games now use the same set of randomness
$\omega\in\Omega_\AD\times\MSPACE\times \MSPACE\setminus\set{y_1}$
where $\Omega_\AD$ stands for the randomness space of $\AD$, we can
treat both games as deterministic functions $\GAME_0^\AD(\omega)$ and
$\GAME_1^\AD(\omega)$. By the construction
$\GAME_0^\AD(\omega)\neq\GAME_1^\AD(\omega)$ only if we reach the if
branch. Consequently,
\begin{align*}
  \set{\omega: \GAME_0^\AD(\omega)\neq\GAME_1^\AD(\omega)}\subseteq
  \set{\omega:y_1(\omega)\neq y_2(\omega)}
\end{align*}
and thus also by definition 
\begin{align*}
  \pr{\omega: \GAME_0^\AD(\omega)\neq\GAME_1^\AD(\omega)}\leq
  \pr{\omega:y_1(\omega)\neq y_2(\omega)}.
\end{align*}
Now note that 
\begin{align*}
  \abs{\pr{\GAME_0^\AD=1}-\pr{\GAME_1^\AD=1}}\leq  \pr{\omega: \GAME_0^\AD(\omega)\neq\GAME_1^\AD(\omega)}\enspace,
\end{align*}
since all other terms cancel out. The claim follows.

To give a general proof, lets modify the code so that the used
randomness is again aligned:
\begin{align*}
  \begin{game}{\GAME_0^\AD}
    & y_0\gets \bot\\
    &\begin{fblock}{\text{For $i\in\set{1,\ldots,q}$ do}}
      & x_i\gets\AD(y_{i-1})\\
      &\IF x_i=\bot\ \THEN \BREAK\\
      & y_i\getsu \MSPACE\\ 
      & \IF y_i\in\set{y_1,\ldots,y_{i-1}}\ \THEN\\
      &\!\!\begin{cblock}
      &z\getsu\MSPACE\setminus\set{y_1,\ldots,y_{i-1}}\\         
      \end{cblock}\\
    \end{fblock}\\
    &\RETURN \AD
  \end{game}
  \begin{game}{\GAME_1^\AD}
    & y_0\gets \bot\\
    &\begin{fblock}{\text{For $i\in\set{1,\ldots,q}$ do}}
      & x_i\gets\AD(y_{i-1})\\
      &\IF x_i=\bot\ \THEN \BREAK\\
      & y_i\getsu \MSPACE\\
      & \IF y_i\in\set{y_1,\ldots,y_{i-1}}\ \THEN\\
      &\!\!\begin{cblock}
      &y_i\getsu\MSPACE\setminus\set{y_1,\ldots,y_{i-1}}\\         
      \end{cblock}\\
    \end{fblock}\\
    &\RETURN \AD
  \end{game}
\end{align*}
Again by construction, the deterministic functions
$\GAME_0^\AD(\omega)$ and $\GAME_1^\AD(\omega)$ can diverge
$\GAME_0^\AD(\omega)\neq \GAME_1^\AD(\omega)$ on arguments $\omega$
for which some if branch is executed. Hence
\begin{align*}
    \abs{\pr{\GAME_0^\AD=1}-\pr{\GAME_1^\AD=1}}\leq 
    \pr{\omega:\text{some if branch is reached}}\enspace.
\end{align*}

\paragraph{Remark.} This divergence argument generalises for any game
pair with aligned randomness. As if branches containing different code
can be aligned to use the same randomness, the statistical distance of
games with different if branches is always bounded by the probability
of executing one of these branches. The corresponding code
transformation is known as the \textsc{Bad} reduction schema.

\end{solution}

\end{document}

