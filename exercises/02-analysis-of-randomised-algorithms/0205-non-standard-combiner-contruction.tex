\documentclass{crypto-exercise}
\author[Formalisation of folklore]{Sven Laur}
\contributor[Initial solution]{Filipp Ivanov}
\editor{Sven Laur}
\tags{Combiner construction, discrete logarithm, Jensen inequality}
\begin{document}
\begin{exercise}{Analysis of combiner constructions}
Let $\GG$ be a finite $q$-element group such that all elements $y\in\GG$ can be expressed as powers of $g\in\GG$. Let $\AD_1$ be a solver that finds the first bit of the discrete logarithm with probability $\varepsilon_1$, i.e., $\pr{x\gets \ZZ_{q}: \AD_1(g^x)=x_1}\geq\varepsilon_1$. Similarly, let  $\AD_2$ be a solver that finds the second bit of the discrete logarithm with probability $\varepsilon_1$ and so on. Now let $\ADB$ be the combiner algorithm that combines the outputs of $\AD_1,\ldots,\AD_n$ for $n=\ceil{\log_2 q}$ to restore the entire discrete logarithm:   
\begin{align*}
    \begin{fblock}{\ADB(y)}
    & x_1\gets\AD_1(y), \ldots, x_n\gets\AD_n(y)\\
    & \RETURN x_n\ldots x_1  
    \end{fblock}     
\end{align*}%
The construction guarantees that $\ADB$ succeeds in finding the discrete logarithm of $y$ if all $x_i$ are correct. Find a good lower bound on the advantage $\ADVDL{\GG}{\ADB}=\pr{x\gets \ZZ_{q}: \ADB(g^x)=x}$.
\end{exercise}
  
\begin{solution} 

\end{solution}
\end{document}
