\documentclass{crypto-exercise}
\author[Formalisation of folklore]{Sven Laur}
\editor{Sven Laur}
\tags{expected running time, strict running time, Markov inequality}

\DeclareMathOperator*{\EXPO} {\mathbf{E}}
\begin{document}


\begin{exercise}{From expected to strict running time}
Let $\AD$ be an algorithm that always provides a solution to a puzzle and is guaranteed to have expected running-time $\tau$ and we need to construct a $t$-time algorithm that fails with low probability. One way to solve this is to stop the original algorithm $\AD$ after $s$ time steps. Let $\AD_s$ be the corresponding algorithm which returns $\bot$ if $\AD$ does not stop in $s$ time steps and  whatever $\AD$ returns otherwise. Let $\ADB$ be the algorithm that runs $\AD_s$ up to $\floor{t/s}$ times to get the correct answer.
Use Markov inequality to estimate the failure probability of $\ADB$.  What is the minimal failure probability $\delta$ for fixed time-bound $t$? What is the minimal time-bound $t$ to achieve failure probability $\delta$. Graph the region of feasible solutions on $t\delta$-plane. \\ 
\end{exercise}

\begin{solution}
W.l.o.g. We can assume that the algorithm $\AD$  realises the Markov bound. If not we can modify the algorithm $\AD$ such way that if it succeeds earlier it does empty computations until the time-bound $s$ is reached and only then returns the answer.  


\end{solution}



\end{document}
