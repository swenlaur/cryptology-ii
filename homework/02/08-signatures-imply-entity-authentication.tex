\documentclass{crypto-exercise}
\usepackage{amsthm}
\author{Sven Laur}
\editor{Sven Laur}
\tags{entity authentication, signature scheme}


\begin{document}
\begin{exercise}{Signatures $\Rightarrow$ Entity authentication}
Let $(\GEN,\SIGN,\VERIFY)$ be a signature scheme that is $(t,\varepsilon)$-secure against universal one-more signature attack where the message distribution is uniform distribution over the message space $\MMM$.  
Prove that the entity authentication protocol where the verifier $\VVV$ chooses $m\getsu \MMM$ and the prover sends back the signature $s\gets \SIGN_\SK(m)$ is secure in the most powerful setting where the adversary can run several identification protocols concurrently in order to impersonate true signer.
\end{exercise}
\begin{solution}

\end{solution}
\end{document}