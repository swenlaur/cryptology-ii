\documentclass{crypto-exercise}
\usepackage{amsthm}
\author{Sven Laur}
\editor{Sven Laur}
\tags{entity authentication, signature scheme, proof of posession}


\begin{document}
\begin{exercise}{Signatures $\Rightarrow$ Entity authentication}
Let $(\GEN,\SIGN,\VERIFY)$ be a signature scheme that is $(t,\varepsilon)$-secure against universal one-more signature attack where the message distribution is uniform distribution over the message space $\MMM$.  
Prove that the entity authentication protocol where the verifier $\VVV$ chooses $m\getsu \MMM$ and the prover sends back the signature $s\gets \SIGN_\SK(m)$ there can be no black-box knowledge extractors for the secret key that is also efficient.
\end{exercise}
\begin{solution}
Let $\KKK^{\PPP_*}$ be a black-box knowledge extractor algorithm that succeeds in time $t_2$ and with probability $\varepsilon_2$ for all provers $\PPP_*$ that run in time $t_1$ and are at least $\varepsilon_1$ successful. Then we can construct an adversary $\ADB$ can conduct successful one-more signature attacks....  

\end{solution}
\end{document}