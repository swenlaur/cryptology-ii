\documentclass{crypto-exercise}
\usepackage{amsthm}
\usepackage{float}
\author{Sven Laur}
\contributor{Sven Laur}
\tags{ind-cpa, adversary restriction}

\begin{document}
\begin{exercise}{Trivial restriction to IND-CPA adversary}
Prove that that any adversary $\AD$ against IND-CPA games
\begin{align*}
&\begin{game}{\GAME_0}
 & (\SK,\PK)\gets\GEN\\
 & (m_0,m_1)\gets\AD(\PK)\\
 & c\gets\ENC_\PK(m_0)\\
 & \RETURN \AD(c)  
\end{game}
&
&\begin{game}{\GAME_1}
 & (\SK,\PK)\gets\GEN\\
 & (m_0,m_1)\gets\AD(\PK)\\
 & c\gets\ENC_\PK(m_1)\\
 & \RETURN \AD(c)  
\end{game}
\end{align*}
can be converted to a new adversary $\ADB$ against IND-CPA games that always outputs two different challenge messages $m_0\neq m_1$ so that the advantage remains the same and the computational overhead is constant.
\end{exercise}
\begin{solution}

\end{solution}
\textbf{Hint:} Decompose probability wrt condition $m_0=m_1$

\end{document}
