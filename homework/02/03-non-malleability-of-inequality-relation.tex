\documentclass{crypto-exercise}
\usepackage{amsthm}
\usepackage{float}
\author{Sven Laur}
\contributor{Sven Laur}
\tags{fixed nm-cpa, fixed relation}

\begin{document}
\begin{exercise}{NM-CPA security for inequality relation}
Explain why IND-CPA adversary $\AD$ can be converted to the adversary $\ADB$ against non-malleability game for inequality relation 
\begin{align*}
&\begin{game}{\BGAME_0}
 & (\SK,\PK)\gets\GEN\\
 & \MMM_0\gets\ADB(\PK)\\
 & m\gets\MMM_0\\
 & c\gets\ENC_\PK(m)\\
 & \hat{c}\gets\ADB(c)\\
 & \IF c=\hat{c}\ \THEN \RETURN 0\\ 
 & \RETURN m\neq\DEC_\SK(\hat{c}) 
\end{game}
&
&\begin{game}{\BGAME_1}
 & (\SK,\PK)\gets\GEN\\
 & \MMM_0\gets\ADB(\PK)\\
 & m, \overline{m}\gets\MMM_0\\
 & \overline{c}\gets\ENC_\PK(\overline{m})\\
 & \hat{c}\gets\ADB(\overline{c})\\
 & \IF c=\hat{c}\ \THEN \RETURN 0\\ 
 & \RETURN m\neq\DEC_\SK(\hat{c}) 
\end{game}
\end{align*}
How does the analysis change if we consider equality relation
\end{exercise}
\begin{solution}
\textbf{Hint:} What would be the best option to win the game if $\AD$ is a perfect adversary against IND-CPA games?

\end{solution}
\end{document}
