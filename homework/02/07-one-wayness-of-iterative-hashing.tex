\documentclass{crypto-exercise}
\usepackage{amsthm}
\author{Sven Laur}
\tags{partial-onewayness, iterative hashing}


\begin{document}


\begin{exercise}{Onewayness is closed under iterative hashing}
Let $h:\SSS\times\SSS\to\SSS$ be $(t,\varepsilon)$-oneway function. Show that function families defined by the construction
\begin{align*}
g_2(m_1, m_2) &= h(m_1,m_2)\enspace,\\
g_3(m_1, m_2, m_3)&=h(g_2(m_1, m_2), m_3)\enspace,\\
&\ldots
\end{align*}
are also one-way functions. Explain how is this result can be generalised to Merkle trees.

\end{exercise}
\begin{solution} 
\

\vspace*{2ex}
\noindent
\textsc{Simplified problem.}
Let us prove the onewayness of $g_2$. Let there be a collision, i.e., \ldots
Then \ldots 


\vspace*{2ex}
\noindent
\textsc{General solution.}
The analysis done above is suitable for any $i$. Indeed, let $g_{i-1}$ be \ldots. 
 \ldots


\vspace*{2ex}
\noindent
\textsc{Qualitative analysis.} Note that the success bound grows \ldots 

\end{solution}
\end{document}
