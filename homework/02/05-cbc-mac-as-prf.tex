\documentclass{crypto-exercise}
\usepackage{amsthm}
\author{Sven Laur}
\contributor[Initial solution]{Alisa Pankova}
\contributor[Complete rewrite]{Sven Laur}
\editor{Sven Laur}
\tags{CBC mode, pseudorandom function}

\newcommand{\ADVMAC}[2]{\ADV^{\mathsf{mac}}_{#1}(#2)}
\newcommand{\ADVCR}[2]{\ADV^{\mathsf{cr}}_{#1}(#2)}

\newcommand{\VER}{\mathsf{Ver}}
\newcommand{\HASH}{\mathsf{Hash}}
\newcommand{\MACH}{\mathsf{HashMac}}


\begin{document}
\begin{exercise}{Fixed domain CBC is PRF}
Let $\MMM$ is an Abelian group and let $\FFF_{all}$ be a family of all functions of type $f:\MMM\to\MMM$.
Show that functions 
\begin{align*}
g_1(m_1) &= f(m_1)\enspace,\\
g_2(m_1,m_2)&=f(g_1(m_1)+m_2)\enspace,\\
g_3(m_1,m_2,m_3)&=f(g_2(m_1,m_2)+m_3)\enspace, \\
&\ldots
\end{align*}
are pseudorandom functions. Explain why these functions are easily distinguishable from random if you can query two functions simultaneously, i.e., evaluate CBC construction on different input sizes.
\end{exercise}

\begin{solution}
Recall that $\FFF$ is $(t,q,\varepsilon)$-pseudorandom function family if any $t$-time adversary $\AD$ that makes at most $q$ oracle queries finds \ldots.

\vspace*{2ex}
\noindent
\textsc{Simplified problem.}
Let us prove the pseudorandomness of $g_2$. \ldots. 
For clarity let $(x_1,y_1)\ldots, (x_q, y_q)$ be the queries to the oracle $g_2(\cdot,\cdot)$.
Let $z_i=f(x_i)$ and $w_i=z_i+y_i$.
Now for a moment assume that all $x_i$ are different. 
Then by the definition of $\FFF_{all}$, we get \ldots  and thus we can replace $g_2$ with a random function\ldots


\vspace*{2ex}
\noindent
\textsc{General solution.}
The analysis done above is suitable for any $i$. Indeed, let $g_{i-1}$ be \ldots. 
 \ldots


\vspace*{2ex}
\noindent
\textsc{Qualitative analysis.} Note that the success bound grows \ldots 








\end{solution}
\end{document}  
